
\theoremstyle{definition}
\newcommand{\vect}[1]{\boldsymbol{#1}}
\newtheorem{proposition}{Proposition}[section]
\newtheorem{example}{Example}[section]
\newtheorem{definition}{Definition}[section]
\newtheorem{theorem}{Theorem}[section]

\newcommand{\KNOSOS}
{{\texttt{KNOSOS}}}
\newcommand{\proposicion}[1]
{
	\begin{proposition}
		\normalfont #1
	\end{proposition}
}
\newcommand{\diag}[1]{
	\text{diag}{\left({#1}\right)}
}

\DeclareMathOperator{\sgn}{sgn}
\DeclareMathOperator{\supp}{supp}
\renewcommand{\vec}{\mbox{vec}}
\newcommand{\Mod}[1]{\ \mathrm{mod}\ #1}

\DeclarePairedDelimiter\floor{\lfloor}{\rfloor}

\renewcommand{\i}{\mathrm{i}}

\newcommand{\ejemplo}[1]
{
	\begin{example}
		\normalfont #1
	\end{example}
}



\newtheorem{defn}{Definition}[section]% 
\newcommand{\definicion}[1]
{
	\begin{defn}
		\normalfont #1
	\end{defn}
}


\newcommand{\Lorentz}
{
	\mathcal{L}
}

\newcommand{\Vp}
{
	\mathcal{V}_{\parallel}
}

\newcommand{\Ad}
{
	\mathcal{A}_{d}
}



\newcommand{\Laguerre}[2]
{
	{L}^{(#1)}_{#2}
}


\newcommand{\Eqn}[2]
{
	\begin{align}
		#1
		\label{#2}
	\end{align}
}

\newcommand{\Set}[2]
{
	\left\lbrace #1  \,\middle|\, #2 \right\rbrace
}

\newcommand{\scalar}[2]{
	\langle#1,#2\rangle
}

\newcommand{\R}[1]
{ 
	\mathbb{R}^{#1}
}
\newcommand{\C}[1]
{ 
\mathbb{C}^{#1}
}

\newcommand{\Q}[1]
{ 
	\mathbb{Q}^{#1}
}

\newcommand{\M}[2]
{
	\mathcal{M}_{#1 \times #2}
}

\newcommand{\application}[3]
{
	#1 : #2 \longrightarrow #3
}

\newcommand{\applicationr}[5]
{
	\begin{tikzcd}[row sep = 0ex, ampersand replacement=\&]
		#1: #2  \arrow[r]         \& #3 \\
		#4  \arrow[r, mapsto] \& #5  
	\end{tikzcd}
}
%&{#2}\longrightarrow {#3}\\
%&{#4}\longmapsto {#5}
%\underset{#4}{#2}  \underset{\longmapsto }{ \longrightarrow }  \underset{#5}{#3}




\newcommand{\vp}{v_\parallel}

%\newcommand{\C}[1]
%{ 
%	\mathbb{C}^{#1}
%}

\newcommand{\N}[1]
{ 
	\mathbb{N}^{#1}
}

\newcommand{\Z}[1]
{ 
	\mathbb{Z}^{#1}
}

\newcommand{\K}[1]
{ 
	\mathbb{K}^{#1}
}

\newcommand{\BB}
{ 
	\vect{B}
}
\newcommand{\bb}
{ 
	\vect{b}
}

\renewcommand{\P}
{ 
	\mathcal{P}
}

\newcommand{\T}
{ 
	\mathcal{T}
}

\renewcommand{\L}
{ 
	\mathcal{L}
}


\DeclarePairedDelimiterX{\mean}[1]{\langle}{\rangle}{#1}

\newcommand{\Eval}[1]
{
	#1|
}
\newcommand{\Frac}[2]
{
	\left.{#1} \right/ {#2}
}
\newcommand{\dotprod}[2]
{
	\mean*{#1,#2}
}

\newcommand{\Matrix}[2]
{
	\mleft[
	\begin{array}{#1}
		#2
	\end{array}
	\mright]
}

\newcommand{\Determinant}[2]
{
	\left|
	\begin{array}{#1}
		#2
	\end{array}
	\right|
}

\newcommand\ddfrac[2]{\frac{\displaystyle #1}{\displaystyle #2}}




% NUMBERING OF EQUATIONS DEPENDING ON SECTION: e.g: (2.3)
% (For articles)

\numberwithin{equation}{section}

%\newcommand{\mean}[1]
%{
%	\langle {#1} \rangle
%}