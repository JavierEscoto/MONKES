\section{Installation and required libraries}
In this section it is explained how to install {\MONKES}. 
\subsection{Fortran compiler}
{\MONKES} is written in Fortran language and thus requires having an installed Fortran compiler. This compiler must be compatible with LAPACK and NetCDF libraries. One possible compiler is the Intel Fortran compiler.

\subsection{NetCDF library}

In order to read \texttt{BOOZER\_XFORM} output files, {\MONKES} needs to use the NetCDF library. An example of the minimal required libraries for running {\MONKES} in a UNIX machine is displayed below. Modules 1) and 2) are the NetCDF library compatible with the Fortran compiler. Module 3) is the Fortran compiler version, in this case \texttt{intel17/17.0.8}.
\begin{verbatim}
	Currently Loaded Modules:
	1) netcdf-fortran-4.4.4-intel-17.0.8-4me7upi   
	2) netcdf-4.6.1-intel-17.0.8-i5cn5xw   
	3) intel17/17.0.8
	
\end{verbatim} 

\subsection{LAPACK library}
For using LAPACK library there are two options. One can use the static or the dynamic version of the library. In listing \ref{lst:LAPACK_linking}, the dynamic case is illustrated.
\listings{../../bin/Makefile}{Compiler selection}{LFLAGS}{
	\texttt{Makefile} \label{lst:LAPACK_linking}
}
 The variable \texttt{LFLAGS} includes the linking of LAPACK libraries. Specifically, those flags which include the variable \texttt{\$MKL\_HOME} (variable containing the location of the dynamic library) set the instructions for where to find the library. The flag \texttt{-mkl=parallel} allows for LAPACK multithreading functionalities. In listing \ref{lst:main_MONKES_compiling_instruction}, it is shown how the monkes executable is generated. 
 
 
 \listings{../../bin/Makefile}{main_monkes.x:}{f90comp}{
 	\texttt{Makefile} \label{lst:main_MONKES_compiling_instruction}
 }
 