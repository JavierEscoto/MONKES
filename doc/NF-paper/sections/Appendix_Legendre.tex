\section{Legendre modes of the drift-kinetic equation}\label{sec:Appendix_Legendre}

Legendre polynomials are the eigenfunctions of the Sturm-Liouville problem in the interval $\xi\in[-1,1]$ defined by the differential equation%
\begin{align}
	2\Lorentz P_k(\xi) = -k(k+1) P_k(\xi), \label{eq:Legendre_eigenvalues}
\end{align}
and regularity boundary conditions at $\xi = \pm 1 $
%
\begin{align}
	\eval{(1-\xi^2)\dv{P_k}{\xi}}_{\xi = \pm 1} = 0,
\end{align}
where $k\ge 0$ is an integer. 

As $\Lorentz$ has a discrete spectrum and is self-adjoint with respect to the inner product
%
\begin{align}
	\mean*{f,g}_\Lorentz := \int_{-1}^{1} fg \dd{\xi} ,
\end{align}
in the space of functions that satisfy the regularity condition, $\{P_k\}_{k=0}^{\infty}$ is an orthogonal basis satisfying $\mean*{P_j,P_k}_\Lorentz = 2 \delta_{jk}/(2k+1)$. Hence, these polynomials satisfy the three-term recurrence formula
%
\begin{align}
	(2k+1)\xi P_k(\xi) = (k+1)P_{k+1}(\xi) + k P_{k-1}(\xi),
	\label{eq:Legendre_Three_Term_Recurrence}
\end{align}
obtained by Gram-Schmidt orthogonalization. Starting from the initial values $P_0=1$ and $P_1=\xi$, the recurrence defines the rest of the Legendre polynomials. Additionally, they satisfy the differential identity
%
\begin{align}
	(1-\xi^2)\dv{P_k}{\xi} = k P_{k-1}(\xi) - k \xi P_k(\xi).
	\label{eq:Legendre_Differential_Recurrence}
\end{align}
%
Identities (\ref{eq:Legendre_Three_Term_Recurrence}) and (\ref{eq:Legendre_Differential_Recurrence}) are useful to represent tridiagonally the left-hand side of equation (\ref{eq:DKE}) when we use the expansion (\ref{eq:Legendre_expansion}). The $k-$th Legendre mode of the term $\xi\vb*{b}\cdot \nabla f $ is expressed in terms of the modes $f^{(k-1)}$ and $f^{(k+1)}$ using (\ref{eq:Legendre_Three_Term_Recurrence})
\begin{align}	
	\mean*{\xi \vb*{b}\cdot\nabla f, P_k}_\Lorentz
	=
	\frac{2}{2k+1}
	\left[
	\frac{k}{2k-1} 
	\right.
	& \vb*{b}\cdot\nabla f^{(k-1)} \nonumber
	  \\+
	\frac{k+1}{2k+3} 
	& 
	\left.\vb*{b}\cdot\nabla f^{(k+1)} 
	\right].
\end{align}
 Combining both (\ref{eq:Legendre_Three_Term_Recurrence}) and (\ref{eq:Legendre_Differential_Recurrence}) allows to express the $k-$th Legendre mode of the mirror term $\nabla\cdot \vb*{b}((1-\xi^2)/2)\pdv*{f}{\xi}$ in terms of the modes $f^{(k-1)}$ and $f^{(k+1)}$ as
 \begin{align} 	
 	& \mean*{ 
 		\frac{1}{2}(1-\xi^2)
 		\nabla\cdot\vb*{b}  
 		\pdv{f}{\xi}, P_k}_\Lorentz
 	=	\\
 	&\frac{\vb*{b}\cdot\nabla \ln B}{2k+1}
 	\left[
 	\frac{k(k-1)}{2k-1} 
% 	\right. &
 	f^{(k-1)} 
 	-
 	\frac{(k+1)(k+2)}{2k+3}
% 	& \left. 
 	f^{(k+1)} 
 	\right], \nonumber
 \end{align}
where we have also used $\nabla\cdot\vb*{b}  = - \vb*{b}\cdot \nabla \ln B$. The term proportional to $\hat{E}_\psi$ is diagonal in a Legendre representation
%
\begin{align}
	&
	\mean*{\frac{\hat{E}_\psi}{\mean*{B^2}}
		\vb*{B}\times \nabla\psi \cdot\nabla f , P_k}_\Lorentz
	=
	\\ 
	&
	\frac{2}{2k+1}
	\frac{\hat{E}_\psi}{\mean*{B^2}}
	\vb*{B}\times \nabla\psi \cdot\nabla f^{(k)}.
	\nonumber
\end{align}
For the collision operator used in equation (\ref{eq:DKE}), as Legendre polynomials are eigenfunctions of the pitch-angle scattering operator, using (\ref{eq:Legendre_eigenvalues}) we obtain the diagonal representation 
%
\begin{align}
	\mean*{\hat{\nu} \Lorentz f , P_k}_\Lorentz
	&
	=
	-\hat{\nu}
	\frac{k(k+1)}{2k+1}	
	f^{(k)}.
\end{align}

Finally, we briefly comment on why the truncation error from (\ref{eq:Legendre_expansion}) implies that the solution to (\ref{eq:DKE_Legendre_expansion}) and (\ref{eq:kernel_elimination_condition_Legendre}) is an approximation of the Legendre spectrum of the exact solution to (\ref{eq:DKE}) satisfying (\ref{eq:kernel_elimination_condition}). For this, we will assume that the solution to (\ref{eq:DKE}) and (\ref{eq:kernel_elimination_condition}) is unique (which it is, see \ref{sec:Appendix_Invertibility}). We denote this exact solution by $f_{\text{ex}}$ and its Legendre modes by $f^{(k)}_{\text{ex}}$. The Legendre modes $f^{(k)}_{\text{ex}}$ satisfy (\ref{eq:DKE_Legendre_expansion}) for all values of $k$, including $k>N_\xi$ and, in general, $f^{(N_\xi+1)}_{\text{ex}}\ne 0$. Denoting the error of the solution $f^{(k)}$ to (\ref{eq:DKE_Legendre_expansion}) and (\ref{eq:kernel_elimination_condition_Legendre}) by
%
\begin{align}
	E^{(k)} : = f^{(k)}_{\text{ex}} - f^{(k)} ,
\end{align}
is easy to prove that
%
\begin{align}
	L_k E^{(k-1)} + D_k E^{(k)} + U_k E^{(k+1)} = 0, 
	\label{eq:Error_DKE_Legendre_k}
\end{align}
for $k =0,1,\ldots, N_\xi-1$ and 
%
\begin{align}
	L_{N_\xi} E^{(N_\xi-1)} + D_{N_\xi} E^{(N_\xi)} = - U_{N_\xi} f^{(N_\xi+1)}_{\text{ex}}.
	\label{eq:Error_DKE_Legendre_Nxi}
\end{align}
Note that the system of equations constituted by (\ref{eq:Error_DKE_Legendre_k}) and (\ref{eq:Error_DKE_Legendre_Nxi}) for the error is identical to (\ref{eq:DKE_Legendre_expansion}) substituting $f^{(k)}$ by $E^{(k)}$ and $s^{(k)}$ by $- U_{N_\xi} f^{(N_\xi+1)}_{\text{ex}}$. Hence, by assumption, the solution to (\ref{eq:Error_DKE_Legendre_k}) and (\ref{eq:Error_DKE_Legendre_Nxi}) satisfying (\ref{eq:kernel_elimination_condition_Legendre}) is unique, implying that $E^{(k)}\ne 0$ unless $ {U}_{N_\xi} f^{(N_\xi+1)}_{\text{ex}} = 0$.