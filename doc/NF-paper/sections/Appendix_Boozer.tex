\section{Boozer coordinates} \label{sec:Appendix_Boozer}
For the spatial domain, in section \ref{sec:Algorithm}, we will employ right-handed Boozer coordinates $(\psi,\theta,\zeta)\in[0,\psi_{\text{lcfs}}]\times[0,2\pi)\times[0,2\pi/N_p)$. In these coordinates $2\pi \psi$ is the toroidal flux of the magnetic field and $\theta$, $\zeta$ are respectively the poloidal and toroidal (in a single period) angles. The integer $N_p\ge 1$ denotes the number of periods of the device. In Boozer coordinates the magnetic field can be written as
\begin{align}
	\vb*{B} & = \nabla\psi \times \nabla\theta - \iota(\psi) \nabla\psi \times \nabla\zeta 
	\nonumber\\
	& = B_\psi(\psi,\theta,\zeta) \nabla \psi + B_\theta(\psi) \nabla \theta + B_\zeta(\psi) \nabla \zeta,
\end{align}
and the Jacobian of the transformation reads 
%
\begin{align}
	\sqrt{g}(\psi,\theta,\zeta) 
	:=( 
	\nabla\psi \times \nabla \theta \cdot \nabla\zeta  
	)^{-1} 
	= 
	\frac{B_\zeta + \iota B_\theta}{B^2},
\end{align}
where we have denoted $B:=|\vb*{B}|$ and $\iota =\vb*{B} \cdot \nabla \theta / \vb*{B} \cdot \nabla \zeta $ is the rotational transform.


Here, the symbol $\mean*{...}$ stands for the flux-surface average operation, which in Boozer coordinates $(\theta,\zeta)$ takes the form
%
\begin{align}
	\mean*{F} = \frac{1}{V'(\psi)}\oint \oint \sqrt{g}(\psi,\theta,\zeta) F(\psi,\theta,\zeta) \dd{\theta}\dd{\zeta}
\end{align}
for any integrable function $F(\psi,\theta,\zeta)$ and $V'(\psi)$ is fixed from the condition $\mean*{1}=1$. Also, the spatial differential operators involved in equation (\ref{eq:DKE_Original}) in Boozer coordinates take the form
%
\begin{align}
	\vb*{b} \cdot \nabla & = 
	\frac{B}{B_\zeta + \iota B_\theta}
	\left(
	\iota \pdv{\theta}
	+ 
	\pdv{\zeta} 
	\right), %\label{eq:Parallel_streaming_spatial_operator}
	\\
	\vb*{B}\times\nabla\psi \cdot \nabla & = 
	\frac{B^2}{B_\zeta + \iota B_\theta}
	\left(
	B_\zeta \pdv{\theta}
	-
	B_\theta \pdv{\zeta}
	\right). %\label{eq:ExB_spatial_operator}
\end{align}





