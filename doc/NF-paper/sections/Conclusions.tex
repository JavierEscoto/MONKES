In this paper we have presented the new code {\MONKES}, which can provide fast and accurate calculations of the monoenergetic transport coefficients at low collisionality in a single core. By means of a thorough convergence study we have shown that it is possible to evaluate the monoenergetic coefficients in the $1/\nu$ and $\sqrt{\nu}$-$\nu$ regimes in approximately 1 minute. Besides, when there are sufficient computational resources available, the code can run even faster using several cores in parallel. A natural application is the inclusion of {\MONKES} in a stellarator optimization suite. {\MONKES} rapid calculations will allow direct optimization of the bootstrap current and radial transport from low ($1/\nu$ and $\sqrt{\nu}$-$\nu$) to moderate (plateau) collisionalities. The low collisionality regimes are important in reactor relevant scenarios while the plateau regime can be important close to the edge, where the plasma is cooler.  Massive evaluation of configurations to study the parametric dependence of $\widehat{D}_{31}$ or other coefficients on specific quantities of the magnetic configuration can also be done. Another application is its inclusion in predictive transport frameworks, which require neoclassical calculations to determine the evolution of plasma profiles. The neoclassical quantities required for these simulations can be calculated using {\MONKES}.


Equation (\ref{eq:DKE_Original}), solved by {\MONKES}, includes a collision operator which does not preserve momentum. An important continuation of this work would be the implementation of momentum-correction techniques, such as the ones presented in \cite{Taguchi,Sugama-PENTA,Sugama2008,MaasbergMomentumCorrection}. As each calculation from {\MONKES} can be executed in a single core, the scan in $v$ (i.e. in $\hat{\nu}$) required to perform the integrals of the monoenergetic coefficients is parallelizable. Therefore, it seems possible for the near future to obtain fast calculations of neoclassical transport with a model collision operator that preserves momentum. With this minor extension, {\MONKES} could also be used for self-consistent optimization of magnetic fields in a similar manner to \cite{Landreman_SelfConsistent} for general geometry. 
 

 





