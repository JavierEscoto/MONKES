Once we have chosen the resolutions $(N_\theta,N_\zeta,N_\xi)$ for each case, we need to verify that these selections indeed provide sufficiently accurate calculations of all the monoenergetic coefficients in the interval $\hat{\nu}\in[10^{-5},300]$ $\text{m}^{-1} $. In all cases, {\MONKES} calculations of the $\widehat{D}_{11}$ and $\widehat{D}_{31}$ coefficients will be benchmarked against converged calculations from {\DKES} (see \ref{sec:Appendix_DKES_Bounds}) and from \texttt{SFINCS}\footnote{\texttt{SFINCS} calculations are converged up to 3\% in the three independent variables.}. The parallel conductivity coefficient will be benchmarked only against {\DKES}. The benchmarking of the coefficient $\widehat{D}_{11}$ is shown in figure \ref{fig:D11_Benchmark}. For W7-X, perfect agreement with and without $\hat{E}_r$ is obtained. This is shown in figures \ref{subfig:D11_Benchmark_W7X_EIM} and \ref{subfig:D11_Benchmark_W7X_KJM}. The good agreement in the $\widehat{D}_{11}$ calculation between the three codes for the CIEMAT-QI cases is shown in figure \ref{subfig:D11_Benchmark_CIEMAT_QI}. 

Figures \ref{subfig:D31_Benchmark_W7X_EIM} and \ref{subfig:D31_Benchmark_W7X_KJM} show that {\MONKES} calculations of the bootstrap current coefficient for W7-X EIM and W7-X KJM are in total agreement with those of {\DKES} and \texttt{SFINCS}. For the CIEMAT-QI cases with and without $\hat{E}_r$, there is also very good agreement, as shown in figure \ref{subfig:D31_Benchmark_CIEMAT_QI}.
%

For the parallel conductivity coefficient, the agreement between the results of {\MONKES} and {\DKES} is very good. We can observe in figure \ref{fig:D33_Benchmark} that, for all the cases, is difficult to distinguish between the results of {\MONKES} and {\DKES}. Due to the weak effect of the radial electric field in the $\widehat{D}_{33}$ coefficient, the symbols for this plot have been changed. 
%



\begin{figure*}[h]	
	\captionsetup[sub]{skip=-1.75pt, margin=95pt}
	\centering
	\tikzexternaldisable
	\ref{legend}
	\tikzexternalenable
	\tikzsetnextfilename{Benchmark-W7X-EIM-s0200-D11}
	
	\begin{subfigure}[t]{0.32\textwidth}	
			\begin{tikzpicture}
		\begin{axis}[
			width=\textwidth, 
			xlabel=$\hat{\nu}$  ${[\text{m}^{-1}]}$, ylabel=$\widehat{D}_{11}$ ${[\text{m}]}$, 
			xmode=log, ymode=log, enlarge x limits=false,
			legend pos = north west, 
			mark size = 1.5pt
			]
						
			%%% MONKES RESULTS
			\addplot+[no markers, blue]
			table[skip first n=1, 
			x expr=\thisrowno{0},
			y expr=\thisrowno{2}*0.5237*0.5237,  
			restrict expr to domain={\thisrowno{1}}{0:0}]
			{./results/W7X-EIM/0.200/Monoenergetic_Database_Example_23_55_140/Monoenergetic_1.plt};	
%			\addlegendentry{\texttt{MONKES} $\hat{E}_r=0$}			
			
			
			%%% DKES RESULTS				
			\addplot[only marks, mark=*, blue, error bars/.cd, y explicit,
			y dir=both,
			error bar style={thick}] table[
			x expr=\thisrowno{0},
			y expr={ (\thisrowno{4} + \thisrowno{5})/2*0.5237*0.5237 }, 
			restrict expr to domain={\thisrowno{0}}{1e-5:3e2},
			restrict expr to domain={\thisrowno{1}}{0:0},
			y error plus expr=(\thisrowno{4}-\thisrowno{5})/2,
			y error minus expr=(\thisrowno{4}-\thisrowno{5})/2] {./results/W7X-EIM/0.200/results.data};
%			\addlegendentry{\texttt{DKES} $\hat{E}_r=0$}
			
			%%% SFINCS RESULTS
			\addplot[mark repeat = 2, only marks, blue, mark=triangle*]
			table[col sep = comma, skip first n=1, 
			x expr=\thisrowno{0},
			y expr=\thisrowno{1}*0.5237*0.5237
			]
			{./results/SFINCS_results/W7X-EIM/20230420-01-027-W7X-EIM_Er0_nuScan_sfincs_results.dat};	
			
			%%% MONKES RESULTS
			\addplot+[no markers, red]
			table[skip first n=1, 
			x expr=\thisrowno{0},
			y expr=\thisrowno{2}*0.5237*0.5237,  
			restrict expr to domain={\thisrowno{1}}{3e-4:3e-4}]
			{./results/W7X-EIM/0.200/Monoenergetic_Database_Example_27_55_140/Monoenergetic_5.plt};	
%			\addlegendentry{\texttt{MONKES} $\hat{E}_r\ne 0$}
						
			
			%%% DKES RESULTS				
			\addplot[only marks, mark=square*, red, error bars/.cd, y explicit,
			y dir=both,
			error bar style={thick}] table[
			x expr=\thisrowno{0},
			y expr={ (\thisrowno{4} + \thisrowno{5})/2*0.5237*0.5237 }, 
			restrict expr to domain={\thisrowno{0}}{1e-5:3e2},
			restrict expr to domain={\thisrowno{1}}{3e-4:3e-4},
			y error plus expr=(\thisrowno{4}-\thisrowno{5})/2,
			y error minus expr=(\thisrowno{4}-\thisrowno{5})/2] {./results/W7X-EIM/0.200/results.data};
%			\addlegendentry{\texttt{DKES} $\hat{E}_r\ne 0$}			
			
			%%% SFINCS RESULTS
			\addplot[mark repeat =2, only marks, red, mark=triangle*, mark options={rotate=60}]
			table[col sep = comma, skip first n=1, 
			x expr=\thisrowno{0},
			y expr=\thisrowno{1}*0.5237*0.5237
			]
			{./results/SFINCS_results/W7X-EIM/20230420-01-018-W7X-EIM_withEr_nuScan_sfincs_results.dat};	
		\end{axis}
		
	\end{tikzpicture}


		\caption{}
		\label{subfig:D11_Benchmark_W7X_EIM}
	\end{subfigure}
	%
	\tikzsetnextfilename{Benchmark-W7X-KJM-s0204-D11}
	\begin{subfigure}[t]{0.32\textwidth}		
		
	\begin{tikzpicture}
		\begin{axis}[
			width=\textwidth, 
			xlabel=$\hat{\nu}$  ${[\text{m}^{-1}]}$, ylabel=$\widehat{D}_{11} $ ${[\text{m}]}$, 
			xmode=log, ymode=log, enlarge x limits=false,
			legend pos = north west, 
			mark size = 1.5pt, 
			xtick = {1e-5, 1e-3, 1e-1, 1e1},
			ymax = 10
%			legend style={at={(axis cs:1e-4,14)},anchor=north west}
			]
			
			%%% MONKES RESULTS
			\addplot+[no markers, blue]
			table[skip first n=1, 
			x expr=\thisrowno{0},
			y expr=\thisrowno{2}*0.5132*0.5132,  
			restrict expr to domain={\thisrowno{1}}{0:0}]
			{./results/W7X-KJM/0.204/Monoenergetic_Database_Example_23_63_140/Monoenergetic_1.plt};	
%			\addlegendentry{\texttt{MONKES} $\hat{E}_r=0$}
						
			%%% DKES RESULTS				
			\addplot[only marks, blue, mark=*, error bars/.cd, y explicit,
			y dir=both,
			error bar style={thick}] table[
			x expr=\thisrowno{0},
			y expr={ (\thisrowno{4} + \thisrowno{5})/2*0.5132*0.5132 }, 
			restrict expr to domain={\thisrowno{0}}{1e-5:3e2},
			restrict expr to domain={\thisrowno{1}}{0:0},
			y error plus expr=(\thisrowno{4}-\thisrowno{5})/2,
			y error minus expr=(\thisrowno{4}-\thisrowno{5})/2] {./results/W7X-KJM/0.204/results.data};
%			\addlegendentry{\texttt{DKES} $\hat{E}_r=0$}
									
			%%% SFINCS RESULTS				
			\addplot[mark repeat =2,only marks, blue, mark=triangle*] 
			table[skip first n=1,
			x expr=\thisrowno{0},
			y expr=\thisrowno{1}*0.5132*0.5132
			] {./results/SFINCS_results/W7X-KJM/20230420-01-020-W7X-KJM_Er0_nuScan_sfincs_results.dat};
%			\addlegendentry{\texttt{SFINCS} $\hat{E}_r = 0$}
			
			%%% MONKES RESULTS
			\addplot+[no markers, red]
			table[skip first n=1, 
			x expr=\thisrowno{0},
			y expr=\thisrowno{5}*0.5132*0.5132,  
			restrict expr to domain={\thisrowno{0}}{1e-5:3e+2},  
			restrict expr to domain={\thisrowno{1}}{3e-4:3e-4}
			]
			{./results/W7X-KJM/0.204/Monoenergetic_19_79_180/monkes_Monoenergetic_Database.dat};
%			\addlegendentry{\texttt{MONKES} $\hat{E}_r\ne 0$}
			
			
			
			%%% DKES RESULTS				
			\addplot[only marks, mark=square*, red, error bars/.cd, y explicit,
			y dir=both,
			error bar style={thick}] table[
			x expr=\thisrowno{0},
			y expr={ (\thisrowno{4} + \thisrowno{5})/2*0.5132*0.5132 }, 
			restrict expr to domain={\thisrowno{0}}{1e-5:3e2},
			restrict expr to domain={\thisrowno{1}}{3e-4:3e-4},
			y error plus expr=(\thisrowno{4}-\thisrowno{5})/2,
			y error minus expr=(\thisrowno{4}-\thisrowno{5})/2] {./results/W7X-KJM/0.204/results.data};
%			\addlegendentry{\texttt{DKES} $\hat{E}_r\ne 0$}			
							
			%%% SFINCS RESULTS				
			\addplot[mark repeat =2, only marks, red, mark=diamond*] 
			table[skip first n=1,
			x expr=\thisrowno{0},
			y expr=\thisrowno{1}*0.5132*0.5132
			] {./results/SFINCS_results/W7X-KJM/20230420-01-017-W7X-KJM_withEr_nuScan_sfincs_results.dat};
%			\addlegendentry{\texttt{SFINCS} $\hat{E}_r =  3 \cdot 10^{-4}$}
			
		\end{axis}
		
	\end{tikzpicture}


		\caption{}
		\label{subfig:D11_Benchmark_W7X_KJM}
	\end{subfigure}
	%
	\tikzsetnextfilename{Benchmark-CIEMAT-QI-s0250-D11}
	\begin{subfigure}[t]{0.32\textwidth}		
%		\tikzexternaldisable
		
	\begin{tikzpicture}
		\begin{axis}[
			width=\textwidth,  
			xlabel=$\hat{\nu}$  ${[\text{m}^{-1}]}$, ylabel=$\widehat{D}_{11}$ ${[\text{m}]}$, 
			xmode=log, 
			ymode=log, 
			enlarge x limits=false,
%			xmax = 0.1, 
%			ymax = 4,
			legend pos = north west,
			legend columns=6, 
			legend style={/tikz/every even column/.append style={column sep=1.1cm}, cells={align=left}},
			legend to name={legend},
			mark size = 1.5pt
			]
			
			%%% MONKES RESULTS
			\addplot[ no markers, blue]
			table[skip first n=1, 
			x expr=\thisrowno{0},
			y expr=\thisrowno{2}*0.4674*0.4674]
			{./results/CIEMAT-QI/0.250/Monoenergetic_Database_Example_15_119_180/Monoenergetic_1.plt};	
			\addlegendentry{\texttt{MONKES} \\ $\widehat{E}_r=0$}			
						
			%%% DKES RESULTS	
			\foreach \nu in 
			{1e-5,1e-4,3e-4,3e-3,1e-2,3e-2,1e-1,3e-1,1e-0,3e-0,1e+1,3e+1,1e+2,3e+2}
			{
				\addplot[forget plot, only marks, mark=*, blue] table[skip first n=2, 
				x expr=\thisrowno{0},
				y expr={ (\thisrowno{4} + \thisrowno{5})/2*0.4674*0.4674 }, 
				restrict expr to domain={\thisrowno{1}}{0:0}] {./results/CIEMAT-QI/0.250/omega_0e-0/cl_\nu/results.stellopt};
			} 			
			
			%%% DKES RESULTS				
			\addplot[ only marks, mark=*, blue, error bars/.cd, y explicit,
			y dir=both,
			error bar style={thick}] table[
			x expr=\thisrowno{0},
			y expr={ (\thisrowno{4} + \thisrowno{5})/2*0.4674*0.4674 }, 
			restrict expr to domain={\thisrowno{0}}{1e-5:3e2},
			restrict expr to domain={\thisrowno{1}}{0:0},
			y error plus expr=(\thisrowno{6}-\thisrowno{6})/2,
			y error minus expr=(\thisrowno{6}-\thisrowno{6})/2] {./results/CIEMAT-QI/0.250/results.data};
			\addlegendentry{\texttt{DKES} \\ $\widehat{E}_r=0$}
			
			%%% DKES RESULTS	
			\foreach \nu in 
			{1e-5,1e-4,3e-4,3e-3,1e-2,3e-2,1e-1,3e-1,1e-0,3e-0,1e+1,3e+1,1e+2,3e+2}
			{
				\addplot[forget plot, only marks, mark=square*, red] table[skip first n=2, 
				x expr=\thisrowno{0},
				y expr={ (\thisrowno{4} + \thisrowno{5})/2*0.4674*0.4674 }, 
				restrict expr to domain={\thisrowno{1}}{1e-3:1e-3}] {./results/CIEMAT-QI/0.250/omega_1e-3/cl_\nu/results.stellopt};
			}
			%%% SFINCS RESULTS
			\addplot[mark repeat = 2, only marks, blue, mark=triangle*]
			table[col sep = comma, skip first n=1, 
			x expr=\thisrowno{0},
			y expr=\thisrowno{1}*0.4674*0.4674
			]
			{./results/SFINCS_results/CIEMAT-QI/20230420-01-058-CIEMAT-QI_s0p238_Er0_nuScan_sfincs_results.dat};	
			\addlegendentry{\texttt{SFINCS}\\ $\widehat{E}_r=0$}	
			
			%%% MONKES RESULTS
			\addplot[no markers, red]
			table[skip first n=1, 
			x expr=\thisrowno{0},
			y expr=\thisrowno{2}*0.4674*0.4674,  
			restrict expr to domain={\thisrowno{1}}{1e-3:1e-3}]
			{./results/CIEMAT-QI/0.250/Monoenergetic_Database_Example_15_119_180/Monoenergetic_6.plt};	
			\addlegendentry{\texttt{MONKES}\\ $\widehat{E}_r\ne 0$}			
			
			%%% DKES RESULTS				
			\addplot[only marks, mark=square*, red, error bars/.cd, y explicit,
			y dir=both,
			error bar style={thick}] table[
			x expr=\thisrowno{0},
			y expr={ (\thisrowno{4} + \thisrowno{5})/2*0.4674*0.4674 }, 
			restrict expr to domain={\thisrowno{0}}{1e-4:3e2},
			restrict expr to domain={\thisrowno{1}}{1e-3:1e-3},
			y error plus expr=(\thisrowno{6}-\thisrowno{6})/2,
			y error minus expr=(\thisrowno{6}-\thisrowno{6})/2] {./results/CIEMAT-QI/0.250/results.data};
			\addlegendentry{\texttt{DKES} \\ $\widehat{E}_r\ne 0$}				
			
			%%% SFINCS RESULTS
			\addplot[mark repeat =2, only marks, red, mark=triangle*, mark options={rotate=60}]
			table[col sep = comma, skip first n=1, 
			x expr=\thisrowno{0},
			y expr=\thisrowno{1}*0.4674*0.4674
			]
			{./results/SFINCS_results/CIEMAT-QI/20230420-01-047-CIEMAT-QI_s0p238_withEr_nuScan_sfincs_results.dat};	
			\addlegendentry{\texttt{SFINCS}\\ $\widehat{E}_r\ne 0$}			
			
		\end{axis}
		
	\end{tikzpicture}


%		\tikzexternalenable
		\caption{}
		\label{subfig:D11_Benchmark_CIEMAT_QI}
	\end{subfigure}
	\caption{Calculation of $\widehat{D}_{11}$ by \texttt{MONKES}, \texttt{DKES} and \texttt{SFINCS} for zero and finite $\hat{E}_r$. (a) W7-X EIM at the surface $\psi /\psi_{\text{lcfs}}=0.200$. (b) W7-X KJM at the surface $\psi /\psi_{\text{lcfs}}=0.204$. (c) CIEMAT-QI at the surface $\psi /\psi_{\text{lcfs}}=0.250$. }
	\label{fig:D11_Benchmark}
\end{figure*}
\begin{figure*}[h]
	\centering
	\captionsetup[sub]{skip=-1.75pt, margin=80pt}
	\tikzexternaldisable
	\ref{legend}
	\tikzexternalenable
    \tikzsetnextfilename{Benchmark-W7X-EIM-s0200-D31}
    \begin{subfigure}[t]{0.32\textwidth}	
        	\begin{tikzpicture}
		\begin{axis}[
			width=\textwidth, 
			xlabel=$\hat{\nu}$  ${[\text{m}^{-1}]}$, ylabel=$\widehat{D}_{31}$ ${[\text{m}]}$, 
			xmode=log, enlarge x limits=false,
			scaled y ticks=base 10:1,
			y tick label style={
				/pgf/number format/.cd,
				fixed,
				fixed zerofill,
				precision=0,
				/tikz/.cd}, 
			legend pos = north east,
			mark size = 1.5pt
			]
			
			%%% MONKES RESULTS
			\addplot+[no markers, blue]
			table[skip first n=1, 
			x expr=\thisrowno{0},
			y expr=\thisrowno{3}*0.5237						
			]
			{data/W7X-EIM/MONKES/Monoenergetic_Database_Example_23_55_140/Monoenergetic_1.plt};				
			\addlegendentry{{\MONKES} $\widehat{E}_r=0$}
			
			%%% DKES RESULTS				
			\addplot[only marks, mark=*, blue, error bars/.cd, y explicit,
			y dir=both,
			error bar style={thick}] table[
			x expr=\thisrowno{0},
			y expr={ (\thisrowno{6} + \thisrowno{7})/2*0.5237 }, 
			restrict expr to domain={\thisrowno{0}}{1e-5:3e2},
			restrict expr to domain={\thisrowno{1}}{0:0},
			y error plus expr=(\thisrowno{6}-\thisrowno{6})/2,
			y error minus expr=(\thisrowno{6}-\thisrowno{6})/2] {data/W7X-EIM/DKES/results.data};
			\addlegendentry{{\DKES} $\widehat{E}_r=0$}
			
			%%% SFINCS RESULTS
			\addplot[mark repeat =2,only marks, blue, mark=triangle*]
			table[col sep = comma, skip first n=1, 
			x expr=\thisrowno{0},
			y expr=\thisrowno{2}*0.5237
			]
			{data/W7X-EIM/SFINCS/20230420-01-027-W7X-EIM_Er0_nuScan_sfincs_results.dat};	

			
			
			%%% MONKES RESULTS
			\addplot+[no markers, red]
			table[skip first n=1, 
			x expr=\thisrowno{0},
			y expr=\thisrowno{6}*0.5237,  
			restrict expr to domain={\thisrowno{1}}{3e-4:3e-4}]
			{data/W7X-EIM/MONKES/Convergence_nu_1e-5_Er_3e-4/monkes_Monoenergetic_Database.dat};	
%			\addlegendentry{\texttt{MONKES} $\sqrt{\nu}$}
			
			
			%%% DKES RESULTS				
			\addplot[only marks, mark=square*, red, error bars/.cd, y explicit,
			y dir=both,
			error bar style={thick}] table[
			x expr=\thisrowno{0},
			y expr={ (\thisrowno{6} + \thisrowno{7})/2*0.5237}, 
			restrict expr to domain={\thisrowno{0}}{1e-5:3e2},
			restrict expr to domain={\thisrowno{1}}{3e-4:3e-4},
			y error plus expr=(\thisrowno{6}-\thisrowno{6})/2,
			y error minus expr=(\thisrowno{6}-\thisrowno{6})/2] {data/W7X-EIM/DKES/results.data};
%			\addlegendentry{\texttt{DKES} $\sqrt{\nu}$}			
			
			%%% SFINCS RESULTS
			\addplot[mark repeat =2, only marks, red, mark=triangle*, mark options={rotate=60}]
			table[col sep = comma, skip first n=1, 
			x expr=\thisrowno{0},
			y expr=\thisrowno{2}*0.5237
			]
			{data/W7X-EIM/SFINCS/20230420-01-018-W7X-EIM_withEr_nuScan_sfincs_results.dat};	
		\end{axis}
		
	\end{tikzpicture}


        \caption{}
        \label{subfig:D31_Benchmark_W7X_EIM}
    \end{subfigure}
    %
    \tikzsetnextfilename{Benchmark-W7X-KJM-s0204-D31}
    \begin{subfigure}[t]{0.32\textwidth}		
    	
	\begin{tikzpicture}
		\begin{axis}[
			width=\textwidth, 
			xlabel=$\hat{\nu}$  ${[\text{m}^{-1} ]}$, ylabel=$\widehat{D}_{31}$ ${[\text{m}]}$, 
			xmode=log, enlarge x limits=false,
			legend pos = north east,
			scaled y ticks=base 10:1,
			y tick label style={
				/pgf/number format/.cd,
				fixed,
				fixed zerofill,
				precision=0,
				/tikz/.cd}, 
			xtick = {1e-5, 1e-3, 1e-1, 1e1},
			mark size = 1.5pt
			]
			
			%%% MONKES RESULTS
			\addplot+[forget plot, no markers, blue]
			table[skip first n=1, 
			x expr=\thisrowno{0},
			y expr=\thisrowno{3} *0.5132,  
			restrict expr to domain={\thisrowno{1}}{0:0}]
			{data/W7X-KJM/MONKES/Monoenergetic_Database_Example_23_63_140/Monoenergetic_1.plt};	
			
			%%% DKES RESULTS				
			\addplot[forget plot, only marks, blue, mark=*, error bars/.cd, y explicit,
			y dir=both,
			error bar style={thick}] table[
			x expr=\thisrowno{0},
			y expr={ (\thisrowno{6} + \thisrowno{7})/2  *0.5132 }, 
			restrict expr to domain={\thisrowno{0}}{1e-5:3e2},
			restrict expr to domain={\thisrowno{1}}{0:0},
			y error plus expr=(\thisrowno{6}-\thisrowno{6})/2,
			y error minus expr=(\thisrowno{6}-\thisrowno{6})/2] {data/W7X-KJM/DKES/results.data};
			
			%%% SFINCS RESULTS				
			\addplot[mark repeat = 2, only marks, blue, mark=triangle*] 
		    table[skip first n=1,
			x expr=\thisrowno{0},
			y expr=\thisrowno{2}*0.5132
			] {data/W7X-KJM/SFINCS/20230420-01-020-W7X-KJM_Er0_nuScan_sfincs_results.dat};
			\addlegendentry{\texttt{SFINCS} $\widehat{E}_r=0$}
			%%% MONKES RESULTS
			\addplot+[no markers, red]
			table[skip first n=1, 
			x expr=\thisrowno{0},
			y expr=\thisrowno{6} *0.5132,  
			restrict expr to domain={\thisrowno{0}}{1e-5:3e+2},  
			restrict expr to domain={\thisrowno{1}}{3e-4:3e-4}
			]
			{data/W7X-KJM/MONKES/Monoenergetic_19_79_180/monkes_Monoenergetic_Database.dat};
			\addlegendentry{\texttt{MONKES} $\widehat{E}_r\ne0$}
			
			%%% DKES RESULTS				
			\addplot[only marks, mark=square*, red, error bars/.cd, y explicit,
			y dir=both,
			error bar style={thick}] table[
			x expr=\thisrowno{0},
			y expr={ (\thisrowno{6} + \thisrowno{7})/2 *0.5132 }, 
			restrict expr to domain={\thisrowno{0}}{1e-5:3e2},
			restrict expr to domain={\thisrowno{1}}{3e-4:3e-4},
			y error plus expr=(\thisrowno{6}-\thisrowno{6})/2,
			y error minus expr=(\thisrowno{6}-\thisrowno{6})/2] {data/W7X-KJM/DKES/results.data};
			
			%%% SFINCS RESULTS				
			\addplot[mark repeat =2, only marks, red, mark=triangle*, mark options={rotate=60}] 
			table[skip first n=1,
			x expr=\thisrowno{0},
			y expr=\thisrowno{2}*0.5132
			] {data/W7X-KJM/SFINCS/20230420-01-017-W7X-KJM_withEr_nuScan_sfincs_results.dat};
			
		\end{axis}
		
	\end{tikzpicture}


    	\caption{}
    	\label{subfig:D31_Benchmark_W7X_KJM}
    \end{subfigure}
    %
    \tikzsetnextfilename{Benchmark-CIEMAT-QI-s0250-D31}
    \begin{subfigure}[t]{0.32\textwidth}		
    	\begin{tikzpicture}
		\begin{axis}[
			width=\textwidth, 
			xlabel=$\hat{\nu}$  ${[\text{m}^{-1}]}$, ylabel=$\widehat{D}_{31} $ ${[\text{m}]}$, 
			xmode=log, enlarge x limits=false,
%			xtick = {1e-4, 1e-2, 1e0, 1e2}, 
			xtick = {1e-5, 1e-3, 1e-1, 1e1},
%			xmax = 0.5e0, 
			ymin = -0.018*0.4674,
			ymax = 0.128*0.4674,
%			enlarge y limits = false, 
			scaled y ticks=base 10:2,
			y tick label style={
				/pgf/number format/.cd,
				fixed,
				fixed zerofill,
				precision=0,
				/tikz/.cd}, 
			legend pos = north east, 
			mark size = 1.5pt
			]
			
			%%% MONKES RESULTS
			\addplot+[forget plot, no markers, blue]
			table[skip first n=1, 
			x expr=\thisrowno{0},
			y expr=\thisrowno{3}*0.4674
			]
			{data/CIEMAT-QI/MONKES/Monoenergetic_Database_Example_15_119_180/Monoenergetic_1.plt};	

			%%% DKES RESULTS	
			\foreach \nu in 
			{1e-5,3e-5,1e-4,3e-4,3e-3,1e-2,3e-2,1e-1,3e-1,1e-0,3e-0,1e+1,3e+1,1e+2,3e+2}
			{
				\addplot[forget plot, only marks, mark=*, blue] table[skip first n=2, 
				x expr=\thisrowno{0},
				y expr={ (\thisrowno{6} + \thisrowno{7})/2*0.4674 }, 
				restrict expr to domain={\thisrowno{1}}{0:0}] {data/CIEMAT-QI/DKES/omega_0e-0/cl_\nu/results.stellopt};
			} 
			
			%%% DKES RESULTS	
			\foreach \nu in 
			{1e-5,1e-4,3e-4,3e-3,1e-2,3e-2,1e-1,3e-1,1e-0,3e-0,1e+1,3e+1,1e+2,3e+2}
			{
				\addplot[forget plot, only marks, mark=square*, red] table[skip first n=2, 
				x expr=\thisrowno{0},
				y expr={ (\thisrowno{6} + \thisrowno{7})/2*0.4674 }, 
				restrict expr to domain={\thisrowno{1}}{1e-3:1e-3}] {data/CIEMAT-QI/DKES/omega_1e-3/cl_\nu/results.stellopt};
			} 
		
			%%% SFINCS RESULTS
			\addplot[forget plot, mark repeat = 2, only marks, blue, mark=triangle*]
			table[col sep = comma, skip first n=1, 
			x expr=\thisrowno{0},
			y expr=\thisrowno{2}*0.4674
			]
			{data/CIEMAT-QI/SFINCS/20230420-01-058-CIEMAT-QI_s0p238_Er0_nuScan_sfincs_results.dat};	
			
			%%% MONKES RESULTS
			\addplot+[forget plot, no markers, red]
			table[skip first n=1, 
			x expr=\thisrowno{0},
			y expr=\thisrowno{3}*0.4674,  
			restrict expr to domain={\thisrowno{1}}{1e-3:1e-3}]
			{data/CIEMAT-QI/MONKES/Monoenergetic_Database_Example_15_119_180/Monoenergetic_6.plt};	
			
			%%% DKES RESULTS				
			\addplot[only marks, mark=square*, red, error bars/.cd, y explicit,
			y dir=both,
			error bar style={thick}] table[
			x expr=\thisrowno{0},
			y expr={ (\thisrowno{6} + \thisrowno{7})/2*0.4674 }, 
			restrict expr to domain={\thisrowno{0}}{1e-4:3e2},
			restrict expr to domain={\thisrowno{1}}{1e-3:1e-3},
			y error plus expr=(\thisrowno{6}-\thisrowno{6})/2,
			y error minus expr=(\thisrowno{6}-\thisrowno{6})/2] {data/CIEMAT-QI/DKES/results.data};
			\addlegendentry{\texttt{DKES} $\widehat{E}_r \ne 0$}
			
			%%% SFINCS RESULTS
			\addplot[mark repeat = 2, only marks, red, mark=triangle*, mark options={rotate=60}]
			table[col sep = comma, skip first n=1, 
			x expr=\thisrowno{0},
			y expr=\thisrowno{2}*0.4674
			]
			{data/CIEMAT-QI/SFINCS/20230420-01-047-CIEMAT-QI_s0p238_withEr_nuScan_sfincs_results.dat};	 		
			
			\addlegendentry{\texttt{SFINCS} $\widehat{E}_r \ne 0$}
		\end{axis}
		
	\end{tikzpicture}


    	\caption{}
    	\label{subfig:D31_Benchmark_CIEMAT_QI}
    \end{subfigure}
    \caption{Calculation of $\widehat{D}_{31}$ by \texttt{MONKES}, \texttt{DKES} and \texttt{SFINCS} for zero and finite $\widehat{E}_r$. (a) W7-X EIM at the surface $\psi /\psi_{\text{lcfs}}=0.200$. (b) W7-X KJM at the surface $\psi /\psi_{\text{lcfs}}=0.204$. (c) CIEMAT-QI at the surface $\psi /\psi_{\text{lcfs}}=0.250$. }
    \label{fig:D31_Benchmark}
\end{figure*}

\begin{figure*}[h]
	\captionsetup[sub]{skip=-1.75pt, margin=95pt}
	\centering
	\tikzexternaldisable
	\ref{legendD33}
	\tikzexternalenable
	
	\tikzsetnextfilename{Benchmark-W7X-EIM-s0200-D33}
	\begin{subfigure}[t]{0.32\textwidth}	
		
	\begin{tikzpicture}
		\begin{axis}[
			width=0.97\textwidth, 
			xlabel=$\hat{\nu}$  ${[\text{m}^{-1}]}$, ylabel=$\widehat{D}_{33} - \widehat{D}_{33, \text{Sp}}$ ${[\text{m}]}$,
			xmode=log, 
			ymode=log, 
%			ymax=3e6,
%            xmax=0.01,
            xtick = {1e-5, 1e-3, 1e-1, 1e1},
            ytick={1e-10, 1e-3,1e4},
			enlarge x limits=false,
%			enlarge y limits=false,
			legend pos = south west,
			mark size = 1.5pt
			]
			
			%%% MONKES RESULTS
			\addplot+[no markers, blue]
			table[skip first n=1, 
			x expr=\thisrowno{0},
			y expr=(\thisrowno{5} - \thisrowno{6})
			]
			{./results/W7X-EIM/0.200/Monoenergetic_Database_Example_23_55_140/Monoenergetic_1.plt};	
%			\addlegendentry{\texttt{MONKES} $1/{\nu}$}			
						
			%%% SFINCS RESULTS
			%			\addplot[only marks, blue, mark=triangle*]
			%			table[col sep = comma, skip first n=1, 
			%			x expr=\thisrowno{0},
			%			y expr=\thisrowno{2}
			%			]
			%			{./results/SFINCS_results/W7X-EIM/20230420-01-027-W7X-EIM_Er0_nuScan_sfincs_results.dat};	
			%			\addlegendentry{\texttt{SFINCS} $\hat{E}_r = 0$}			
					
					
			%%% DKES RESULTS				
			\addplot[only marks, mark=*, blue, error bars/.cd, y explicit,
			y dir=both,
			error bar style={thick}] table[
			x expr=\thisrowno{0},
			y expr={ (\thisrowno{8} + \thisrowno{9})/2 }, 
			restrict expr to domain={\thisrowno{0}}{1e-5:3e2},
			restrict expr to domain={\thisrowno{1}}{0:0},
			y error plus expr=(\thisrowno{6}-\thisrowno{6})/2,
			y error minus expr=(\thisrowno{6}-\thisrowno{6})/2] {./results/W7X-EIM/0.200/results.data};
%			\addlegendentry{\texttt{DKES} $1/{\nu}$}
			
			%%% MONKES RESULTS
			\addplot+[no markers, red, dashed]
			table[skip first n=1, 
			x expr=\thisrowno{0},			
			y expr=(\thisrowno{5} - \thisrowno{6}),  
			restrict expr to domain={\thisrowno{1}}{3e-4:3e-4}]
			{./results/W7X-EIM/0.200/Monoenergetic_Database_Example_27_55_140/Monoenergetic_5.plt};	
%			\addlegendentry{\texttt{MONKES} $\sqrt{\nu}$}
			
			%%% DKES RESULTS				
			\addplot[only marks, mark=square, red, error bars/.cd, y explicit,
			y dir=both,
			error bar style={thick}] table[
			x expr=\thisrowno{0},
			y expr={ (\thisrowno{8} + \thisrowno{9})/2 }, 
			restrict expr to domain={\thisrowno{0}}{1e-5:3e2},
			restrict expr to domain={\thisrowno{1}}{3e-4:3e-4},
			y error plus expr=(\thisrowno{6}-\thisrowno{6})/2,
			y error minus expr=(\thisrowno{6}-\thisrowno{6})/2] {./results/W7X-EIM/0.200/results.data};
%			\addlegendentry{\texttt{DKES} $\sqrt{\nu}$}			
			
		\end{axis}
		
	\end{tikzpicture}


		\caption{}
		\label{subfig:D33_Benchmark_W7X_EIM}
	\end{subfigure}
	%
	\tikzsetnextfilename{Benchmark-W7X-KJM-s0204-D33}
	\begin{subfigure}[t]{0.32\textwidth}		
		
	\begin{tikzpicture}
		\begin{axis}[
			width=0.97\textwidth, 
			xlabel=$\hat{\nu}$  ${[\text{m}^{-1}]}$, ylabel=$\widehat{D}_{33} - \widehat{D}_{33, \text{Sp}}$ ${[\text{m}]}$,
			xmode=log, 
			ymode=log, 
			ytick={1e-10, 1e-3, 1e4},
			xtick = {1e-5, 1e-3, 1e-1, 1e1},
%			xmax = 183.5, 
			enlarge x limits=false,
%			enlarge y limits=false,
			legend pos = south west,
			mark size = 1.5pt
			%			legend style={at={(axis cs:1e-4,14)},anchor=north west}
			]
			
			%%% MONKES RESULTS
			\addplot+[forget plot, no markers, blue]
			table[skip first n=1, 
			x expr=\thisrowno{0},
			y expr=(\thisrowno{5}-\thisrowno{6}),  
			restrict expr to domain={\thisrowno{1}}{0:0}]
			{./results/W7X-KJM/0.204/Monoenergetic_Database_Example_23_63_140/Monoenergetic_1.plt};	
%			\addlegendentry{\texttt{MONKES} $1/{\nu}$}
			
			%%% DKES RESULTS				
			\addplot[forget plot, only marks, blue, mark=*, error bars/.cd, y explicit,
			y dir=both,
			error bar style={thick}] table[
			x expr=\thisrowno{0},
			y expr={ (\thisrowno{8} + \thisrowno{9})/2 }, 
			restrict expr to domain={\thisrowno{0}}{1e-5:3e2},
			restrict expr to domain={\thisrowno{1}}{0:0},
			y error plus expr=(\thisrowno{6}-\thisrowno{6})/2,
			y error minus expr=(\thisrowno{6}-\thisrowno{6})/2] {./results/W7X-KJM/0.204/results.data};
%			\addlegendentry{\texttt{DKES} $1/{\nu}$}
			
			%%% SFINCS RESULTS				
			%			\addplot[only marks, blue, mark=triangle*] 
			%			table[skip first n=1,
			%			x expr=\thisrowno{0},
			%			y expr=\thisrowno{1}
			%			] {./results/SFINCS_results/W7X-KJM/20230420-01-020-W7X-KJM_Er0_nuScan_sfincs_results.dat};
			%			\addlegendentry{\texttt{SFINCS} $\hat{E}_r = 0$}
			
			%%% MONKES RESULTS
			\addplot+[no markers, red, dashed]
			table[skip first n=1, 
			x expr=\thisrowno{0},
			y expr=abs(\thisrowno{9}-\thisrowno{8}),  
			restrict expr to domain={\thisrowno{0}}{1e-5:3e+2},  
			restrict expr to domain={\thisrowno{1}}{3e-4:3e-4}
			]
			{./results/W7X-KJM/0.204/Monoenergetic_19_79_180/monkes_Monoenergetic_Database.dat};
%			\addlegendentry{\texttt{MONKES} $\sqrt{\nu}$}
			
			
			
			%%% DKES RESULTS				
			\addplot[only marks, mark=square, red, error bars/.cd, y explicit,
			y dir=both,
			error bar style={thick}] table[
			x expr=\thisrowno{0},
			y expr={ (\thisrowno{8} + \thisrowno{9})/2 }, 
			restrict expr to domain={\thisrowno{0}}{1e-5:3e2},
			restrict expr to domain={\thisrowno{1}}{3e-4:3e-4},
			y error plus expr=(\thisrowno{6}-\thisrowno{6})/2,
			y error minus expr=(\thisrowno{6}-\thisrowno{6})/2] {./results/W7X-KJM/0.204/results.data};
%			\addlegendentry{\texttt{DKES} $\sqrt{\nu}$}			
			
			%%% SFINCS RESULTS				
			%			\addplot[only marks, red, mark=diamond*] 
			%			table[skip first n=1,
			%			x expr=\thisrowno{0},
			%			y expr=\thisrowno{1}
			%			] {./results/SFINCS_results/W7X-KJM/20230420-01-017-W7X-KJM_withEr_nuScan_sfincs_results.dat};
			%			\addlegendentry{\texttt{SFINCS} $\hat{E}_r =  3 \cdot 10^{-4}$}
			
		\end{axis}
		
	\end{tikzpicture}
%
		\caption{}
		\label{subfig:D33_Benchmark_W7X_KJM}
	\end{subfigure}
	%
	\tikzsetnextfilename{Benchmark-CIEMAT-QI-s0250-D33}
	\begin{subfigure}[t]{0.32\textwidth}
%		\tikzexternaldisable
		
	\begin{tikzpicture}
		\begin{axis}[
			width=0.97\textwidth,
			xlabel=$\hat{\nu}$  ${[\text{m}^{-1}]}$, ylabel=$\widehat{D}_{33} - \widehat{D}_{33, \text{Sp}}$ ${[\text{m}]}$,
			xmode=log, 
			ymode=log, 
			enlarge x limits=false,
			xtick = {1e-5, 1e-3, 1e-1, 1e1},
%			enlarge y limits=false,
%			legend pos = south west,
			legend columns=4, 
			legend style={/tikz/every even column/.append style={column sep=1.1cm}, cells={align=left}},
			legend to name={legendD33},
%			xmax=0.01,
			mark size = 1.5pt
			]
			
			%%% MONKES RESULTS
			\addplot[no markers, blue]
			table[skip first n=1, 
			x expr=\thisrowno{0},
			y expr=\thisrowno{5}-\thisrowno{6}
			]
			{./results/CIEMAT-QI/0.250/Monoenergetic_Database_Example_15_119_180/Monoenergetic_1.plt};	
			\addlegendentry{\texttt{MONKES}\\ $\widehat{E}_r=0$}			
			
			
			%%% DKES RESULTS	
			\foreach \nu in 
			{1e-5,1e-4,3e-4,3e-3,1e-2,3e-2,1e-1,3e-1,1e-0,3e-0,1e+1,3e+1,1e+2,3e+2}
			{
				\addplot[forget plot, only marks, mark=*, blue] table[skip first n=2, 
				x expr=\thisrowno{0},
				y expr={ (\thisrowno{8} + \thisrowno{9})/2 }, 
				restrict expr to domain={\thisrowno{1}}{0:0}] {./results/CIEMAT-QI/0.250/omega_0e-0/cl_\nu/results.stellopt};
			} 
			%%% DKES RESULTS				
			\addplot[only marks, mark=*, blue, error bars/.cd, y explicit,
			y dir=both,
			error bar style={thick}] table[
			x expr=\thisrowno{0},
			y expr={ (\thisrowno{8} + \thisrowno{9})/2 }, 
			restrict expr to domain={\thisrowno{0}}{1e-5:3e2},
			restrict expr to domain={\thisrowno{1}}{0:0},
			y error plus expr=(\thisrowno{6}-\thisrowno{6})/2,
			y error minus expr=(\thisrowno{6}-\thisrowno{6})/2] {./results/CIEMAT-QI/0.250/results.data};
			\addlegendentry{\texttt{DKES} \\ $\widehat{E}_r=0$}
			
			%%% MONKES RESULTS
			\addplot[no markers, red, dashed]
			table[skip first n=1, 
			x expr=\thisrowno{0},
			y expr=\thisrowno{5}-\thisrowno{6}, 
			restrict expr to domain={\thisrowno{1}}{1e-3:1e-3}]
			{./results/CIEMAT-QI/0.250/Monoenergetic_Database_Example_15_119_180/Monoenergetic_6.plt};	
			\addlegendentry{\texttt{MONKES}\\ $\widehat{E}_r\ne 0$}
			
			%%% DKES RESULTS				
			\addplot[only marks, mark=square, red, error bars/.cd, y explicit,
			y dir=both,
			error bar style={thick}] table[
			x expr=\thisrowno{0},
			y expr={ (\thisrowno{8} + \thisrowno{9})/2 }, 
			restrict expr to domain={\thisrowno{0}}{1e-5:3e2},
			restrict expr to domain={\thisrowno{1}}{1e-3:1e-3},
			y error plus expr=(\thisrowno{6}-\thisrowno{6})/2,
			y error minus expr=(\thisrowno{6}-\thisrowno{6})/2] {./results/CIEMAT-QI/0.250/results.data};
			\addlegendentry{\texttt{DKES} \\ $\widehat{E}_r\ne0$}	
			
			
			%%% DKES RESULTS	
			\foreach \nu in 
			{1e-5,1e-4,3e-4,3e-3,1e-2,3e-2,1e-1,3e-1,1e-0,3e-0,1e+1,3e+1,1e+2,3e+2}
			{
				\addplot[forget plot, only marks, mark=square, red] table[skip first n=2, 
				x expr=\thisrowno{0},
				y expr={ (\thisrowno{8} + \thisrowno{9})/2 }, 
				restrict expr to domain={\thisrowno{1}}{1e-3:1e-3}] {./results/CIEMAT-QI/0.250/omega_1e-3/cl_\nu/results.stellopt};
			} 
		\end{axis}
		
	\end{tikzpicture}


%		\tikzexternalenable
		\caption{}
		\label{subfig:D33_Benchmark_CIEMAT_QI}
	\end{subfigure}
	\caption{Calculation of $\widehat{D}_{33}$ by \texttt{MONKES} and \texttt{DKES} for zero and finite $\hat{E}_r$. (a) W7-X EIM at the surface $\psi /\psi_{\text{lcfs}}=0.200$. (b) W7-X KJM at the surface $\psi /\psi_{\text{lcfs}}=0.204$. (c) CIEMAT-QI at the surface $\psi /\psi_{\text{lcfs}}=0.250$.}
	\label{fig:D33_Benchmark}
\end{figure*}

%\FloatBarrier