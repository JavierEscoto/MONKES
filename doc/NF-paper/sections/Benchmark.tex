Once we have chosen the resolutions $(N_\theta,N_\zeta,N_\xi)$ for each case, we need to verify that these selections indeed provide sufficiently accurate calculations of all the monoenergetic coefficients in the interval $\hat{\nu}\in[10^{-5},300]$ $\text{m}^{-1} $. In all cases, {\MONKES} calculations of the $\widehat{D}_{11}$ and $\widehat{D}_{31}$ coefficients will be benchmarked against converged calculations from {\DKES} (see \ref{sec:Appendix_DKES_Bounds}) and from \texttt{SFINCS}\footnote{\texttt{SFINCS} calculations are converged up to 3\% in the three independent variables.}. The parallel conductivity coefficient will be benchmarked only against {\DKES}. The benchmarking of the coefficient $\widehat{D}_{11}$ is shown in figure \ref{fig:D11_Benchmark}. For W7-X, perfect agreement with and without $\hat{E}_r$ is obtained. This is shown in figures \ref{subfig:D11_Benchmark_W7X_EIM} and \ref{subfig:D11_Benchmark_W7X_KJM}. The good agreement in the $\widehat{D}_{11}$ calculation between the three codes for the CIEMAT-QI cases is shown in figure \ref{subfig:D11_Benchmark_CIEMAT_QI}. 

Figures \ref{subfig:D31_Benchmark_W7X_EIM} and \ref{subfig:D31_Benchmark_W7X_KJM} show that {\MONKES} calculations of the bootstrap current coefficient for W7-X EIM and W7-X KJM are in total agreement with those of {\DKES} and \texttt{SFINCS}. For the CIEMAT-QI cases with and without $\hat{E}_r$, there is also very good agreement, as shown in figure \ref{subfig:D31_Benchmark_CIEMAT_QI}.
%

For the parallel conductivity coefficient, the agreement between the results of {\MONKES} and {\DKES} is very good. We can observe in figure \ref{fig:D33_Benchmark} that, for all the cases, is difficult to distinguish between the results of {\MONKES} and {\DKES}. Due to the weak effect of the radial electric field in the $\widehat{D}_{33}$ coefficient, the symbols for this plot have been changed. 
%


\input{sections/D11_Benchmark.tex}
\begin{figure*}[h]
	\centering
	\captionsetup[sub]{skip=-1.75pt, margin=80pt}
	\tikzexternaldisable
	\ref{legend}
	\tikzexternalenable
    \tikzsetnextfilename{Benchmark-W7X-EIM-s0200-D31}
    \begin{subfigure}[t]{0.32\textwidth}	
        	\begin{tikzpicture}
		\begin{axis}[
			width=\textwidth, 
			xlabel=$\hat{\nu}$  ${[\text{m}^{-1}]}$, ylabel=$\widehat{D}_{31}$ ${[\text{m}]}$, 
			xmode=log, enlarge x limits=false,
			scaled y ticks=base 10:1,
			y tick label style={
				/pgf/number format/.cd,
				fixed,
				fixed zerofill,
				precision=0,
				/tikz/.cd}, 
			legend pos = north east,
			mark size = 1.5pt
			]
			
			%%% MONKES RESULTS
			\addplot+[no markers, blue]
			table[skip first n=1, 
			x expr=\thisrowno{0},
			y expr=\thisrowno{3}*0.5237						
			]
			{data/W7X-EIM/MONKES/Monoenergetic_Database_Example_23_55_140/Monoenergetic_1.plt};				
			\addlegendentry{{\MONKES} $\widehat{E}_r=0$}
			
			%%% DKES RESULTS				
			\addplot[only marks, mark=*, blue, error bars/.cd, y explicit,
			y dir=both,
			error bar style={thick}] table[
			x expr=\thisrowno{0},
			y expr={ (\thisrowno{6} + \thisrowno{7})/2*0.5237 }, 
			restrict expr to domain={\thisrowno{0}}{1e-5:3e2},
			restrict expr to domain={\thisrowno{1}}{0:0},
			y error plus expr=(\thisrowno{6}-\thisrowno{6})/2,
			y error minus expr=(\thisrowno{6}-\thisrowno{6})/2] {data/W7X-EIM/DKES/results.data};
			\addlegendentry{{\DKES} $\widehat{E}_r=0$}
			
			%%% SFINCS RESULTS
			\addplot[mark repeat =2,only marks, blue, mark=triangle*]
			table[col sep = comma, skip first n=1, 
			x expr=\thisrowno{0},
			y expr=\thisrowno{2}*0.5237
			]
			{data/W7X-EIM/SFINCS/20230420-01-027-W7X-EIM_Er0_nuScan_sfincs_results.dat};	

			
			
			%%% MONKES RESULTS
			\addplot+[no markers, red]
			table[skip first n=1, 
			x expr=\thisrowno{0},
			y expr=\thisrowno{6}*0.5237,  
			restrict expr to domain={\thisrowno{1}}{3e-4:3e-4}]
			{data/W7X-EIM/MONKES/Convergence_nu_1e-5_Er_3e-4/monkes_Monoenergetic_Database.dat};	
%			\addlegendentry{\texttt{MONKES} $\sqrt{\nu}$}
			
			
			%%% DKES RESULTS				
			\addplot[only marks, mark=square*, red, error bars/.cd, y explicit,
			y dir=both,
			error bar style={thick}] table[
			x expr=\thisrowno{0},
			y expr={ (\thisrowno{6} + \thisrowno{7})/2*0.5237}, 
			restrict expr to domain={\thisrowno{0}}{1e-5:3e2},
			restrict expr to domain={\thisrowno{1}}{3e-4:3e-4},
			y error plus expr=(\thisrowno{6}-\thisrowno{6})/2,
			y error minus expr=(\thisrowno{6}-\thisrowno{6})/2] {data/W7X-EIM/DKES/results.data};
%			\addlegendentry{\texttt{DKES} $\sqrt{\nu}$}			
			
			%%% SFINCS RESULTS
			\addplot[mark repeat =2, only marks, red, mark=triangle*, mark options={rotate=60}]
			table[col sep = comma, skip first n=1, 
			x expr=\thisrowno{0},
			y expr=\thisrowno{2}*0.5237
			]
			{data/W7X-EIM/SFINCS/20230420-01-018-W7X-EIM_withEr_nuScan_sfincs_results.dat};	
		\end{axis}
		
	\end{tikzpicture}


        \caption{}
        \label{subfig:D31_Benchmark_W7X_EIM}
    \end{subfigure}
    %
    \tikzsetnextfilename{Benchmark-W7X-KJM-s0204-D31}
    \begin{subfigure}[t]{0.32\textwidth}		
    	
	\begin{tikzpicture}
		\begin{axis}[
			width=\textwidth, 
			xlabel=$\hat{\nu}$  ${[\text{m}^{-1} ]}$, ylabel=$\widehat{D}_{31}$ ${[\text{m}]}$, 
			xmode=log, enlarge x limits=false,
			legend pos = north east,
			scaled y ticks=base 10:1,
			y tick label style={
				/pgf/number format/.cd,
				fixed,
				fixed zerofill,
				precision=0,
				/tikz/.cd}, 
			xtick = {1e-5, 1e-3, 1e-1, 1e1},
			mark size = 1.5pt
			%			legend style={at={(axis cs:1e-4,14)},anchor=north west}
			]
			
			%%% MONKES RESULTS
			\addplot+[no markers, blue]
			table[skip first n=1, 
			x expr=\thisrowno{0},
			y expr=\thisrowno{3} *0.5132,  
			restrict expr to domain={\thisrowno{1}}{0:0}]
			{./results/W7X-KJM/0.204/Monoenergetic_Database_Example_23_63_140/Monoenergetic_1.plt};	
%			\addlegendentry{\texttt{MONKES} $\sqrt{\nu}$}
			
			%%% DKES RESULTS				
			\addplot[only marks, blue, mark=*, error bars/.cd, y explicit,
			y dir=both,
			error bar style={thick}] table[
			x expr=\thisrowno{0},
			y expr={ (\thisrowno{6} + \thisrowno{7})/2  *0.5132 }, 
			restrict expr to domain={\thisrowno{0}}{1e-5:3e2},
			restrict expr to domain={\thisrowno{1}}{0:0},
			y error plus expr=(\thisrowno{6}-\thisrowno{6})/2,
			y error minus expr=(\thisrowno{6}-\thisrowno{6})/2] {./results/W7X-KJM/0.204/results.data};
%			\addlegendentry{\texttt{DKES} $1/{\nu}$}
			
			%%% SFINCS RESULTS				
			\addplot[mark repeat = 2, only marks, blue, mark=triangle*] 
		    table[skip first n=1,
			x expr=\thisrowno{0},
			y expr=\thisrowno{2}*0.5132
			] {./results/SFINCS_results/W7X-KJM/20230420-01-020-W7X-KJM_Er0_nuScan_sfincs_results.dat};
			%			\addlegendentry{\texttt{SFINCS} $\hat{E}_r = 0$}
			
			%%% MONKES RESULTS
			\addplot+[no markers, red]
			table[skip first n=1, 
			x expr=\thisrowno{0},
			y expr=\thisrowno{6} *0.5132,  
			restrict expr to domain={\thisrowno{0}}{1e-5:3e+2},  
			restrict expr to domain={\thisrowno{1}}{3e-4:3e-4}
			]
			{./results/W7X-KJM/0.204/Monoenergetic_19_79_180/monkes_Monoenergetic_Database.dat};	
%			\addlegendentry{\texttt{MONKES} $\sqrt{\nu}$}
			
			
			
			%%% DKES RESULTS				
			\addplot[only marks, mark=square*, red, error bars/.cd, y explicit,
			y dir=both,
			error bar style={thick}] table[
			x expr=\thisrowno{0},
			y expr={ (\thisrowno{6} + \thisrowno{7})/2 *0.5132 }, 
			restrict expr to domain={\thisrowno{0}}{1e-5:3e2},
			restrict expr to domain={\thisrowno{1}}{3e-4:3e-4},
			y error plus expr=(\thisrowno{6}-\thisrowno{6})/2,
			y error minus expr=(\thisrowno{6}-\thisrowno{6})/2] {./results/W7X-KJM/0.204/results.data};
%			\addlegendentry{\texttt{DKES} $\sqrt{\nu}$}			
			
			%%% SFINCS RESULTS				
			\addplot[mark repeat =2, only marks, red, mark=triangle*, mark options={rotate=60}] 
			table[skip first n=1,
			x expr=\thisrowno{0},
			y expr=\thisrowno{2}*0.5132
			] {./results/SFINCS_results/W7X-KJM/20230420-01-017-W7X-KJM_withEr_nuScan_sfincs_results.dat};
			%			\addlegendentry{\texttt{SFINCS} $\hat{E}_r =  3 \cdot 10^{-4}$}
			
		\end{axis}
		
	\end{tikzpicture}


    	\caption{}
    	\label{subfig:D31_Benchmark_W7X_KJM}
    \end{subfigure}
    %
    \tikzsetnextfilename{Benchmark-CIEMAT-QI-s0250-D31}
    \begin{subfigure}[t]{0.32\textwidth}		
    	\begin{tikzpicture}
		\begin{axis}[
			width=\textwidth, 
			xlabel=$\hat{\nu}$  ${[\text{m}^{-1}]}$, ylabel=$\widehat{D}_{31} $ ${[\text{m}]}$, 
			xmode=log, enlarge x limits=false,
			xtick = {1e-4, 1e-2, 1e0, 1e2}, 
			xmax = 0.5e0, 
			ymin = -0.018*0.4674,
			ymax = 0.128*0.4674,
%			enlarge y limits = false, 
			scaled y ticks=base 10:2,
			y tick label style={
				/pgf/number format/.cd,
				fixed,
				fixed zerofill,
				precision=0,
				/tikz/.cd}, 
			legend pos = north east, 
			mark size = 1.5pt
			]
			
			%%% MONKES RESULTS
			\addplot+[forget plot, no markers, blue]
			table[skip first n=1, 
			x expr=\thisrowno{0},
			y expr=\thisrowno{3}*0.4674
			]
			{./results/CIEMAT-QI/0.250/Monoenergetic_Database_Example_15_119_180/Monoenergetic_1.plt};	
%			\addlegendentry{\texttt{MONKES} $1/{\nu}$}						
			%%% DKES RESULTS	
			\foreach \nu in 
			{1e-5,3e-5,1e-4,3e-4,3e-3,1e-2,3e-2,1e-1,3e-1,1e-0,3e-0,1e+1,3e+1,1e+2,3e+2}
			{
				\addplot[forget plot, only marks, mark=*, blue] table[skip first n=2, 
				x expr=\thisrowno{0},
				y expr={ (\thisrowno{6} + \thisrowno{7})/2*0.4674 }, 
				restrict expr to domain={\thisrowno{1}}{0:0}] {./results/CIEMAT-QI/0.250/omega_0e-0/cl_\nu/results.stellopt};
			} 
			
			%%% DKES RESULTS	
			\foreach \nu in 
			{1e-5,1e-4,3e-4,3e-3,1e-2,3e-2,1e-1,3e-1,1e-0,3e-0,1e+1,3e+1,1e+2,3e+2}
			{
				\addplot[forget plot, only marks, mark=square*, red] table[skip first n=2, 
				x expr=\thisrowno{0},
				y expr={ (\thisrowno{6} + \thisrowno{7})/2*0.4674 }, 
				restrict expr to domain={\thisrowno{1}}{1e-3:1e-3}] {./results/CIEMAT-QI/0.250/omega_1e-3/cl_\nu/results.stellopt};
			} 
			%%% DKES RESULTS				
%			\addplot[forget plot, only marks, mark=*, blue, error bars/.cd, y explicit,
%			y dir=both,
%			error bar style={thick}] table[
%			x expr=\thisrowno{0},
%			y expr={ (\thisrowno{6} + \thisrowno{7})/2*0.4674 }, 
%			restrict expr to domain={\thisrowno{0}}{1e-4:1e-4},
%			restrict expr to domain={\thisrowno{1}}{0:0},
%			y error plus expr=(\thisrowno{6}-\thisrowno{6})/2,
%			y error minus expr=(\thisrowno{6}-\thisrowno{6})/2]
%			{./results/CIEMAT-QI/0.250/results.data};
			%			\addlegendentry{\texttt{DKES} $1/{\nu}$}
			
			
			
			
			%%% SFINCS RESULTS
			\addplot[mark repeat = 2, only marks, blue, mark=triangle*]
			table[col sep = comma, skip first n=1, 
			x expr=\thisrowno{0},
			y expr=\thisrowno{2}*0.4674
			]
			{./results/SFINCS_results/CIEMAT-QI/20230420-01-058-CIEMAT-QI_s0p238_Er0_nuScan_sfincs_results.dat};	
%			\addlegendentry{\texttt{SFINCS} $1/{\nu}$}	
			
			%%% MONKES RESULTS
			\addplot+[forget plot, no markers, red]
			table[skip first n=1, 
			x expr=\thisrowno{0},
			y expr=\thisrowno{3}*0.4674,  
			restrict expr to domain={\thisrowno{1}}{1e-3:1e-3}]
			{./results/CIEMAT-QI/0.250/Monoenergetic_Database_Example_15_119_180/Monoenergetic_6.plt};	
%			\addlegendentry{\texttt{MONKES} $\sqrt{\nu}$}
			
			
			%%% DKES RESULTS				
			\addplot[forget plot, only marks, mark=square*, red, error bars/.cd, y explicit,
			y dir=both,
			error bar style={thick}] table[
			x expr=\thisrowno{0},
			y expr={ (\thisrowno{6} + \thisrowno{7})/2*0.4674 }, 
			restrict expr to domain={\thisrowno{0}}{1e-4:3e2},
			restrict expr to domain={\thisrowno{1}}{1e-3:1e-3},
			y error plus expr=(\thisrowno{6}-\thisrowno{6})/2,
			y error minus expr=(\thisrowno{6}-\thisrowno{6})/2] {./results/CIEMAT-QI/0.250/results.data};
%			\addlegendentry{\texttt{DKES} $\sqrt{\nu}$}	
			
			
			%%% SFINCS RESULTS
			\addplot[mark repeat = 2, only marks, red, mark=triangle*, mark options={rotate=60}]
			table[col sep = comma, skip first n=1, 
			x expr=\thisrowno{0},
			y expr=\thisrowno{2}*0.4674
			]
			{./results/SFINCS_results/CIEMAT-QI/20230420-01-047-CIEMAT-QI_s0p238_withEr_nuScan_sfincs_results.dat};	
%			\addlegendentry{\texttt{SFINCS} $\sqrt{\nu}$}			
			
		\end{axis}
		
	\end{tikzpicture}


    	\caption{}
    	\label{subfig:D31_Benchmark_CIEMAT_QI}
    \end{subfigure}
    \caption{Calculation of $\widehat{D}_{31}$ by \texttt{MONKES}, \texttt{DKES} and \texttt{SFINCS} for zero and finite $\widehat{E}_r$. (a) W7-X EIM at the surface $\psi /\psi_{\text{lcfs}}=0.200$. (b) W7-X KJM at the surface $\psi /\psi_{\text{lcfs}}=0.204$. (c) CIEMAT-QI at the surface $\psi /\psi_{\text{lcfs}}=0.250$. }
    \label{fig:D31_Benchmark}
\end{figure*}
\input{sections/D33_Benchmark.tex}

%\FloatBarrier