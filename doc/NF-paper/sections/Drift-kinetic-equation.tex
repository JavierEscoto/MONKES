{\MONKES} solves the drift-kinetic equation 
%
\begin{align}
(v \xi \vb*{b}  + \vb*{v}_E) \cdot \nabla h_a 
	+
	v\nabla \cdot \vb*{b} \frac{(1-\xi^2)}{2}  \pdv{h_a}{\xi}  
%	& 
	& - \nu^{a} \Lorentz h_a
		\nonumber \\
	& = S_a,
	\label{eq:DKE_Original}
\end{align}
where $\vb*{b}:= \vb*{B}/B$ is the unit vector tangent to magnetic field lines and we have employed as velocity coordinates the cosine of the pitch-angle $\xi := \vb*{v}\cdot\vb*{b}/|\vb*{v}|$ and the magnitude of the velocity $v:=|\vb*{v}|$. 

We assume that the magnetic configuration has nested flux-surfaces. We denote by $\psi\in[0,\psi_{\text{lcfs}}]$ a radial coordinate that labels flux-surfaces, where $\psi_{\text{lcfs}}$ denotes the label of the last closed flux-surface. In equation (\ref{eq:DKE_Original}), $h_a$ is the non-adiabatic component of the deviation of the distribution function from a local Maxwellian for a plasma species $a$ 
%
\begin{align}
	f_{\text{M}a}(\psi, v) :=   n_a(\psi)  \pi^{-3/2}  {v_{\text{t}a}^{-3}(\psi)}  \exp(-\frac{v^2}{v_{\text{t}a}^2(\psi)}).
\end{align}
Here, $n_a$ is the density of species $a$, $v_{\text{t}a} := \sqrt{2T_a/m_a}$ is its thermal velocity, $T_a$ its temperature (in energy units) and $m_a$ its mass. 

%The quantity $\phi_1$ is the piece of the electrostatic potential that fluctuates in the flux-surface. With an appropriate selection of the electromagnetic gauge \cite{Landreman_2012} it can be written as
%%
%\begin{align}
%	\phi_1 = -\int_{0}^{l}\left( \vb*{E}\cdot\vb*{B} - \mean*{\vb*{E}\cdot\vb*{B}} \frac{B^2}{\mean*{B^2}} \right) \frac{\dd{l'}}{B}.
%\end{align}
%
%Here, $n_a$ is the density of species $a$, $v_{\text{t}a} := \sqrt{2T_a/m_a}$ is its thermal velocity, $T_a$ its temperature (in energy units), $m_a$ its mass and $e_a$ its charge. We denote the length along magnetic field lines by $l$, $\vb*{E}$ is the electric field and the symbol $\mean*{...}$ stands for the flux-surface average operation. 

For the convective term in equation (\ref{eq:DKE_Original})
%
\begin{align}
	\vb*{v}_E 
	:= 
	\frac{\vb*{E}_0\times\vb*{B}}{\mean*{B^2}} 
	= 
	- 
	\frac{E_\psi}{\mean*{B^2}}\vb*{B}\times \nabla\psi
	\label{eq:Incompressible_ExB_definition}
\end{align}
denotes the incompressible $\vb*{E}\times\vb*{B}$ drift approximation \cite{dherbemont2022} and $\vb*{E}_0 = E_\psi(\psi) \nabla \psi$ is the electrostatic piece of the electric field $\vb*{E}$ perpendicular to the flux-surface. The symbol $\mean*{...}$ stands for the flux-surface average operation. Denoting by $V(\psi)$ the volume enclosed by the flux-surface labelled by $\psi$, the flux-surface average of a function $f$ can be defined as the limit
%
\begin{align}
	\mean*{f}
	:=
	\lim_{\delta \psi \rightarrow 0} 
	\dfrac{\int_{V(\psi+\delta\psi)} f \dd[3]{\vb*{r}}- \int_{V(\psi)} f\dd[3]{\vb*{r}}}
	{V(\psi+\delta\psi) - V(\psi)},
	\label{eq:FSA}
\end{align}
where $\dd[3]{\vb*{r}}$ is the spatial volume form.

We denote the Lorentz pitch-angle scattering operator by $\Lorentz$, which in coordinates $(\xi,v)$ takes the form
\begin{align}
 \Lorentz   := \frac{1}{2}  \pdv{\xi}\left( (1-\xi^2)\pdv{}{\xi} \right).
 \label{eq:Pitch_angle_scattering_operator}
\end{align}
In the collision operator, $\nu^a(v) =\sum_{b}\nu^{ab}(v)$ and
%
\begin{align}
	\nu^{ab}(v) := 
	\frac{4 \pi n_b e_a^2 e_b^2}
	{m_a^2 v_{\text{t}a}^3}
	\log\Lambda
	\frac{ \erf(v/v_{\text{t}b}) - G(v/v_{\text{t}b})}{v^3/v_{\text{t}a}^3}
\end{align}
stands for the pitch-angle collision frequency between species $a$ and $b$. We denote the respective charges of each species by $e_a$ and $e_b$, the Chandrasekhar function by $G(x)=\left[\erf(x) - (2x/\sqrt{\pi}) \exp(-x^2)\right]/(2x^2)$, $\erf(x)$ is the error function and $\log\Lambda$ is the Coulomb logarithm \cite{Helander_2005}. 
 



On the right-hand-side of equation (\ref{eq:DKE_Original}) 
%
\begin{align}
	S_a 
	& :=  
	- \vb*{v}_{\text{m} a} \cdot \nabla \psi 
	\left(
	A_{1a} 
	+  \frac{v^2}{v_{\text{t}a}^2}
	A_{2a}
	\right)
	f_{\text{M}a}
	\nonumber \\ 
	& + 
	B v \xi A_{3a}f_{\text{M}a}
	\label{eq:DKE_Original_Source}
\end{align}
is the source term, 
\begin{align}
	\vb*{v}_{\text{m} a}\cdot\nabla\psi
	=
	-\frac{Bv^2}{\Omega_a}
	\frac{1+\xi^2}{2B^3}
	\vb*{B}\times\nabla\psi \cdot \nabla B 
\end{align}
is the expression of the radial magnetic drift assuming ideal magnetohydrodynamical equilibrium, $\Omega_a = e_a B / m_a$ is the gyrofrecuency of species $a$ and the flux-functions 
%
\begin{align}
	A_{1a}(\psi) & := \dv{\ln n_a}{\psi} - \frac{3}{2} \dv{\ln T_a}{\psi} - \frac{e_a E_\psi}{T_a}, 
	\\
	A_{2a}(\psi) & := \dv{\ln T_a}{\psi} , 
	\\
	A_{3a}(\psi) & :=  \frac{e_a }{T_a} \frac{\mean*{\vb*{E}\cdot\vb*{B}}}{\mean*{B^2}}
\end{align}
are the so-called thermodynamical forces.

Mathematically speaking, there are still two additional conditions to completely determine the solution to equation (\ref{eq:DKE_Original}). First, equation (\ref{eq:DKE_Original}) must be solved imposing regularity conditions at $\xi =\pm 1$
%
\begin{align}
	\eval{\left((1-\xi^2) \pdv{h_a}{\xi}\right)}_{\xi =\pm 1} = 0.
	\label{eq:Regularity_conditions}
\end{align}
Second, as the differential operator on the left-hand-side of equation (\ref{eq:DKE_Original}) has a non trivial kernel, the solution to equation (\ref{eq:DKE_Original}) is determined up to an additive function $g(\psi,v)$. This function is unimportant as it does not contribute to the neoclassical transport quantities of interest. Nevertheless, in order to have a unique solution to the drift-kinetic equation, it must be fixed by imposing an appropriate additional constraint. We will select this free function (for fixed $(\psi,v)$) by imposing
%
\begin{align}
	\mean*{  \int_{-1}^{1} h_a \dd{\xi}  } = C,
	\label{eq:kernel_elimination_condition}
\end{align}
for some $C\in\mathbb{R}$. We will discuss this further in section \ref{sec:Algorithm}. 

The drift-kinetic equation (\ref{eq:DKE_Original}) is the one solved by the standard neoclassical code {\DKES} \cite{DKES1986, VanRij_1989} using a variational principle. 
Although the main feature of the code \texttt{SFINCS} \cite{Landreman_2014} is to solve a more complete neoclassical drift-kinetic equation, it can also solve equation (\ref{eq:DKE_Original}).

Taking the moments $\{\vb*{v}_{\text{m} a} \cdot \nabla\psi,  (v^2/v_{\text{t}a}^2)\vb*{v}_{\text{m} a} \cdot \nabla\psi, v\xi B/B_0\}$ of $h_a$ and then the flux-surface average yields, respectively, the radial particle flux, the radial heat flux and the parallel flow
%
\begin{align}
	\mean*{\vb*{\Gamma}_a \cdot \nabla \psi} & := 
	\mean*{
		\int
		\vb*{v}_{\text{m} a} \cdot \nabla\psi	
		\ h_a
		\dd[3]{\vb*{v}}
	},
   \label{eq:Particle_flux_Original}
	\\
	\mean*{\frac{\vb*{Q}_a \cdot \nabla \psi}{T_a}} & := 
	\mean*{
		\int
		\frac{v^2}{v_{\text{t}a}^2}\vb*{v}_{\text{m} a} \cdot \nabla\psi	
		\ h_a
		\dd[3]{\vb*{v}}
	},
    \label{eq:Heat_flux_Original}
	\\
	\frac{\mean*{n_a \vb*{V}_{a} \cdot\vb*{B}}}{B_{0}} & :=
	\mean*{
		\frac{B}{B_0}
		\int
		v \xi 
		\ h_a
		\dd[3]{\vb*{v}}
	},
    \label{eq:Parallel_flow_Original}
\end{align}
where $B_0(\psi)$ is a reference value for the magnetic field strength on the flux-surface (its explicit definition is given in section \ref{sec:Algorithm}).

It is a common practice for linear drift-kinetic equations (e.g. \cite{DKES1986, Beidler_2011,Landreman_2014}) to apply superposition and split $h_a$ into several additive terms. As in the drift-kinetic equation (\ref{eq:DKE_Original}) there are no derivatives or integrals along $\psi$ nor $v$, it is convenient to use the splitting
%
\begin{align}
	h_a 
	= 
	f_{\text{M}a}
	\left[
	\frac{B v}{\Omega_a} 
	\left(
	A_{1a} f_1 
	+ 
	A_{2a}  
	\frac{v^2}{v_{\text{t}a}^2}f_2
	\right)
	+
	B_0 A_{3a} f_3
	\right].
	\label{eq:Distribution_function_superposition}
\end{align}
The splitting is chosen so that the functions $\{f_j\}_{j=1}^{3}$ are solutions to
%
\begin{align}
	\xi \vb*{b}  \cdot 
	\nabla f_j
	& +
	\nabla \cdot \vb*{b} \frac{(1-\xi^2)}{2}  \pdv{f_j}{\xi}  
%	& 
	\nonumber\\
	&
	- 
	\frac{\widehat{E}_\psi}{\mean*{B^2}}
	\vb*{B}\times \nabla\psi\cdot \nabla f_j
%	 
	\label{eq:DKE}
	%	\\
	%	&
	- \hat{\nu}\Lorentz f_j
	=  s_j, \quad 
\end{align}
for $j=1,2,3$, where $\hat{\nu} := \nu(v) / v$ and $\widehat{E}_\psi := {E}_\psi/v$. The source terms are defined as
%
\begin{align}
	s_1 := - \vb*{v}_{\text{m} a} \cdot \nabla\psi \frac{\Omega_a}{B v^2},
	\quad
	s_2 :=  s_1, 
	\quad
	s_3 := \xi \frac{B}{B_0}.
	\label{eq:DKE_Sources}
\end{align} 
Note that each source $s_j$ corresponds to one of the three thermodynamic forces on the right-hand side of definition (\ref{eq:DKE_Original_Source}).

The relation between $h_a$ and $f_j$ given by equation (\ref{eq:Distribution_function_superposition}) is such that the transport quantities (\ref{eq:Particle_flux_Original}), (\ref{eq:Heat_flux_Original}) and (\ref{eq:Parallel_flow_Original}) can be written in terms of four transport coefficients which, for fixed $(\hat{\nu}, \widehat{E}_\psi)$, depend only on the magnetic configuration. As $\dv*{\hat{\nu}}{v}$ never vanishes, the dependence of $f_j$ on the velocity $v$ can be parametrized by its dependence on $\hat{\nu}$. Thus, for fixed $(\hat{\nu}, \widehat{E}_\psi)$, equation (\ref{eq:DKE}) is completely determined by the magnetic configuration. Hence, its unique solutions $f_j$ that satisfy conditions (\ref{eq:Regularity_conditions}) and (\ref{eq:kernel_elimination_condition}) are also completely determined by the magnetic configuration. The assumptions that lead to $\psi$ and $v$ appearing as parameters in the drift-kinetic equation (\ref{eq:DKE_Original}) comprise the so-called local monoenergetic approximation to neoclassical transport (see e.g. \cite{Landreman_Monoenergetic}).

Using splitting (\ref{eq:Distribution_function_superposition}) we can write the transport quantities (\ref{eq:Particle_flux_Original}), (\ref{eq:Heat_flux_Original}) and (\ref{eq:Parallel_flow_Original}) in terms of the Onsager matrix
%
\begin{align}
	&\Matrix{c}
	{
		\mean*{\vb*{\Gamma}_a \cdot \nabla \psi} \\
		\mean*{ \dfrac{\vb*{Q}_a \cdot \nabla \psi}{T_a} }     \\ 
		\dfrac{\mean*{n_a \vb*{V}_{a} \cdot\vb*{B}}}{B_0}
	}
	=
	\Matrix{ccc}
	{
		L_{11a} & L_{12a}  & L_{13a} \\
		L_{21a} & L_{22a}  & L_{23a} \\
		L_{31a} & L_{32a}  & L_{33a} 
	}
	\Matrix{c}
	{ 
		A_{1a} \\
		A_{2a} \\
		A_{3a} 
	}.
\end{align}
Here, we have defined the thermal transport coefficients as 
%
\begin{align}
	L_{ija} :=    
	\int_{0}^{\infty}
	2\pi v^2
	f_{\text{M}a} 
	w_i w_j 
	D_{ija} 
	\dd{v}, \ \ %
	%	i,j=1,2,3
	%	\in\{1,2,3\},
\end{align}
where $w_1=w_3=1$, $w_2=v^2/v_{\text{t}a}^2$ and we have used that $\int g\dd[3]{\vb*{v}} = 2\pi \int_{0}^{\infty}\int_{-1}^{1} g v^2 \dd{\xi}\dd{v}$ for any integrable function $g(\xi,v)$. The quantities $D_{ija}$ are defined as
%
\begin{align}
	D_{ija} & :=-\frac{B^2v^3}{\Omega_a^2} \widehat{D}_{ij}, &\quad i,j \in\{1,2\},
	\\
	D_{i3a} & :=  
	- \frac{B_0 B v^2}{\Omega_a} \widehat{D}_{i3}, &\quad i \in\{1,2\},
	\\
	D_{3ja} & := \frac{B v^2}{\Omega_a} \widehat{D}_{3j}, &\quad j \in\{1,2\},
	\\
	D_{33a} & := v B_0 \widehat{D}_{33}, &
\end{align}
where  
%
\begin{align}
	\widehat{D}_{ij}(\psi,v) := \mean*{ \int_{-1}^{1}  s_i f_j   \dd{\xi} }, \quad i,j\in\{1,2,3\}
	\label{eq:Monoenergetic_geometric_coefficients}
\end{align} 
are the monoenergetic geometric coefficients. Note that (unlike $D_{ija}$) the monoenergetic geometric coefficients $\widehat{D}_{ij}$ do not depend on the species for fixed $\hat{\nu}$ (however the correspondent value of $v$ associated to each $\hat{\nu}$ varies between species) and depend only on the magnetic geometry. In general, four independent monoenergetic geometric coefficients can be obtained by solving (\ref{eq:DKE}): $\widehat{D}_{11}$, $\widehat{D}_{13}$, $\widehat{D}_{31}$ and $\widehat{D}_{33}$. However, when the magnetic field possesses stellarator symmetry \cite{DEWAR1998275} or there is no radial electric field, Onsager symmetry implies $\widehat{D}_{13} = -\widehat{D}_{31}$ \cite{VanRij_1989} making only three of them independent (for further details see \ref{sec:Appendix_Onsager_symmetry}). Hence, obtaining the transport coefficients for all species requires to solve (\ref{eq:DKE}) for two different source terms $s_1$ and $s_3$. The algorithm for solving equation (\ref{eq:DKE}) is described in section \ref{sec:Algorithm}. 

Finally, we briefly comment on the validity of the coefficients provided by equation (\ref{eq:DKE}) for the calculation of the bootstrap current. The pitch-angle scattering collision operator used in equation (\ref{eq:DKE_Original}) lacks parallel momentum conservation. Besides, the pitch-angle scattering operator is not adequate for calculating parallel flow of electrons, which is a quantity required to compute the bootstrap current. Hence, in principle, the parallel transport directly predicted by equation (\ref{eq:DKE_Original}) is not correct. Fortunately, there exist techniques  \cite{Taguchi,Sugama-PENTA,Sugama2008,MaasbergMomentumCorrection} to calculate the radial and parallel transport associated to more accurate momentum conserving collision operators by just solving the simplified drift-kinetic equation (\ref{eq:DKE}). This has been done successfully in the past by the code \texttt{PENTA} \cite{Sugama-PENTA, Spong-PENTA}, using the results of {\DKES}. Nevertheless, the momentum restoring technique is not needed for minimizing the bootstrap current. In the method presented in section V of \cite{MaasbergMomentumCorrection}, when there is no net parallel inductive electric field (i.e. $A_{3a}=0$), the parallel flow with the correct collision operator for any species vanishes when two integrals in $v$ of $\widehat{D}_{31}$ vanish. Thus, minimizing $\widehat{D}_{31}$ translates in a minimization of the parallel flows of all species involved in the bootstrap current calculation, and therefore of this current.  
%
%\begin{align*}
%	\mean*{\vb*{\Gamma}_a \cdot \nabla \psi} & = 
%	A_{1a}
%	\int_{0}^{\infty}
%	2\pi v^2 f_{\text{M}a} 
%	\mean*{
%		\int_{-1}^{1}
%		\vb*{v}_{\text{m} a} \cdot \nabla\psi	
%		\left(
%		 \frac{B v}{\Omega_a} f_1
%		\right)
%		\dd{\xi}
%	}
%    \dd{v}
%	\\
%	& + 
%	A_{2a}
%	\int_{0}^{\infty}
%	2\pi v^2 f_{\text{M}a} 
%	\mean*{
%		\int_{-1}^{1}
%		\vb*{v}_{\text{m} a} \cdot \nabla\psi	
%		\left(  
%		\frac{B v}{\Omega_a} \frac{v^2}{v_{\text{t}a}^2}f_2
%		\right)
%		\dd{\xi}
%	}
%    \dd{v}
%	\\
%	& + 
%	A_{3a} 
%	\int_{0}^{\infty}
%	2\pi v^2 f_{\text{M}a} 
%	\mean*{
%		\int_{-1}^{1}
%		\vb*{v}_{\text{m} a} \cdot \nabla\psi	
%		\left(
%		f_3
%		\right)
%		\dd{\xi}
%	}
%    \dd{v}
%	\\
%	\frac{ \mean*{\vb*{Q}_a \cdot \nabla \psi}}{T_a} & = 
%	A_{1a}
%	\int_{0}^{\infty} 
%	2\pi v^2 f_{\text{M}a} 
%	\mean*{		
%		\int_{-1}^{1}
%		\frac{v^2}{v_{\text{t}a}^2}\vb*{v}_{\text{m} a} \cdot \nabla\psi	
%		\left(
%		\frac{B v}{\Omega_a} f_1
%		\right)
%		 \dd{\xi}
%	}
%    \dd{v}
%	\\
%	& + 
%	A_{2a}
%	\int_{0}^{\infty}
%	2\pi v^2 f_{\text{M}a}
%	\mean*{
%		\int_{-1}^{1}
%		\frac{v^2}{v_{\text{t}a}^2}\vb*{v}_{\text{m} a} \cdot \nabla\psi	
%		\left(  
%		\frac{B v}{\Omega_a}  \frac{v^2}{v_{\text{t}a}^2}f_2
%		\right)
%		 \dd{\xi}
%	}
%    \dd{v}
%	\\
%	& + 
%	A_{3a} 
%	\int_{0}^{\infty}
%	2\pi v^2	
%	f_{\text{M}a}
%	\mean*{
%		\int_{-1}^{1}
%		\frac{v^2}{v_{\text{t}a}^2}\vb*{v}_{\text{m} a} \cdot \nabla\psi	
%		\left(
%		 f_3
%		\right)
%		 \dd{\xi}
%	}
%    \dd{v}
%	\\
%	\mean*{n_a \vb*{V}_{a} \cdot\vb*{B}} & =
%	A_{1a} 
%	\int_{0}^{\infty}
%	2\pi v^2 f_{\text{M}a}
%	\mean*{
%		B
%		\int_{-1}^{1}
%		v \xi 
%		\left(
%		\frac{B v}{\Omega_a}  f_1 
%		\right)
%		 \dd{\xi}
%	}
%    \dd{v}
%    \\
%    & +
%    A_{2a}
%    \int_{0}^{\infty}
%    2\pi v^2 f_{\text{M}a}
%    \mean*{
%    	B
%    	\int_{-1}^{1}
%    	v \xi 
%    	\left(  
%    	\frac{B v}{\Omega_a}  \frac{v^2}{v_{\text{t}a}^2}f_2
%    	\right)
%    	\dd{\xi}
%    }
%    \dd{v}
%    \\
%    & +
%    A_{3a}
%    \int_{0}^{\infty}
%    2\pi v^2
%    f_{\text{M}a}
%    \mean*{
%    	B
%    	\int_{-1}^{1}
%    	v \xi 
%    	\left(
%    	 f_3
%    	\right) \dd{\xi}
%    }
%    \dd{v} 
%\end{align*}
