In this subsection we will demonstrate how the monoenergetic coefficients computed by {\MONKES} converge with $N_\theta$, $N_\zeta$ and $N_\xi$ at low collsionality for three different magnetic configurations at a single flux surface. Two of them correspond to two modes of operation of W7-X: W7-X EIM (also called standard), W7-KJM (also called high mirror). The third one corresponds to the new QI ``flat mirror'' configuration CIEMAT-QI \cite{Sanchez_2023}. The calculations are for the $1/\nu$ ($\hat{E}_r=0$) and $\sqrt{\nu}$ regimes at the low collisionality value $\hat{\nu}=10^{-5}$. In table \ref{tab:Convergence_cases} the list of cases considered is enlisted with their correspondent values of $\hat{E}_r$.
\begin{table}[]
	\centering
	\begin{tabular}{@{}lccc@{}}
		\toprule
		Configuration & $\psi/\psi_{\text{lcfs}}$ & $\hat{\nu}$ $[\text{m}^{-1}]$ & $\hat{E}_r$  $[\text{kV}\cdot\text{s}/\text{m}^2]$   \\ \midrule
		CIEMAT-QI     & 0.250                     & $10^{-5}$   & 0       \\
		CIEMAT-QI     & 0.250                     & $10^{-5}$   & $10^{-3}$       \\
		W7X-EIM       & 0.200                     & $10^{-5}$   & 0 \\
		W7X-EIM       & 0.200                     & $10^{-5}$   & $3\cdot10^{-4}$ \\
		W7X-KJM       & 0.204                     & $10^{-5}$   & 0 \\
		W7X-KJM       & 0.204                     & $10^{-5}$   & $3\cdot10^{-4}$ \\ \bottomrule
	\end{tabular}
	\caption{Cases considered to study the convergence of monoenergetic coefficients and values of $(\hat{\nu},\hat{E}_r)$.}
	\label{tab:Convergence_cases}
\end{table}
 
% The dominance of convection can be roughly estimated with the ratio $|\xi \vb*{b}\cdot\nabla f_j /\hat{\nu}\Lorentz f_j | \sim \hat{\nu}^{-1} B_{00}/(B_\zeta + \iota B_\theta)$, being more dominant whenever this ratio is bigger. On table \ref{tab:Magnetic_Configuration_Convergence_cases} we display the values of $B_{00}/(B_\zeta + \iota B_\theta)$ for each case considered. Therefore, it is expected that for CIEMAT-QI it will be more difficult to obtain converged values of the monoenergetic coefficients. 
%
%\begin{table}[]
%	\centering
%	\begin{tabular}{@{}lccc@{}}
%		\toprule
%		Configuration & $\psi/\psi_{\text{lcfs}}$ & $B_{00} $ $[\text{T}]$ & $\frac{B_{00}}{B_\zeta + \iota B_\theta} $  $[\text{m}^{-1}]$   \\ \midrule
%		CIEMAT-QI     & 0.250   & 5.4714  & 0.49138  \\
%		W7X-EIM       & 0.200   & 2.4311  & 0.17257  \\
%		W7X-KJM       & 0.204   & 2.5003  & 0.17363  \\ \bottomrule
%	\end{tabular}
%	\caption{Cases considered to study the convergence of monoenergetic coefficients and .}
%	\label{tab:Magnetic_Configuration_Convergence_cases}
%\end{table}




For each configuration we will proceed in the same manner. First, we plot the (spatially converged) coefficients $\widehat{D}_{ij}$ as functions of the number of Legendre modes in the interval $N_\xi \in [20,200]$. From the curve of $\widehat{D}_{31}$ vs $N_\xi$ we define our converged reference value, which we denote by $\widehat{D}_{31}^{\text{r}}$, as the (spatially converged) calculation for the highest number of Legendre modes, i.e. for $N_\xi=200$. From this converged reference value we define a region $\mathcal{R}_{\epsilon}:=[(1-\epsilon/100)\widehat{D}_{31}^{\text{r}}, (1+\epsilon/100)\widehat{D}_{31}^{\text{r}} ]$ of calculations that deviate less than or equal to an $\epsilon$\% with respect to $\widehat{D}_{31}^{\text{r}}$. Now we can define properly what it means to be ``sufficiently converged''. We say that, for fixed $(N_\theta,N_\zeta,N_\xi)$ and $\epsilon$, a calculation $\widehat{D}_{31}\in\mathcal{R}_{\epsilon}$ is sufficiently converged if, two conditions are satisfied:
%
\begin{enumerate}
	\item Spatially converged calculations with $N_\xi'>N_\xi$ still belong to $\mathcal{R}_{\epsilon}$.
	\item Increasing $N_\theta$ and $N_\zeta$ keeping $N_\xi$ constant produces calculations which belong to $\mathcal{R}_{\epsilon}$.
\end{enumerate}

Condition (i) is used to select the number of Legendre modes $N_\xi$ and condition (ii) is used to select the values of $N_\theta$ and $N_\zeta$ once $N_\xi$ is fixed. 

On figure \ref{fig:Convergence_W7X_EIM_Er_0} the convergence of monoenergetic coefficients with the number of Legendre modes is shown for W7-X EIM in the $1/\nu$ regime. From figure \ref{subfig:D31_Convergence_W7X_EIM_Er_0_Legendre} we can check that the calculation for $N_\xi=200$ can be safely considered as our converged reference value $\widehat{D}_{31}^{\text{r}}$. We choose the value of $\epsilon = 5 \%$ to define $\mathcal{R}_{5}$ as shown in figure \ref{subfig:D31_convergence_Legendre_W7X_EIM_0200_Erho_0_Detail}. We conclude from this figure that any value of $N_\xi\ge 80$ satisfies condition (i) of our definition of sufficiently converged. In order to not to be too indulgent with {\MONKES} calculations we select, as shown in figures \ref{subfig:D31_convergence_theta_zeta_W7X_EIM_0200_Erho_0} and \ref{subfig:D31_Clock_time_W7X_EIM_0200_Erho_0} the values $(N_\theta,N_\zeta)=(23,55)$ to satisfy condition (ii). On figure \ref{subfig:D31_Clock_time_W7X_EIM_0200_Erho_0} we plot the wall-clock time spent in {\MONKES} fast calculations of the monoenergetic coefficients, which is of 22 seconds.
%

\begin{figure*}[h]
	\centering
	\begin{subfigure}[t]{0.32\textwidth}
		\tikzsetnextfilename{Convergence-Legendre-W7X-EIM-s0200-Er-0-D11}
		\begin{tikzpicture}
	\begin{axis}[
		%		height=0.85\textwidth, 
		width=\textwidth, 
		scaled y ticks=base 10:2,
		y tick label style={
			/pgf/number format/.cd,
			fixed,
			fixed zerofill,
			precision=1,
			/tikz/.cd}, 
		xlabel = $N_\xi$, ylabel=$\widehat{D}_{11} $ ${[\text{m}]}$
		]
		
		
		\foreach \Nxi in {20, 40, ...,100}{		
			\addplot[blue, mark=+, only marks] table[skip first n=1, 
			x expr=\thisrowno{4},
			y expr=\thisrowno{5} * 0.5237 * 0.5237			
			,
			restrict expr to domain={\thisrowno{2}}{15:47},
			restrict expr to domain={\thisrowno{3}}{65:179},
			restrict expr to domain={\thisrowno{4}}{\Nxi:\Nxi}
			]{data/W7X-EIM/MONKES/DKE_zeta_Convergence_Example_Nxi_20/Gamma_11_Gamma_31_nu_0.100E-04_E_rho_0.000E+00.plt};
		}
		
		\foreach \Nxi in {120, 140, ...,200}{		
			\addplot[blue, mark=+, only marks] table[skip first n=1, 
			x expr=\thisrowno{4},
			y expr=\thisrowno{5} * 0.5237 * 0.5237,
			restrict expr to domain={\thisrowno{2}}{19:47},
			restrict expr to domain={\thisrowno{3}}{65:179},
			restrict expr to domain={\thisrowno{4}}{\Nxi:\Nxi}
			]{data/W7X-EIM/MONKES/DKE_zeta_Convergence_Example_Nxi_120/Gamma_11_Gamma_31_nu_0.100E-04_E_rho_0.000E+00.plt};
		}	
		
		
		\addplot[forget plot,blue, mark=+, only marks] table[skip first n=1, 
		x expr=\thisrowno{4},
		y expr=\thisrowno{5}*0.5237*0.5237,
		restrict expr to domain={\thisrowno{1}}{0:0},
		restrict expr to domain={\thisrowno{4}}{220:380}
		]{data/W7X-EIM/MONKES/Convergence_Nxi/N_theta_47_N_zeta_153/monkes_Monoenergetic_Database.dat};
	\end{axis}
\end{tikzpicture}
%
		\caption{}
		\label{subfig:D11_Convergence_W7X_EIM_Er_0_Legendre}
	\end{subfigure}
	%\hfill
	%	%\hfill
	\begin{subfigure}[t]{0.32\textwidth}
		\tikzsetnextfilename{Convergence-Legendre-W7X-EIM-s0200-Er-0-D33}
		\begin{tikzpicture}
	\begin{axis}[
		%		height=0.85\textwidth, 
		width=\textwidth, 
		scaled y ticks=base 10:-4,		
		y tick label style={
			/pgf/number format/.cd,
			fixed,
			fixed zerofill,
			precision=1,			
			/tikz/.cd}, 
		xlabel = $N_\xi$, ylabel=$\widehat{D}_{33} $ ${[\text{m}]}$
		]
			
		\foreach \Nxi in {20, 40, ...,100}{		
			\addplot[blue, mark=+, only marks] table[skip first n=1, 
			x expr=\thisrowno{4},
			y expr=\thisrowno{8},
			restrict expr to domain={\thisrowno{2}}{19:47},
			restrict expr to domain={\thisrowno{3}}{65:179},
			restrict expr to domain={\thisrowno{4}}{\Nxi:\Nxi}
			]{results/W7X-EIM/0.200/DKE_zeta_Convergence_Example_Nxi_20/Gamma_11_Gamma_31_nu_0.100E-04_E_rho_0.000E+00.plt};
		}
		
		\foreach \Nxi in {120, 140, ...,200}{		
			\addplot[blue, mark=+, only marks] table[skip first n=1, 
			x expr=\thisrowno{4},
			y expr=\thisrowno{8},
			restrict expr to domain={\thisrowno{2}}{27:47},
			restrict expr to domain={\thisrowno{3}}{65:179},
			restrict expr to domain={\thisrowno{4}}{\Nxi:\Nxi}
			]{results/W7X-EIM/0.200/DKE_zeta_Convergence_Example_Nxi_120/Gamma_11_Gamma_31_nu_0.100E-04_E_rho_0.000E+00.plt};
		}	
				
		\addplot[forget plot,blue, mark=+, only marks] table[skip first n=1, 
		x expr=\thisrowno{4},
		y expr=-\thisrowno{8},
		restrict expr to domain={\thisrowno{1}}{0:0},
		restrict expr to domain={\thisrowno{4}}{220:380}
		]{results/W7X-EIM/0.200/Convergence_Nxi/N_theta_47_N_zeta_153/monkes_Monoenergetic_Database.dat};
		
	\end{axis}
\end{tikzpicture}
%
		\caption{}
		\label{subfig:D33_Convergence_W7X_EIM_Er_0_Legendre}
	\end{subfigure}
    
    \begin{subfigure}[t]{0.32\textwidth}
    	\tikzsetnextfilename{Convergence-Legendre-W7X-EIM-s0200-Er-0-D31-Detail}
    	\begin{tikzpicture}
	\begin{axis}[
		%		height=0.85\textwidth, 
		width=\textwidth, 
		scaled y ticks=base 10:1,
%		extra y ticks ={3.64E-1*2.4311*0.5237}, 
		y tick label style={
			/pgf/number format/.cd,
			fixed,
			fixed zerofill,
			precision=1,
			/tikz/.cd}, 
		xlabel = $N_\xi$, ylabel=$\widehat{D}_{31}$ ${[\text{m}]}$,
		ymax = 0.46*0.5237, 
		legend pos = north west,
		legend columns = 2
		]
		
		
			
		
		\foreach \Nxi in {40,60,80,100}
		{			
					
			\addplot[forget plot, blue, mark=+, only marks] table[skip first n=1, 
			x expr=\thisrowno{4},
			y expr=\thisrowno{6}*0.5237,
			restrict expr to domain={\thisrowno{2}}{27:47},
			restrict expr to domain={\thisrowno{3}}{65:179},
			restrict expr to domain={\thisrowno{4}}{\Nxi:\Nxi}
			]{results/W7X-EIM/0.200/DKE_zeta_Convergence_Example_Nxi_20/Gamma_11_Gamma_31_nu_0.100E-04_E_rho_0.000E+00.plt};
			
			\addplot[forget plot, blue, mark=o, only marks] table[skip first n=1, 
			x expr=\thisrowno{4},
			y expr=-\thisrowno{7}*0.5237,
			restrict expr to domain={\thisrowno{2}}{27:47},
			restrict expr to domain={\thisrowno{3}}{65:179},
			restrict expr to domain={\thisrowno{4}}{\Nxi:\Nxi}
			]{results/W7X-EIM/0.200/DKE_zeta_Convergence_Example_Nxi_20/Gamma_11_Gamma_31_nu_0.100E-04_E_rho_0.000E+00.plt};
			
		}
		
		\foreach \Nxi in {120, 160, 180}{		
			\addplot[forget plot, blue, mark=+, only marks] table[skip first n=1, 
			x expr=\thisrowno{4},
			y expr=\thisrowno{6}*0.5237,
			restrict expr to domain={\thisrowno{2}}{47:67},
			restrict expr to domain={\thisrowno{3}}{71:179},
			restrict expr to domain={\thisrowno{4}}{\Nxi:\Nxi}
			]{results/W7X-EIM/0.200/DKE_zeta_Convergence_Example_Nxi_120/Gamma_11_Gamma_31_nu_0.100E-04_E_rho_0.000E+00.plt};
			
					
			\addplot[forget plot, blue, mark=o, only marks] table[skip first n=1, 
			x expr=\thisrowno{4},
			y expr=-\thisrowno{7}*0.5237,
			restrict expr to domain={\thisrowno{2}}{47:67},
			restrict expr to domain={\thisrowno{3}}{71:179},
			restrict expr to domain={\thisrowno{4}}{\Nxi:\Nxi}
			]{results/W7X-EIM/0.200/DKE_zeta_Convergence_Example_Nxi_120/Gamma_11_Gamma_31_nu_0.100E-04_E_rho_0.000E+00.plt};
			
		}	
		
		
		\foreach \Nxi in {200}{		
			\addplot[blue, mark=+, only marks] table[skip first n=1, 
			x expr=\thisrowno{4},
			y expr=\thisrowno{6}*0.5237,
			restrict expr to domain={\thisrowno{2}}{47:67},
			restrict expr to domain={\thisrowno{3}}{71:179},
			restrict expr to domain={\thisrowno{4}}{\Nxi:\Nxi}
			]{results/W7X-EIM/0.200/DKE_zeta_Convergence_Example_Nxi_120/Gamma_11_Gamma_31_nu_0.100E-04_E_rho_0.000E+00.plt};
			
			\addlegendentry{$\widehat{D}_{31}$}
			
			\addplot[blue, mark=o, only marks] table[skip first n=1, 
			x expr=\thisrowno{4},
			y expr=-\thisrowno{7}*0.5237,
			restrict expr to domain={\thisrowno{2}}{47:67},
			restrict expr to domain={\thisrowno{3}}{71:179},
			restrict expr to domain={\thisrowno{4}}{\Nxi:\Nxi}
			]{results/W7X-EIM/0.200/DKE_zeta_Convergence_Example_Nxi_120/Gamma_11_Gamma_31_nu_0.100E-04_E_rho_0.000E+00.plt};
			\addlegendentry{$-\widehat{D}_{13}$}
			
		}
		
		\addplot[forget plot,blue, mark=+, only marks] table[skip first n=1, 
		x expr=\thisrowno{4},
		y expr=\thisrowno{6}*0.5237,
		restrict expr to domain={\thisrowno{1}}{0:0},
		restrict expr to domain={\thisrowno{4}}{220:380}
		]{results/W7X-EIM/0.200/Convergence_Nxi/N_theta_47_N_zeta_153/monkes_Monoenergetic_Database.dat};
		
		\addplot[forget plot,blue, mark=o, only marks] table[skip first n=1, 
		x expr=\thisrowno{4},
		y expr=-\thisrowno{7}*0.5237,
		restrict expr to domain={\thisrowno{1}}{0:0},
		restrict expr to domain={\thisrowno{4}}{220:380}
		]{results/W7X-EIM/0.200/Convergence_Nxi/N_theta_47_N_zeta_153/monkes_Monoenergetic_Database.dat};	
		
		\addplot[forget plot, name path=Upper2,blue!20, 
		domain = 40:380] {0.364E+00*1.05*0.5237};	
		\addplot[forget plot, name path=Lower2,blue!20, 
		domain = 40:380] {0.364E+00*0.95*0.5237};		
		\addplot[blue!20] fill between[of=Upper2 and Lower2];
		\addlegendentry{ $\mathcal{R}_5$ }	
		
		
		\addplot[forget plot, name path=Upper2,red!20, 
		domain = 40:380] {(0.364E+00)*0.5237+0.005};	
		\addplot[forget plot, name path=Lower2,red!20, 
		domain = 40:380] {(0.364E+00)*0.5237-0.005};		
		\addplot[red!20] fill between[of=Upper2 and Lower2];
		\addlegendentry{ $\mathcal{A}_{0.005}$ }
			
		
		\addplot[ultra thick, Green, mark = star, mark size = 5 pt, only marks] table[skip first n=1, 
		x expr=\thisrowno{4},
		y expr=\thisrowno{6}*0.5237,
		restrict expr to domain={\thisrowno{2}}{47:67},
		restrict expr to domain={\thisrowno{3}}{71:179},
		restrict expr to domain={\thisrowno{4}}{140:140}
		]{results/W7X-EIM/0.200/DKE_zeta_Convergence_Example_Nxi_120/Gamma_11_Gamma_31_nu_0.100E-04_E_rho_0.000E+00.plt};
		\addlegendentry{Selected}
	\end{axis}
\end{tikzpicture}%
    	\caption{}
    	\label{subfig:D31_convergence_Legendre_W7X_EIM_0200_Erho_0_Detail}
    \end{subfigure}
    %\hfill
    \begin{subfigure}[t]{0.32\textwidth}
    	\tikzsetnextfilename{Convergence-theta-zeta-W7X-EIM-s0200-Er-0-D31}
    	\begin{tikzpicture}
	\begin{axis}[
		%		height=0.85\textwidth, 
		width=\textwidth, 
		scaled y ticks=base 10:1,
%		xtick={80,120,160,200, 240},
		y tick label style={
			/pgf/number format/.cd,
			fixed,
			fixed zerofill,
			precision=1,
			/tikz/.cd}, 
		ymin = 0.325/2.4311,
%		ymax = 0.102,
		legend pos = south east, 
		legend columns =2, 
		xlabel = $N_\zeta$, ylabel=$\widehat{D}_{31}$ ${[\text{m}]}$
		]
		
		
		\addplot[forget plot, name path=Upper2,blue!20, 
		domain = 15:131] {0.364E+00*1.05*0.5237};	
		\addplot[forget plot, name path=Lower2,blue!20, 
		domain = 15:131] {0.364E+00*0.95*0.5237};		
		\addplot[forget plot, blue!20] fill between[of=Upper2 and Lower2];
		
		
		\addplot[forget plot, name path=Upper2,red!20, 
		domain = 15:131] {(0.364E+00+0.01)*0.5237};	
		\addplot[forget plot, name path=Lower2,red!20, 
		domain = 15:131] {(0.364E+00-0.01)*0.5237};		
		\addplot[forget plot, red!20] fill between[of=Upper2 and Lower2];
%		\addlegendentry{ $\mathcal{A}_{0.01}$ }
		
		\foreach \Ntheta in {15,19,...,27}
		{
			\addplot+[no markers] table[skip first n=1, 
			x expr=\thisrowno{3},
			y expr=\thisrowno{6}*0.5237,
			restrict expr to domain={\thisrowno{2}}{\Ntheta:\Ntheta},
			restrict expr to domain={\thisrowno{3}}{27:131},
			restrict expr to domain={\thisrowno{4}}{140:140}
			]{results/W7X-EIM/0.200/Convergence_nu_1e-5_Er_0/Nxi_140/monkes_Monoenergetic_Database.dat};
			
			\addlegendimage{empty legend}
			\addlegendentry{$N_\theta=$}
			\expandafter\addlegendentry\expandafter{\Ntheta}
		}	
		
		
		\foreach \Ntheta in {27}{		
			\addplot+[ultra thick, Green, mark = star, mark size = 5 pt, only marks] table[skip first n=1, 
			x expr=\thisrowno{3},
			y expr=\thisrowno{6}*0.5237,
			restrict expr to domain={\thisrowno{2}}{\Ntheta:\Ntheta},
			restrict expr to domain={\thisrowno{3}}{55:55},
			restrict expr to domain={\thisrowno{4}}{140:140}
			]{results/W7X-EIM/0.200/Convergence_nu_1e-5_Er_0/Nxi_140/monkes_Monoenergetic_Database.dat};
		}
%		\foreach \Ntheta in {15}{		
%			\addplot+[only marks, mark = oplus, mark size = 4 pt, blue ] table[skip first n=1, 
%			x expr=\thisrowno{3},
%			y expr=\thisrowno{6},
%			restrict expr to domain={\thisrowno{2}}{\Ntheta:\Ntheta},
%			restrict expr to domain={\thisrowno{3}}{119:119}
%			]{results/CIEMAT-QI/0.250/DKE_zeta_Convergence_Example_Nxi_180/Monoenergetic_nu_0.100E-04_E_rho_0.000E+00.plt};
			
%		}	
		%		\addlegendentry{Spread of 5\%}
		
		
	\end{axis}
\end{tikzpicture}

    	\caption{}
    	\label{subfig:D31_convergence_theta_zeta_W7X_EIM_0200_Erho_0}
    \end{subfigure}
	\caption{Convergence of monoenergetic coefficients with the number of Legendre modes $N_\xi$ for W7X-EIM at the surface labelled by $\psi/\psi_{\text{lcfs}}=0.200$, for $\hat{\nu}(v)=10^{-5}$ $\text{m}^{-1}$ and $\widehat{E}_r(v)=0$ $\text{kV}\cdot\text{s}/\text{m}^2$.}
	\label{fig:Convergence_W7X_EIM_Er_0}
\end{figure*}
\begin{figure}[]
	\centering
	\begin{subfigure}[t]{0.33\textwidth}
		\tikzsetnextfilename{Convergence-Legendre-W7X-EIM-s0200-Er-0-D31-Detail}
		\begin{tikzpicture}
	\begin{axis}[
		%		height=0.85\textwidth, 
		width=\textwidth, 
		scaled y ticks=base 10:1,
%		extra y ticks ={3.64E-1*2.4311*0.5237}, 
		y tick label style={
			/pgf/number format/.cd,
			fixed,
			fixed zerofill,
			precision=1,
			/tikz/.cd}, 
		xlabel = $N_\xi$, ylabel=$\widehat{D}_{31}$ ${[\text{m}]}$,
		ymax = 0.46*0.5237, 
		legend pos = north west,
		legend columns = 2
		]
		
		
			
		
		\foreach \Nxi in {40,60,80,100}
		{			
					
			\addplot[forget plot, blue, mark=+, only marks] table[skip first n=1, 
			x expr=\thisrowno{4},
			y expr=\thisrowno{6}*0.5237,
			restrict expr to domain={\thisrowno{2}}{27:47},
			restrict expr to domain={\thisrowno{3}}{65:179},
			restrict expr to domain={\thisrowno{4}}{\Nxi:\Nxi}
			]{results/W7X-EIM/0.200/DKE_zeta_Convergence_Example_Nxi_20/Gamma_11_Gamma_31_nu_0.100E-04_E_rho_0.000E+00.plt};
			
			\addplot[forget plot, blue, mark=o, only marks] table[skip first n=1, 
			x expr=\thisrowno{4},
			y expr=-\thisrowno{7}*0.5237,
			restrict expr to domain={\thisrowno{2}}{27:47},
			restrict expr to domain={\thisrowno{3}}{65:179},
			restrict expr to domain={\thisrowno{4}}{\Nxi:\Nxi}
			]{results/W7X-EIM/0.200/DKE_zeta_Convergence_Example_Nxi_20/Gamma_11_Gamma_31_nu_0.100E-04_E_rho_0.000E+00.plt};
			
		}
		
		\foreach \Nxi in {120, 160, 180}{		
			\addplot[forget plot, blue, mark=+, only marks] table[skip first n=1, 
			x expr=\thisrowno{4},
			y expr=\thisrowno{6}*0.5237,
			restrict expr to domain={\thisrowno{2}}{47:67},
			restrict expr to domain={\thisrowno{3}}{71:179},
			restrict expr to domain={\thisrowno{4}}{\Nxi:\Nxi}
			]{results/W7X-EIM/0.200/DKE_zeta_Convergence_Example_Nxi_120/Gamma_11_Gamma_31_nu_0.100E-04_E_rho_0.000E+00.plt};
			
					
			\addplot[forget plot, blue, mark=o, only marks] table[skip first n=1, 
			x expr=\thisrowno{4},
			y expr=-\thisrowno{7}*0.5237,
			restrict expr to domain={\thisrowno{2}}{47:67},
			restrict expr to domain={\thisrowno{3}}{71:179},
			restrict expr to domain={\thisrowno{4}}{\Nxi:\Nxi}
			]{results/W7X-EIM/0.200/DKE_zeta_Convergence_Example_Nxi_120/Gamma_11_Gamma_31_nu_0.100E-04_E_rho_0.000E+00.plt};
			
		}	
		
		
		\foreach \Nxi in {200}{		
			\addplot[blue, mark=+, only marks] table[skip first n=1, 
			x expr=\thisrowno{4},
			y expr=\thisrowno{6}*0.5237,
			restrict expr to domain={\thisrowno{2}}{47:67},
			restrict expr to domain={\thisrowno{3}}{71:179},
			restrict expr to domain={\thisrowno{4}}{\Nxi:\Nxi}
			]{results/W7X-EIM/0.200/DKE_zeta_Convergence_Example_Nxi_120/Gamma_11_Gamma_31_nu_0.100E-04_E_rho_0.000E+00.plt};
			
			\addlegendentry{$\widehat{D}_{31}$}
			
			\addplot[blue, mark=o, only marks] table[skip first n=1, 
			x expr=\thisrowno{4},
			y expr=-\thisrowno{7}*0.5237,
			restrict expr to domain={\thisrowno{2}}{47:67},
			restrict expr to domain={\thisrowno{3}}{71:179},
			restrict expr to domain={\thisrowno{4}}{\Nxi:\Nxi}
			]{results/W7X-EIM/0.200/DKE_zeta_Convergence_Example_Nxi_120/Gamma_11_Gamma_31_nu_0.100E-04_E_rho_0.000E+00.plt};
			\addlegendentry{$-\widehat{D}_{13}$}
			
		}
		
		\addplot[forget plot,blue, mark=+, only marks] table[skip first n=1, 
		x expr=\thisrowno{4},
		y expr=\thisrowno{6}*0.5237,
		restrict expr to domain={\thisrowno{1}}{0:0},
		restrict expr to domain={\thisrowno{4}}{220:380}
		]{results/W7X-EIM/0.200/Convergence_Nxi/N_theta_47_N_zeta_153/monkes_Monoenergetic_Database.dat};
		
		\addplot[forget plot,blue, mark=o, only marks] table[skip first n=1, 
		x expr=\thisrowno{4},
		y expr=-\thisrowno{7}*0.5237,
		restrict expr to domain={\thisrowno{1}}{0:0},
		restrict expr to domain={\thisrowno{4}}{220:380}
		]{results/W7X-EIM/0.200/Convergence_Nxi/N_theta_47_N_zeta_153/monkes_Monoenergetic_Database.dat};	
		
		\addplot[forget plot, name path=Upper2,blue!20, 
		domain = 40:380] {0.364E+00*1.05*0.5237};	
		\addplot[forget plot, name path=Lower2,blue!20, 
		domain = 40:380] {0.364E+00*0.95*0.5237};		
		\addplot[blue!20] fill between[of=Upper2 and Lower2];
		\addlegendentry{ $\mathcal{R}_5$ }	
		
		
		\addplot[forget plot, name path=Upper2,red!20, 
		domain = 40:380] {(0.364E+00)*0.5237+0.005};	
		\addplot[forget plot, name path=Lower2,red!20, 
		domain = 40:380] {(0.364E+00)*0.5237-0.005};		
		\addplot[red!20] fill between[of=Upper2 and Lower2];
		\addlegendentry{ $\mathcal{A}_{0.005}$ }
			
		
		\addplot[ultra thick, Green, mark = star, mark size = 5 pt, only marks] table[skip first n=1, 
		x expr=\thisrowno{4},
		y expr=\thisrowno{6}*0.5237,
		restrict expr to domain={\thisrowno{2}}{47:67},
		restrict expr to domain={\thisrowno{3}}{71:179},
		restrict expr to domain={\thisrowno{4}}{140:140}
		]{results/W7X-EIM/0.200/DKE_zeta_Convergence_Example_Nxi_120/Gamma_11_Gamma_31_nu_0.100E-04_E_rho_0.000E+00.plt};
		\addlegendentry{Selected}
	\end{axis}
\end{tikzpicture}
		\caption{}
		\label{subfig:D31_convergence_Legendre_W7X_EIM_0200_Erho_0_Detail}
	\end{subfigure}
	%\hfill
	\begin{subfigure}[t]{0.33\textwidth}
		\tikzsetnextfilename{Convergence-theta-zeta-W7X-EIM-s0200-Er-0-D31}
		\begin{tikzpicture}
	\begin{axis}[
		%		height=0.85\textwidth, 
		width=\textwidth, 
		scaled y ticks=base 10:1,
%		xtick={80,120,160,200, 240},
		y tick label style={
			/pgf/number format/.cd,
			fixed,
			fixed zerofill,
			precision=1,
			/tikz/.cd}, 
		ymin = 0.325/2.4311,
%		ymax = 0.102,
		legend pos = south east, 
		legend columns =2, 
		xlabel = $N_\zeta$, ylabel=$\widehat{D}_{31}$ ${[\text{m}]}$
		]
		
		
		\addplot[forget plot, name path=Upper2,blue!20, 
		domain = 15:131] {0.364E+00*1.05*0.5237};	
		\addplot[forget plot, name path=Lower2,blue!20, 
		domain = 15:131] {0.364E+00*0.95*0.5237};		
		\addplot[forget plot, blue!20] fill between[of=Upper2 and Lower2];
		
		
		\addplot[forget plot, name path=Upper2,red!20, 
		domain = 15:131] {(0.364E+00+0.01)*0.5237};	
		\addplot[forget plot, name path=Lower2,red!20, 
		domain = 15:131] {(0.364E+00-0.01)*0.5237};		
		\addplot[forget plot, red!20] fill between[of=Upper2 and Lower2];
%		\addlegendentry{ $\mathcal{A}_{0.01}$ }
		
		\foreach \Ntheta in {15,19,...,27}
		{
			\addplot+[no markers] table[skip first n=1, 
			x expr=\thisrowno{3},
			y expr=\thisrowno{6}*0.5237,
			restrict expr to domain={\thisrowno{2}}{\Ntheta:\Ntheta},
			restrict expr to domain={\thisrowno{3}}{27:131},
			restrict expr to domain={\thisrowno{4}}{140:140}
			]{results/W7X-EIM/0.200/Convergence_nu_1e-5_Er_0/Nxi_140/monkes_Monoenergetic_Database.dat};
			
			\addlegendimage{empty legend}
			\addlegendentry{$N_\theta=$}
			\expandafter\addlegendentry\expandafter{\Ntheta}
		}	
		
		
		\foreach \Ntheta in {27}{		
			\addplot+[ultra thick, Green, mark = star, mark size = 5 pt, only marks] table[skip first n=1, 
			x expr=\thisrowno{3},
			y expr=\thisrowno{6}*0.5237,
			restrict expr to domain={\thisrowno{2}}{\Ntheta:\Ntheta},
			restrict expr to domain={\thisrowno{3}}{55:55},
			restrict expr to domain={\thisrowno{4}}{140:140}
			]{results/W7X-EIM/0.200/Convergence_nu_1e-5_Er_0/Nxi_140/monkes_Monoenergetic_Database.dat};
		}
%		\foreach \Ntheta in {15}{		
%			\addplot+[only marks, mark = oplus, mark size = 4 pt, blue ] table[skip first n=1, 
%			x expr=\thisrowno{3},
%			y expr=\thisrowno{6},
%			restrict expr to domain={\thisrowno{2}}{\Ntheta:\Ntheta},
%			restrict expr to domain={\thisrowno{3}}{119:119}
%			]{results/CIEMAT-QI/0.250/DKE_zeta_Convergence_Example_Nxi_180/Monoenergetic_nu_0.100E-04_E_rho_0.000E+00.plt};
			
%		}	
		%		\addlegendentry{Spread of 5\%}
		
		
	\end{axis}
\end{tikzpicture}

		\caption{}
		\label{subfig:D31_convergence_theta_zeta_W7X_EIM_0200_Erho_0}
	\end{subfigure}
	%	%\hfill
	\begin{subfigure}[t]{0.33\textwidth}
		\tikzsetnextfilename{Clock-time-W7X-EIM-s0200-Er-0-D31}
		\begin{tikzpicture}
	\begin{axis}[
		%		height=0.85\textwidth, 
		width=0.97\textwidth, 
		xtick = {20,40,80},
		extra x ticks={0.55E+02},
		extra y ticks={0.2245864600000004E+02},
		extra tick style={grid=major, grid style={dashed,black}}, 
%		scaled y ticks=base 10:-1,
		%		xtick={80,120,160,200, 240},
		xmax = 100, 
		y tick label style={
			/pgf/number format/.cd,
			fixed,
			fixed zerofill,
			precision=0,
			/tikz/.cd}, 
		%		ymin = 0.053,
		%		ymax = 0.102,
		legend pos = south east, 
		legend columns =2, 
		xlabel = $N_\zeta$, ylabel=Wall-clock time {[s]}
		]
		
		
%		\addplot[forget plot, name path=Upper2,blue!20, 
%		domain = 15:71] {0.3593392335766615E+00*1.05};	
%		\addplot[forget plot, name path=Lower2,blue!20, 
%		domain = 15:71] {0.3593392335766615E+00*0.95};		
%		\addplot[forget plot, blue!20] fill between[of=Upper2 and Lower2];
		
		\foreach \Ntheta in {23}{		
			\addplot+[Brown,no markers] table[skip first n=1, 
			x expr=\thisrowno{3},
			y expr=\thisrowno{10},
			restrict expr to domain={\thisrowno{2}}{\Ntheta:\Ntheta},
			restrict expr to domain={\thisrowno{3}}{21:131},
			restrict expr to domain={\thisrowno{4}}{140:140}
			]{results/W7X-EIM/0.200/Convergence_nu_1e-5_Er_0/Nxi_140/monkes_Monoenergetic_Database.dat};
			\addlegendimage{empty legend}
			\addlegendentry{$N_\theta=$}
			\expandafter\addlegendentry\expandafter{\Ntheta}
		}	
		
		
		
		\foreach \Ntheta in {23}{		
			\addplot+[ultra thick, Green, mark = star, mark size = 5 pt, only marks] table[skip first n=1, 
			x expr=\thisrowno{3},
			y expr=\thisrowno{10},
			restrict expr to domain={\thisrowno{2}}{\Ntheta:\Ntheta},
			restrict expr to domain={\thisrowno{3}}{55:55},
			restrict expr to domain={\thisrowno{4}}{140:140}
			]{results/W7X-EIM/0.200/Convergence_nu_1e-5_Er_0/Nxi_140/monkes_Monoenergetic_Database.dat};
		}
		
	\end{axis}
\end{tikzpicture}

		\caption{}
		\label{subfig:D31_Clock_time_W7X_EIM_0200_Erho_0}
	\end{subfigure}
	\caption{Selection of the resolution to have a sufficiently accurate calculation of the parallel flow geometric coefficient $\widehat{D}_{31}$ for W7X-EIM at the surface labelled by $\psi/\psi_{\text{lcfs}}=0.200$, for $\hat{\nu}(v)=10^{-5}$ $\text{m}^{-1}$ and $\hat{E}_r(v)=0$ $\text{kV}\cdot\text{s}/\text{m}^2$. }
	\label{fig:Convergence_W7X_EIM_Er_0_Detail}
\end{figure}


For the $\sqrt{\nu}$ regime case of W7-X EIM on figure \ref{subfig:D31_convergence_Legendre_W7X_EIM_0200_Erho_3e-4} we check again that the calculation for $N_\xi=200$ serves as a good converged reference value $\widehat{D}_{31}^{\text{r}}$. 
%
\begin{figure*}[]
	\centering
	\begin{subfigure}[t]{0.32\textwidth}
		\tikzsetnextfilename{Convergence-Legendre-W7X-EIM-s0200-Er-3e-4-D11}
		
\begin{tikzpicture}
	\begin{axis}[
		%		height=0.85\textwidth, 
		width=0.95\textwidth, 
		y tick label style={
			/pgf/number format/.cd,
			fixed,
			fixed zerofill,
			precision=2,
			/tikz/.cd}, 
		xlabel = $N_\xi$, ylabel=$\Gamma_{11}$
		]
		\addplot[blue, mark=*, only marks] table[skip first n=1, 
		x expr=\thisrowno{4},
		y expr=\thisrowno{5},
		restrict expr to domain={\thisrowno{2}}{25:47},
		restrict expr to domain={\thisrowno{3}}{55:97}
		]{results/W7X-EIM/0.200/DKE_zeta_Convergence_Example_Nxi_20/Gamma_11_Gamma_31_nu_0.100E-04_E_rho_0.300E-03.plt};
		%
		\addplot[blue, mark=*, only marks] table[skip first n=1, 
		x expr=\thisrowno{4},
		y expr=\thisrowno{5},
		restrict expr to domain={\thisrowno{2}}{25:47},
		restrict expr to domain={\thisrowno{3}}{55:97}
		]{results/W7X-EIM/0.200/DKE_zeta_Convergence_Example_Nxi_120/Gamma_11_Gamma_31_nu_0.100E-04_E_rho_0.300E-03.plt};			
	\end{axis}
\end{tikzpicture}
%
		\caption{}
		\label{subfig:D11_convergence_Legendre_W7X_EIM_0200_Erho_3e-4}
	\end{subfigure}
	%\hfill
	%\hfill
	\begin{subfigure}[t]{0.32\textwidth}
		\tikzsetnextfilename{Convergence-Legendre-W7X-EIM-s0200-Er-3e-4-D33}
		\begin{tikzpicture}
	\begin{axis}[
		%		height=0.85\textwidth, 
		width=\textwidth, 
		scaled y ticks=base 10:-4,
		y tick label style={
			/pgf/number format/.cd,
			fixed,
			fixed zerofill,
			precision=1,
			/tikz/.cd}, 
		xlabel = $N_\xi$, ylabel=$\widehat{D}_{33}$ ${[\text{m}]}$
		]
			
		
		\foreach \Nxi in {20, 40, ...,100}{		
			\addplot[blue, mark=+, only marks] table[skip first n=1, 
			x expr=\thisrowno{4},
			y expr=\thisrowno{8},
			restrict expr to domain={\thisrowno{2}}{19:47},
			restrict expr to domain={\thisrowno{3}}{65:179},
			restrict expr to domain={\thisrowno{4}}{\Nxi:\Nxi}
			]{data/W7X-EIM/MONKES/DKE_zeta_Convergence_Example_Nxi_20/Gamma_11_Gamma_31_nu_0.100E-04_E_rho_0.300E-03.plt};
		}
		
		\foreach \Nxi in {120, 140, ...,200}{		
			\addplot[blue, mark=+, only marks] table[skip first n=1, 
			x expr=\thisrowno{4},
			y expr=\thisrowno{8},
			restrict expr to domain={\thisrowno{2}}{27:47},
			restrict expr to domain={\thisrowno{3}}{65:179},
			restrict expr to domain={\thisrowno{4}}{\Nxi:\Nxi}
			]{data/W7X-EIM/MONKES/DKE_zeta_Convergence_Example_Nxi_120/Gamma_11_Gamma_31_nu_0.100E-04_E_rho_0.300E-03.plt};
		}	
		
		\addplot[forget plot,blue, mark=+, only marks] table[skip first n=1, 
		x expr=\thisrowno{4},
		y expr=-\thisrowno{8},
		restrict expr to domain={\thisrowno{1}}{3e-4:3e-4},
		restrict expr to domain={\thisrowno{2}}{47:67},
		restrict expr to domain={\thisrowno{3}}{71:179},
		restrict expr to domain={\thisrowno{4}}{220:300}
		]{data/W7X-EIM/MONKES/Convergence_Nxi/N_theta_47_N_zeta_153/monkes_Monoenergetic_Database.dat};
		
	\end{axis}
\end{tikzpicture}
%
		\caption{}
		\label{subfig:D33_convergence_Legendre_W7X_EIM_0200_Erho_3e-4}
	\end{subfigure}

    
    \begin{subfigure}[t]{0.32\textwidth}
    	\tikzsetnextfilename{Convergence-Legendre-W7X-EIM-s0200-Er-3e-4-D31-Detail}
    	\begin{tikzpicture}
	\begin{axis}[
		%		height=0.85\textwidth, 
		width=\textwidth, 
		scaled y ticks=base 10:1,
		y tick label style={
			/pgf/number format/.cd,
			fixed,
			fixed zerofill,
			precision=1,
			/tikz/.cd}, 
		xlabel = $N_\xi$, ylabel=$\widehat{D}_{31}$ ${[\text{m}]}$,
		legend pos = north west,
%		ymin = 0.16,
		ymax = 0.4*0.5237,
		legend columns = 2
		]
		
		\addplot[blue, mark=+, only marks] table[skip first n=1, 
		x expr=\thisrowno{4},
		y expr=\thisrowno{6}*0.5237,
		restrict expr to domain={\thisrowno{1}}{3e-4:3e-4},
		restrict expr to domain={\thisrowno{2}}{47:67},
		restrict expr to domain={\thisrowno{3}}{71:179},
		restrict expr to domain={\thisrowno{4}}{220:300}
		]{data/W7X-EIM/MONKES/Convergence_Nxi/N_theta_47_N_zeta_153/monkes_Monoenergetic_Database.dat};
		\addlegendentry{$\widehat{D}_{31}$}
		
		\addplot[blue, mark=o, only marks] table[skip first n=1, 
		x expr=\thisrowno{4},
		y expr=-\thisrowno{7}*0.5237,
		restrict expr to domain={\thisrowno{1}}{3e-4:3e-4},
		restrict expr to domain={\thisrowno{2}}{47:67},
		restrict expr to domain={\thisrowno{3}}{71:179},
		restrict expr to domain={\thisrowno{4}}{220:300}
		]{data/W7X-EIM/MONKES/Convergence_Nxi/N_theta_47_N_zeta_153/monkes_Monoenergetic_Database.dat};
		\addlegendentry{$-\widehat{D}_{13}$}
		
		\addplot[forget plot, name path=Upper2,blue!20, 
		domain = 40:300] {0.241E+00*1.05*0.5237};	
		\addplot[forget plot, name path=Lower2,blue!20, 
		domain = 40:300] {0.241E+00*0.95*0.5237};		
		\addplot[blue!20] fill between[of=Upper2 and Lower2];
		\addlegendentry{ $\mathcal{R}_5$ }
		
		\addplot[forget plot, name path=Upper2,red!20, 
		domain = 40:300] {(0.241E+00*0.5237+0.005)};
		\addplot[forget plot, name path=Lower2,red!20, 
		domain = 40:300] {(0.241E+00*0.5237-0.005)};		
		\addplot[red!20] fill between[of=Upper2 and Lower2];
		\addlegendentry{ $\mathcal{A}_{0.005}$ }
		
		
%		\addplot[forget plot, name path=Upper2,red!20, 
%		domain = 40:200] {0.2365291332172711E+00+0.01};	
%		\addplot[forget plot, name path=Lower2,red!20, 
%		domain = 40:200] {0.2365291332172711E+00-0.01};		
%		\addplot[red!20] fill between[of=Upper2 and Lower2];	
%		\addlegendentry{ $\pm 0.01$ }	
		
		\foreach \Nxi in {40,60,80,100}{		
			\addplot[forget plot, blue, mark=+, only marks] table[skip first n=1, 
			x expr=\thisrowno{4},
			y expr=\thisrowno{6}*0.5237,
			restrict expr to domain={\thisrowno{2}}{27:47},
			restrict expr to domain={\thisrowno{3}}{65:179},
			restrict expr to domain={\thisrowno{4}}{\Nxi:\Nxi}
			]{data/W7X-EIM/MONKES/DKE_zeta_Convergence_Example_Nxi_20/Gamma_11_Gamma_31_nu_0.100E-04_E_rho_0.300E-03.plt};
			
				
			\addplot[forget plot, blue, mark=o, only marks] table[skip first n=1, 
			x expr=\thisrowno{4},
			y expr=-\thisrowno{7}*0.5237,
			restrict expr to domain={\thisrowno{2}}{27:47},
			restrict expr to domain={\thisrowno{3}}{65:179},
			restrict expr to domain={\thisrowno{4}}{\Nxi:\Nxi}
			]{data/W7X-EIM/MONKES/DKE_zeta_Convergence_Example_Nxi_20/Gamma_11_Gamma_31_nu_0.100E-04_E_rho_0.300E-03.plt};
			
		}
		
		\foreach \Nxi in {120, 140, 180}{		
			\addplot[forget plot, blue, mark=+, only marks] table[skip first n=1, 
			x expr=\thisrowno{4},
			y expr=\thisrowno{6}*0.5237,
			restrict expr to domain={\thisrowno{2}}{47:67},
			restrict expr to domain={\thisrowno{3}}{71:179},
			restrict expr to domain={\thisrowno{4}}{\Nxi:\Nxi}
			]{data/W7X-EIM/MONKES/DKE_zeta_Convergence_Example_Nxi_120/Gamma_11_Gamma_31_nu_0.100E-04_E_rho_0.300E-03.plt};
			
			
			\addplot[forget plot, blue, mark=o, only marks] table[skip first n=1, 
			x expr=\thisrowno{4},
			y expr=\thisrowno{6}*0.5237,
			restrict expr to domain={\thisrowno{2}}{47:67},
			restrict expr to domain={\thisrowno{3}}{71:179},
			restrict expr to domain={\thisrowno{4}}{\Nxi:\Nxi}
			]{data/W7X-EIM/MONKES/DKE_zeta_Convergence_Example_Nxi_120/Gamma_11_Gamma_31_nu_0.100E-04_E_rho_0.300E-03.plt};
			
		}	
	    
	    \foreach \Nxi in {200}
	    {		
	    	\addplot[forget plot, blue, mark=+, only marks] table[skip first n=1, 
	    	x expr=\thisrowno{4},
	    	y expr=\thisrowno{6}*0.5237,
	    	restrict expr to domain={\thisrowno{2}}{47:67},
	    	restrict expr to domain={\thisrowno{3}}{71:179},
	    	restrict expr to domain={\thisrowno{4}}{\Nxi:\Nxi}
	    	]{data/W7X-EIM/MONKES/DKE_zeta_Convergence_Example_Nxi_120/Gamma_11_Gamma_31_nu_0.100E-04_E_rho_0.300E-03.plt};
	    	
	    	
	    	\addplot[forget plot, blue, mark=o, only marks] table[skip first n=1, 
	    	x expr=\thisrowno{4},
	    	y expr=-\thisrowno{7}*0.5237,
	    	restrict expr to domain={\thisrowno{2}}{47:67},
	    	restrict expr to domain={\thisrowno{3}}{71:179},
	    	restrict expr to domain={\thisrowno{4}}{\Nxi:\Nxi}
	    	]{data/W7X-EIM/MONKES/DKE_zeta_Convergence_Example_Nxi_120/Gamma_11_Gamma_31_nu_0.100E-04_E_rho_0.300E-03.plt};
	    	
	    }	
		
		\addplot[ultra thick, Green, mark = star, mark size = 5 pt, only marks] table[skip first n=1, 
		x expr=\thisrowno{4},
		y expr=\thisrowno{6}*0.5237,
		restrict expr to domain={\thisrowno{2}}{45:67},
		restrict expr to domain={\thisrowno{3}}{91:179},
		restrict expr to domain={\thisrowno{4}}{160:160}
		]{data/W7X-EIM/MONKES/Convergence_nu_1e-5_Er_3e-4/Nxi_160/monkes_Monoenergetic_Database.dat};
		\addlegendentry{Selected}	
		
		
	\end{axis}
\end{tikzpicture}
%
    	\caption{}
    	\label{subfig:D31_convergence_Legendre_W7X_EIM_0200_Erho_3e-4_Detail}
    \end{subfigure}
    %\hfill
    \begin{subfigure}[t]{0.32\textwidth}
    	\tikzsetnextfilename{Convergence-theta-zeta-W7X-EIM-s0200-Er-3e-4-D31}
    	\begin{tikzpicture}
	\begin{axis}[
		%		height=0.85\textwidth,
		ymin =0.235/2.4311, 
		width=\textwidth, 
		scaled y ticks=base 10:1,
		%		xtick={80,120,160,200, 240},
		y tick label style={
			/pgf/number format/.cd,
			fixed,
			fixed zerofill,
			precision=1,
			/tikz/.cd}, 
		%		ymin = 0.053,
				ymax = 0.1338,
		legend pos = south east, 
		legend columns =2, 
		xlabel = $N_\zeta$, ylabel=$\widehat{D}_{31}$ ${[\text{m}]}$
		]
		
		\foreach \Ntheta in {27}{		
			\addplot+[forget plot, ultra thick, Green, mark = star, mark size = 5 pt, only marks] table[skip first n=1, 
			x expr=\thisrowno{3},
			y expr=\thisrowno{6}*0.5237,
			restrict expr to domain={\thisrowno{2}}{\Ntheta:\Ntheta},
			restrict expr to domain={\thisrowno{3}}{55:55},
			restrict expr to domain={\thisrowno{4}}{160:160}
			]{results/W7X-EIM/0.200/Convergence_nu_1e-5_Er_3e-4/Nxi_160/monkes_Monoenergetic_Database.dat};
		}
		
		\addplot[forget plot, name path=Upper2,blue!20, 
		domain = 23:131] {0.241E+00*1.05*0.5237};	
		\addplot[forget plot, name path=Lower2,blue!20, 
		domain = 23:131] {0.241E+00*0.95*0.5237};		
		\addplot[forget plot, blue!20] fill between[of=Upper2 and Lower2];
		
		
		\addplot[forget plot, name path=Upper2,red!20, 
		domain = 23:131] {(0.241E+00)*0.5237+0.005};	
		\addplot[forget plot, name path=Lower2,red!20, 
		domain = 23:131] {(0.241E+00)*0.5237-0.005};		
		\addplot[forget plot, red!20] fill between[of=Upper2 and Lower2];
		
		\foreach \Ntheta in {19,23,27,...,31}{		
			\addplot+[no markers] table[skip first n=1, 
			x expr=\thisrowno{3},
			y expr=\thisrowno{6}*0.5237,
			restrict expr to domain={\thisrowno{2}}{\Ntheta:\Ntheta},
			restrict expr to domain={\thisrowno{3}}{27:131},
			restrict expr to domain={\thisrowno{4}}{160:160}
			]{results/W7X-EIM/0.200/Convergence_nu_1e-5_Er_3e-4/Nxi_160/monkes_Monoenergetic_Database.dat};
			
			\addlegendimage{empty legend}
			\addlegendentry{$N_\theta=$}
			\expandafter\addlegendentry\expandafter{\Ntheta}
		}	
				
		
		
		
	\end{axis}
\end{tikzpicture}
%
    	\caption{}
    	\label{subfig:D31_convergence_theta_zeta_W7X_EIM_0200_Erho_3e-4_Detail}
    \end{subfigure}


	\caption{Convergence of monoenergetic coefficients with the number of Legendre modes $N_\xi$ for W7X-EIM at the surface labelled by $\psi/\psi_{\text{lcfs}}=0.200$, for $\hat{\nu}(v)=10^{-5}$ $\text{m}^{-1}$ and $\hat{E}_r=3\cdot 10^{-4}$ $\text{kV}\cdot\text{s}/\text{m}^2$.}
	\label{fig:Convergence_W7X_EIM_Er_3e-4}
\end{figure*}

On figure \ref{subfig:D31_convergence_Legendre_W7X_EIM_0200_Erho_3e-4_Detail} we check that to obtain sufficiently converged results for the region $\mathcal{R}_{5}$ is more difficult than in the previous $1/\nu$ regime. This is in part due to the fact that the $\widehat{D}_{31}$ coefficient is smaller in absolute value and thus, the region $\mathcal{R}_{5}$ is narrower. Inspecting this plot, with the same spirit as before of not trying to be too indulgent, we select $N_\xi=160$ to satisfy condition (i). The selection $(N_\theta,N_\zeta)=(27,55)$ satisfies condition (ii) as shown in figure \ref{subfig:D31_convergence_theta_zeta_W7X_EIM_0200_Erho_3e-4_Detail}. From figure \ref{subfig:D31_Clock_time_W7X_EIM_0200_Erho_3e-4_Detail} we check that it required 40 seconds to compute the monoenergetic coefficients with this selection of $(N_\theta,N_\zeta,N_\xi)$, which is quite fast.

\begin{figure}[]
	\centering
	\begin{subfigure}[t]{0.33\textwidth}
		\tikzsetnextfilename{Convergence-Legendre-W7X-EIM-s0200-Er-3e-4-D31-Detail}
		\begin{tikzpicture}
	\begin{axis}[
		%		height=0.85\textwidth, 
		width=\textwidth, 
		scaled y ticks=base 10:1,
		y tick label style={
			/pgf/number format/.cd,
			fixed,
			fixed zerofill,
			precision=1,
			/tikz/.cd}, 
		xlabel = $N_\xi$, ylabel=$\widehat{D}_{31}$ ${[\text{m}]}$,
		legend pos = north west,
%		ymin = 0.16,
		ymax = 0.4*0.5237,
		legend columns = 2
		]
		
		\addplot[blue, mark=+, only marks] table[skip first n=1, 
		x expr=\thisrowno{4},
		y expr=\thisrowno{6}*0.5237,
		restrict expr to domain={\thisrowno{1}}{3e-4:3e-4},
		restrict expr to domain={\thisrowno{2}}{47:67},
		restrict expr to domain={\thisrowno{3}}{71:179},
		restrict expr to domain={\thisrowno{4}}{220:300}
		]{data/W7X-EIM/MONKES/Convergence_Nxi/N_theta_47_N_zeta_153/monkes_Monoenergetic_Database.dat};
		\addlegendentry{$\widehat{D}_{31}$}
		
		\addplot[blue, mark=o, only marks] table[skip first n=1, 
		x expr=\thisrowno{4},
		y expr=-\thisrowno{7}*0.5237,
		restrict expr to domain={\thisrowno{1}}{3e-4:3e-4},
		restrict expr to domain={\thisrowno{2}}{47:67},
		restrict expr to domain={\thisrowno{3}}{71:179},
		restrict expr to domain={\thisrowno{4}}{220:300}
		]{data/W7X-EIM/MONKES/Convergence_Nxi/N_theta_47_N_zeta_153/monkes_Monoenergetic_Database.dat};
		\addlegendentry{$-\widehat{D}_{13}$}
		
		\addplot[forget plot, name path=Upper2,blue!20, 
		domain = 40:300] {0.241E+00*1.05*0.5237};	
		\addplot[forget plot, name path=Lower2,blue!20, 
		domain = 40:300] {0.241E+00*0.95*0.5237};		
		\addplot[blue!20] fill between[of=Upper2 and Lower2];
		\addlegendentry{ $\mathcal{R}_5$ }
		
		\addplot[forget plot, name path=Upper2,red!20, 
		domain = 40:300] {(0.241E+00*0.5237+0.005)};
		\addplot[forget plot, name path=Lower2,red!20, 
		domain = 40:300] {(0.241E+00*0.5237-0.005)};		
		\addplot[red!20] fill between[of=Upper2 and Lower2];
		\addlegendentry{ $\mathcal{A}_{0.005}$ }
		
		
%		\addplot[forget plot, name path=Upper2,red!20, 
%		domain = 40:200] {0.2365291332172711E+00+0.01};	
%		\addplot[forget plot, name path=Lower2,red!20, 
%		domain = 40:200] {0.2365291332172711E+00-0.01};		
%		\addplot[red!20] fill between[of=Upper2 and Lower2];	
%		\addlegendentry{ $\pm 0.01$ }	
		
		\foreach \Nxi in {40,60,80,100}{		
			\addplot[forget plot, blue, mark=+, only marks] table[skip first n=1, 
			x expr=\thisrowno{4},
			y expr=\thisrowno{6}*0.5237,
			restrict expr to domain={\thisrowno{2}}{27:47},
			restrict expr to domain={\thisrowno{3}}{65:179},
			restrict expr to domain={\thisrowno{4}}{\Nxi:\Nxi}
			]{data/W7X-EIM/MONKES/DKE_zeta_Convergence_Example_Nxi_20/Gamma_11_Gamma_31_nu_0.100E-04_E_rho_0.300E-03.plt};
			
				
			\addplot[forget plot, blue, mark=o, only marks] table[skip first n=1, 
			x expr=\thisrowno{4},
			y expr=-\thisrowno{7}*0.5237,
			restrict expr to domain={\thisrowno{2}}{27:47},
			restrict expr to domain={\thisrowno{3}}{65:179},
			restrict expr to domain={\thisrowno{4}}{\Nxi:\Nxi}
			]{data/W7X-EIM/MONKES/DKE_zeta_Convergence_Example_Nxi_20/Gamma_11_Gamma_31_nu_0.100E-04_E_rho_0.300E-03.plt};
			
		}
		
		\foreach \Nxi in {120, 140, 180}{		
			\addplot[forget plot, blue, mark=+, only marks] table[skip first n=1, 
			x expr=\thisrowno{4},
			y expr=\thisrowno{6}*0.5237,
			restrict expr to domain={\thisrowno{2}}{47:67},
			restrict expr to domain={\thisrowno{3}}{71:179},
			restrict expr to domain={\thisrowno{4}}{\Nxi:\Nxi}
			]{data/W7X-EIM/MONKES/DKE_zeta_Convergence_Example_Nxi_120/Gamma_11_Gamma_31_nu_0.100E-04_E_rho_0.300E-03.plt};
			
			
			\addplot[forget plot, blue, mark=o, only marks] table[skip first n=1, 
			x expr=\thisrowno{4},
			y expr=\thisrowno{6}*0.5237,
			restrict expr to domain={\thisrowno{2}}{47:67},
			restrict expr to domain={\thisrowno{3}}{71:179},
			restrict expr to domain={\thisrowno{4}}{\Nxi:\Nxi}
			]{data/W7X-EIM/MONKES/DKE_zeta_Convergence_Example_Nxi_120/Gamma_11_Gamma_31_nu_0.100E-04_E_rho_0.300E-03.plt};
			
		}	
	    
	    \foreach \Nxi in {200}
	    {		
	    	\addplot[forget plot, blue, mark=+, only marks] table[skip first n=1, 
	    	x expr=\thisrowno{4},
	    	y expr=\thisrowno{6}*0.5237,
	    	restrict expr to domain={\thisrowno{2}}{47:67},
	    	restrict expr to domain={\thisrowno{3}}{71:179},
	    	restrict expr to domain={\thisrowno{4}}{\Nxi:\Nxi}
	    	]{data/W7X-EIM/MONKES/DKE_zeta_Convergence_Example_Nxi_120/Gamma_11_Gamma_31_nu_0.100E-04_E_rho_0.300E-03.plt};
	    	
	    	
	    	\addplot[forget plot, blue, mark=o, only marks] table[skip first n=1, 
	    	x expr=\thisrowno{4},
	    	y expr=-\thisrowno{7}*0.5237,
	    	restrict expr to domain={\thisrowno{2}}{47:67},
	    	restrict expr to domain={\thisrowno{3}}{71:179},
	    	restrict expr to domain={\thisrowno{4}}{\Nxi:\Nxi}
	    	]{data/W7X-EIM/MONKES/DKE_zeta_Convergence_Example_Nxi_120/Gamma_11_Gamma_31_nu_0.100E-04_E_rho_0.300E-03.plt};
	    	
	    }	
		
		\addplot[ultra thick, Green, mark = star, mark size = 5 pt, only marks] table[skip first n=1, 
		x expr=\thisrowno{4},
		y expr=\thisrowno{6}*0.5237,
		restrict expr to domain={\thisrowno{2}}{45:67},
		restrict expr to domain={\thisrowno{3}}{91:179},
		restrict expr to domain={\thisrowno{4}}{160:160}
		]{data/W7X-EIM/MONKES/Convergence_nu_1e-5_Er_3e-4/Nxi_160/monkes_Monoenergetic_Database.dat};
		\addlegendentry{Selected}	
		
		
	\end{axis}
\end{tikzpicture}

		\caption{}
		\label{subfig:D31_convergence_Legendre_W7X_EIM_0200_Erho_3e-4_Detail}
	\end{subfigure}
	%\hfill
	\begin{subfigure}[t]{0.33\textwidth}
		\tikzsetnextfilename{Convergence-theta-zeta-W7X-EIM-s0200-Er-3e-4-D31}
		\begin{tikzpicture}
	\begin{axis}[
		%		height=0.85\textwidth,
		ymin =0.235/2.4311, 
		width=\textwidth, 
		scaled y ticks=base 10:1,
		%		xtick={80,120,160,200, 240},
		y tick label style={
			/pgf/number format/.cd,
			fixed,
			fixed zerofill,
			precision=1,
			/tikz/.cd}, 
		%		ymin = 0.053,
				ymax = 0.1338,
		legend pos = south east, 
		legend columns =2, 
		xlabel = $N_\zeta$, ylabel=$\widehat{D}_{31}$ ${[\text{m}]}$
		]
		
		\foreach \Ntheta in {27}{		
			\addplot+[forget plot, ultra thick, Green, mark = star, mark size = 5 pt, only marks] table[skip first n=1, 
			x expr=\thisrowno{3},
			y expr=\thisrowno{6}*0.5237,
			restrict expr to domain={\thisrowno{2}}{\Ntheta:\Ntheta},
			restrict expr to domain={\thisrowno{3}}{55:55},
			restrict expr to domain={\thisrowno{4}}{160:160}
			]{results/W7X-EIM/0.200/Convergence_nu_1e-5_Er_3e-4/Nxi_160/monkes_Monoenergetic_Database.dat};
		}
		
		\addplot[forget plot, name path=Upper2,blue!20, 
		domain = 23:131] {0.241E+00*1.05*0.5237};	
		\addplot[forget plot, name path=Lower2,blue!20, 
		domain = 23:131] {0.241E+00*0.95*0.5237};		
		\addplot[forget plot, blue!20] fill between[of=Upper2 and Lower2];
		
		
		\addplot[forget plot, name path=Upper2,red!20, 
		domain = 23:131] {(0.241E+00)*0.5237+0.005};	
		\addplot[forget plot, name path=Lower2,red!20, 
		domain = 23:131] {(0.241E+00)*0.5237-0.005};		
		\addplot[forget plot, red!20] fill between[of=Upper2 and Lower2];
		
		\foreach \Ntheta in {19,23,27,...,31}{		
			\addplot+[no markers] table[skip first n=1, 
			x expr=\thisrowno{3},
			y expr=\thisrowno{6}*0.5237,
			restrict expr to domain={\thisrowno{2}}{\Ntheta:\Ntheta},
			restrict expr to domain={\thisrowno{3}}{27:131},
			restrict expr to domain={\thisrowno{4}}{160:160}
			]{results/W7X-EIM/0.200/Convergence_nu_1e-5_Er_3e-4/Nxi_160/monkes_Monoenergetic_Database.dat};
			
			\addlegendimage{empty legend}
			\addlegendentry{$N_\theta=$}
			\expandafter\addlegendentry\expandafter{\Ntheta}
		}	
				
		
		
		
	\end{axis}
\end{tikzpicture}

		\caption{}
		\label{subfig:D31_convergence_theta_zeta_W7X_EIM_0200_Erho_3e-4_Detail}
	\end{subfigure}
	%	%\hfill
	\begin{subfigure}[t]{0.33\textwidth}
		\tikzsetnextfilename{Clock-time-W7X-EIM-s0200-Er-3e-4-D31}
		\begin{tikzpicture}
	\begin{axis}[
		%		height=0.85\textwidth, 
		ymin = 1,
		width=0.97\textwidth, 
		%		scaled y ticks=base 10:-1,
		%		xtick={80,120,160,200, 240},
		xtick = {20,40,80},
		ytick = {10,70,100, 130},
		extra x ticks={0.55E+02},
		extra y ticks={0.40E+02},
		extra tick style={grid=major, grid style={dashed,black}}, 
		y tick label style={
			/pgf/number format/.cd,
			fixed,
			fixed zerofill,
			precision=0,
			/tikz/.cd}, 
		%		ymin = 0.053,
		%		ymax = 0.102,
		legend pos = south east, 
		legend columns =2, 
		xlabel = $N_\zeta$, ylabel=Wall-clock time {[s]}
		]
		
		
		%		\addplot[forget plot, name path=Upper2,blue!20, 
		%		domain = 15:71] {0.3593392335766615E+00*1.05};	
		%		\addplot[forget plot, name path=Lower2,blue!20, 
		%		domain = 15:71] {0.3593392335766615E+00*0.95};		
		%		\addplot[forget plot, blue!20] fill between[of=Upper2 and Lower2];
		
		\foreach \Ntheta in {27}{		
			\addplot+[Brown,no markers] table[skip first n=1, 
			x expr=\thisrowno{3},
			y expr=\thisrowno{10},
			restrict expr to domain={\thisrowno{2}}{\Ntheta:\Ntheta},
			restrict expr to domain={\thisrowno{3}}{21:81},
			restrict expr to domain={\thisrowno{4}}{160:160}
			]{results/W7X-EIM/0.200/Convergence_nu_1e-5_Er_3e-4/Nxi_160/monkes_Monoenergetic_Database.dat};
			\addlegendimage{empty legend}
			\addlegendentry{$N_\theta=$}
			\expandafter\addlegendentry\expandafter{\Ntheta}
		}	
		
		
		\foreach \Ntheta in {27}{		
			\addplot+[forget plot, ultra thick, Green, mark = star, mark size = 5 pt, only marks] table[skip first n=1, 
			x expr=\thisrowno{3},
			y expr=\thisrowno{10},
			restrict expr to domain={\thisrowno{2}}{\Ntheta:\Ntheta},
			restrict expr to domain={\thisrowno{3}}{55:55},
			restrict expr to domain={\thisrowno{4}}{160:160}
			]{results/W7X-EIM/0.200/Convergence_nu_1e-5_Er_3e-4/Nxi_160/monkes_Monoenergetic_Database.dat};
		}
		
		
	\end{axis}
\end{tikzpicture}

		\caption{}
		\label{subfig:D31_Clock_time_W7X_EIM_0200_Erho_3e-4_Detail}
	\end{subfigure}
	\caption{Selection of the resolution to have a sufficiently accurate calculation of the parallel flow geometric coefficient $\widehat{D}_{31}$ for W7X-EIM at the surface labelled by $\psi/\psi_{\text{lcfs}}=0.200$, for $\hat{\nu}(v)=10^{-5}$ $\text{m}^{-1}$ and $\hat{E}_r(v)=3\cdot 10^{-4}$  $\text{kV}\cdot\text{s}/\text{m}^2$.}
	\label{fig:Convergence_W7X_EIM_Er_3e-4_Detail}
\end{figure}

\FloatBarrier
For the high mirror configuration of W7-X in the $1/\nu$ we can use again the calculation for $N_\xi=200$ as the converged reference value $\widehat{D}_{31}^{\text{r}}$ as shown in figure \ref{subfig:D31_convergence_Legendre_W7X_KJM_0204_Erho_0}. Due to the smaller value of $\widehat{D}_{31}^{\text{r}}$ the amplitude of the region $\mathcal{R}_{5}$ is much narrower, being of order $10^{-3}$. On figure \ref{subfig:D31_convergence_Legendre_W7X_KJM_0204_Erho_0_Detail} we see that taking $N_\xi=140$ is sufficient to satisfy convergence condition (i). According to the convergence curves shown in figure \ref{subfig:D31_convergence_theta_zeta_W7X_KJM_0204_Erho_0}, taking $(N_\theta,N_\zeta)=(23,63)$ would yield a calculation that satisfies condition (ii). This calculation is also quite fast as can be checked from figure \ref{subfig:D31_Clock_time_W7X_KJM_0204_Erho_0}, with a wall-clock time of 33 seconds. 
\begin{figure*}[t]
	\centering
	\begin{subfigure}[t]{0.32\textwidth}
		\tikzsetnextfilename{Convergence-Legendre-W7X-KJM-s0204-Er-0-D11}
		
\begin{tikzpicture}
	\begin{axis}[
		%		height=0.85\textwidth, 
		width=0.95\textwidth, 
		y tick label style={
			/pgf/number format/.cd,
			fixed,
			fixed zerofill,
			precision=2,
			/tikz/.cd}, 
		xlabel = $N_\xi$, ylabel=$\Gamma_{11}$
		]
		\addplot[blue, mark=*, only marks] table[skip first n=1, 
		x expr=\thisrowno{4},
		y expr=\thisrowno{5},
		restrict expr to domain={\thisrowno{2}}{25:47},
		restrict expr to domain={\thisrowno{3}}{55:97}
		]{results/W7X-KJM/0.204/DKE_zeta_Convergence_Example_Nxi_20/Gamma_11_Gamma_31_nu_0.100E-04_E_rho_0.000E+00.plt};
		%
		\addplot[blue, mark=*, only marks] table[skip first n=1, 
		x expr=\thisrowno{4},
		y expr=\thisrowno{5},
		restrict expr to domain={\thisrowno{2}}{25:47},
		restrict expr to domain={\thisrowno{3}}{55:97}
		]{results/W7X-KJM/0.204/DKE_zeta_Convergence_Example_Nxi_120/Gamma_11_Gamma_31_nu_0.100E-04_E_rho_0.000E+00.plt};			
	\end{axis}
\end{tikzpicture}
%
		\caption{}
		\label{subfig:D11_convergence_Legendre_W7X_KJM_0204_Erho_0}
	\end{subfigure}
	%\hfill
	%	%\hfill
	\begin{subfigure}[t]{0.32\textwidth}
		\tikzsetnextfilename{Convergence-Legendre-W7X-KJM-s0204-Er-0-D33}
		\begin{tikzpicture}
	\begin{axis}[
		%		height=0.85\textwidth, 
		width=\textwidth, 
		scaled y ticks=base 10:-4,
		y tick label style={
			/pgf/number format/.cd,
			fixed,
			fixed zerofill,
			precision=1,
			/tikz/.cd}, 
		xlabel = $N_\xi$, ylabel=$\widehat{D}_{33}$ ${[\text{m}]}$
		]
		
		
		
		\foreach \Nxi in {20, 40, ...,100}{		
			\addplot[blue, mark=+, only marks] table[skip first n=1, 
			x expr=\thisrowno{4},
			y expr=\thisrowno{8},
			restrict expr to domain={\thisrowno{2}}{19:47},
			restrict expr to domain={\thisrowno{3}}{65:179},
			restrict expr to domain={\thisrowno{4}}{\Nxi:\Nxi}
			]{results/W7X-KJM/0.204/DKE_zeta_Convergence_Example_Nxi_20/Gamma_11_Gamma_31_nu_0.100E-04_E_rho_0.000E+00.plt};
		}
		
		\foreach \Nxi in {120, 140, ...,200}{		
			\addplot[blue, mark=+, only marks] table[skip first n=1, 
			x expr=\thisrowno{4},
			y expr=\thisrowno{8},
			restrict expr to domain={\thisrowno{2}}{27:47},
			restrict expr to domain={\thisrowno{3}}{65:179},
			restrict expr to domain={\thisrowno{4}}{\Nxi:\Nxi}
			]{results/W7X-KJM/0.204/DKE_zeta_Convergence_Example_Nxi_120/Gamma_11_Gamma_31_nu_0.100E-04_E_rho_0.000E+00.plt};
		}	
		
			
		\addplot[forget plot,blue, mark=+, only marks] table[skip first n=1, 
		x expr=\thisrowno{4},
		y expr=-\thisrowno{8},
		restrict expr to domain={\thisrowno{1}}{0:0},
		restrict expr to domain={\thisrowno{2}}{47:67},
		restrict expr to domain={\thisrowno{3}}{71:179},
		restrict expr to domain={\thisrowno{4}}{220:380}
		]{results/W7X-KJM/0.204/Convergence_Nxi/N_theta_47_N_zeta_153/monkes_Monoenergetic_Database.dat};
		
	\end{axis}
\end{tikzpicture}%
		\caption{}
		\label{subfig:D33_convergence_Legendre_W7X_KJM_0204_Erho_0}
	\end{subfigure}

    
    \begin{subfigure}[t]{0.32\textwidth}
    	\tikzsetnextfilename{Convergence-Legendre-W7X-KJM-s0204-Er-0-D31-Detail}
    	\begin{tikzpicture}
	\begin{axis}[
		%		height=0.85\textwidth, 
		width=\textwidth, 
		%		xtick = {60, 100, 140, 180},
		scaled y ticks=base 10:2,
		y tick label style={
			/pgf/number format/.cd,
			fixed,
			fixed zerofill,
			precision=1,
			/tikz/.cd}, 
		xlabel = $N_\xi$, ylabel=$\widehat{D}_{31}$ ${[\text{m}]}$,
		legend pos = north east, 
		legend columns = 2, 
%		ymin = -0.06 * 2.5003 *0.5132
        ymax=0.2/2.5003
		]
				
		\foreach \Nxi in {60,80,100}
		{		
			\addplot[forget plot, blue, mark=+, only marks] table[skip first n=1, 
			x expr=\thisrowno{4},
			y expr=\thisrowno{6} *0.5132,
			restrict expr to domain={\thisrowno{2}}{27:47},
			restrict expr to domain={\thisrowno{3}}{65:179},
			restrict expr to domain={\thisrowno{4}}{\Nxi:\Nxi}
			]{results/W7X-KJM/0.204/DKE_zeta_Convergence_Example_Nxi_20/Gamma_11_Gamma_31_nu_0.100E-04_E_rho_0.000E+00.plt};
				
			\addplot[forget plot, blue, mark=o, only marks] table[skip first n=1, 
			x expr=\thisrowno{4},
			y expr=-\thisrowno{7} *0.5132,
			restrict expr to domain={\thisrowno{2}}{27:47},
			restrict expr to domain={\thisrowno{3}}{65:179},
			restrict expr to domain={\thisrowno{4}}{\Nxi:\Nxi}
			]{results/W7X-KJM/0.204/DKE_zeta_Convergence_Example_Nxi_20/Gamma_11_Gamma_31_nu_0.100E-04_E_rho_0.000E+00.plt};
		}
		
		\foreach \Nxi in {120, 160}{
					
			\addplot[forget plot, blue, mark=+, only marks] table[skip first n=1, 
			x expr=\thisrowno{4},
			y expr=\thisrowno{6} *0.5132,
			restrict expr to domain={\thisrowno{2}}{47:67},
			restrict expr to domain={\thisrowno{3}}{71:179},
			restrict expr to domain={\thisrowno{4}}{\Nxi:\Nxi}
			]{results/W7X-KJM/0.204/DKE_zeta_Convergence_Example_Nxi_120/Gamma_11_Gamma_31_nu_0.100E-04_E_rho_0.000E+00.plt};
			
			\addplot[forget plot, blue, mark=o, only marks] table[skip first n=1, 
			x expr=\thisrowno{4},
			y expr=-\thisrowno{7} *0.5132,
			restrict expr to domain={\thisrowno{2}}{47:67},
			restrict expr to domain={\thisrowno{3}}{71:179},
			restrict expr to domain={\thisrowno{4}}{\Nxi:\Nxi}
			]{results/W7X-KJM/0.204/DKE_zeta_Convergence_Example_Nxi_120/Gamma_11_Gamma_31_nu_0.100E-04_E_rho_0.000E+00.plt};
			
			
		}	
		
		\foreach \Nxi in {180}{		
			\addplot[forget plot, blue, mark=+, only marks] table[skip first n=1, 
			x expr=\thisrowno{4},
			y expr=\thisrowno{6} *0.5132,
			restrict expr to domain={\thisrowno{2}}{31:67},
			restrict expr to domain={\thisrowno{3}}{131:179},
			restrict expr to domain={\thisrowno{4}}{\Nxi:\Nxi}
			]{results/W7X-KJM/0.204/DKE_zeta_Convergence_Example_Nxi_\Nxi/Monoenergetic_nu_0.100E-04_E_rho_0.000E+00.plt};
					
			\addplot[forget plot, blue, mark=o, only marks] table[skip first n=1, 
			x expr=\thisrowno{4},
			y expr=-\thisrowno{7} *0.5132,
			restrict expr to domain={\thisrowno{2}}{31:67},
			restrict expr to domain={\thisrowno{3}}{131:179},
			restrict expr to domain={\thisrowno{4}}{\Nxi:\Nxi}
			]{results/W7X-KJM/0.204/DKE_zeta_Convergence_Example_Nxi_\Nxi/Monoenergetic_nu_0.100E-04_E_rho_0.000E+00.plt};
			
		}
		
		\foreach \Nxi in {200}{		
			\addplot[blue, mark=+, only marks] table[skip first n=1, 
			x expr=\thisrowno{4},
			y expr=\thisrowno{6} *0.5132,
			restrict expr to domain={\thisrowno{2}}{31:67},
			restrict expr to domain={\thisrowno{3}}{131:179},
			restrict expr to domain={\thisrowno{4}}{\Nxi:\Nxi}
			]{results/W7X-KJM/0.204/DKE_zeta_Convergence_Example_Nxi_\Nxi/Monoenergetic_nu_0.100E-04_E_rho_0.000E+00.plt};
			\addlegendentry{ $\widehat{D}_{31}$ }
			
			\addplot[blue, mark=o, only marks] table[skip first n=1, 
			x expr=\thisrowno{4},
			y expr=-\thisrowno{7} *0.5132,
			restrict expr to domain={\thisrowno{2}}{31:67},
			restrict expr to domain={\thisrowno{3}}{131:179},
			restrict expr to domain={\thisrowno{4}}{\Nxi:\Nxi}
			]{results/W7X-KJM/0.204/DKE_zeta_Convergence_Example_Nxi_\Nxi/Monoenergetic_nu_0.100E-04_E_rho_0.000E+00.plt};
			\addlegendentry{ $-\widehat{D}_{13}$ }
			
		}
			
		\addplot[forget plot, name path=Upper2,red!20, 
		domain = 60:380] {(0.672E-01) *0.5132 + 5e-3};
		\addplot[forget plot, name path=Lower2,red!20, 
		domain = 60:380] {(0.672E-01) *0.5132 - 5e-3};		
		\addplot[red!20] fill between[of=Upper2 and Lower2];
		\addlegendentry{ $\mathcal{A}_{0.005}$ }
		
		\addplot[forget plot, name path=Upper2,blue!20, 
		domain = 60:380] {0.672E-01*1.05 *0.5132};	
		\addplot[forget plot, name path=Lower2,blue!20, 
		domain = 60:380] {0.672E-01*0.95 *0.5132};		
		\addplot[blue!20] fill between[of=Upper2 and Lower2];
		\addlegendentry{ $\mathcal{R}_5$ }	
		
		\addplot[ultra thick, Green, mark = star, mark size = 5 pt, only marks] table[skip first n=1, 
		x expr=\thisrowno{4},
		y expr=\thisrowno{6} *0.5132,
		restrict expr to domain={\thisrowno{2}}{47:67},
		restrict expr to domain={\thisrowno{3}}{131:179},
		restrict expr to domain={\thisrowno{4}}{140:140}
		]{results/W7X-KJM/0.204/DKE_zeta_Convergence_Example_Nxi_140/Monoenergetic_nu_0.100E-04_E_rho_0.000E+00.plt};
		\addlegendentry{Selected}	
		
		
		\addplot[forget plot,blue, mark=+, only marks] table[skip first n=1, 
		x expr=\thisrowno{4},
		y expr=\thisrowno{6}*0.5237,
		restrict expr to domain={\thisrowno{1}}{0:0},
		restrict expr to domain={\thisrowno{2}}{47:67},
		restrict expr to domain={\thisrowno{3}}{71:179},
		restrict expr to domain={\thisrowno{4}}{220:380}
		]{results/W7X-KJM/0.204/Convergence_Nxi/N_theta_47_N_zeta_153/monkes_Monoenergetic_Database.dat};
		
		\addplot[forget plot,blue, mark=o, only marks] table[skip first n=1, 
		x expr=\thisrowno{4},
		y expr=-\thisrowno{7}*0.5237,
		restrict expr to domain={\thisrowno{1}}{0:0},
		restrict expr to domain={\thisrowno{2}}{47:67},
		restrict expr to domain={\thisrowno{3}}{71:179},
		restrict expr to domain={\thisrowno{4}}{220:380}
		]{results/W7X-KJM/0.204/Convergence_Nxi/N_theta_47_N_zeta_153/monkes_Monoenergetic_Database.dat};
	\end{axis}
\end{tikzpicture}
%
    	\caption{}
    	\label{subfig:D31_convergence_Legendre_W7X_KJM_0204_Erho_0_Detail}
    \end{subfigure}
    %\hfill
    \begin{subfigure}[t]{0.32\textwidth}
    	\tikzsetnextfilename{Convergence-theta-zeta-W7X-KJM-s0204-Er-0-D31}
    	\begin{tikzpicture}
	\begin{axis}[
		%		height=0.85\textwidth, 
		width=\textwidth, 
		scaled y ticks=base 10:2,
		%		xtick={80,120,160,200, 240},
		y tick label style={
			/pgf/number format/.cd,
			fixed,
			fixed zerofill,
			precision=1,
			/tikz/.cd}, 
		%		ymin = 0.053,
		ymax = 0.13 *0.5132,
		legend pos = north east, 
		legend columns =2, 
		xlabel = $N_\zeta$, ylabel=$\widehat{D}_{31}$ ${[\text{m}]}$
		]		
		
		
		\addplot[forget plot, name path=Upper2,red!20, 
		domain = 33:131] {(0.672E-01) *0.5132 + 5e-3};
		\addplot[forget plot, name path=Lower2,red!20, 
		domain = 33:131] {(0.672E-01) *0.5132 - 5e-3};		
		\addplot[forget plot,red!20] fill between[of=Upper2 and Lower2];
		
		
		\addplot[forget plot, name path=Upper2,blue!20, 
		domain = 33:131] {0.672E-01*1.05 *0.5132};	
		\addplot[forget plot, name path=Lower2,blue!20, 
		domain = 33:131] {0.672E-01*0.95 *0.5132};		
		\addplot[forget plot, blue!20] fill between[of=Upper2 and Lower2];
		
		\foreach \Ntheta in {15,19,...,27}{		
			\addplot+[no markers] table[skip first n=1, 
			x expr=\thisrowno{3},
			y expr=\thisrowno{6} *0.5132,
			restrict expr to domain={\thisrowno{2}}{\Ntheta:\Ntheta},
			restrict expr to domain={\thisrowno{3}}{37:131},
			restrict expr to domain={\thisrowno{4}}{140:140}
			]{results/W7X-KJM/0.204/DKE_zeta_Convergence_Example_Nxi_140/Monoenergetic_nu_0.100E-04_E_rho_0.000E+00.plt};
			\addlegendimage{empty legend}
			\addlegendentry{$N_\theta=$}
			\expandafter\addlegendentry\expandafter{\Ntheta}
		}			
		
		
		\foreach \Ntheta in {23}{		
			\addplot+[ultra thick, Green, mark = star, mark size = 5 pt, only marks] table[skip first n=1, 
			x expr=\thisrowno{3},
			y expr=\thisrowno{6} *0.5132,
			restrict expr to domain={\thisrowno{2}}{\Ntheta:\Ntheta},
			restrict expr to domain={\thisrowno{3}}{63:63},
			restrict expr to domain={\thisrowno{4}}{140:140}
			]{results/W7X-KJM/0.204/DKE_zeta_Convergence_Example_Nxi_140/Monoenergetic_nu_0.100E-04_E_rho_0.000E+00.plt};
		}
		
	\end{axis}
\end{tikzpicture}
%
    	\caption{}
    	\label{subfig:D31_convergence_theta_zeta_W7X_KJM_0204_Erho_0}
    \end{subfigure}
	\caption{Convergence of monoenergetic coefficients with the number of Legendre modes $N_\xi$ for W7X-KJM at the surface labelled by $\psi/\psi_{\text{lcfs}}=0.204$, for $\hat{\nu}(v)=10^{-5}$ $\text{m}^{-1}$ and $\hat{E}_r(v)=0$ $\text{kV}\cdot\text{s}/\text{m}^2$.}
	\label{fig:Convergence_W7X_KJM_Er_0}
\end{figure*}

\begin{figure}
	\centering
	\begin{subfigure}[t]{0.33\textwidth}
		\tikzsetnextfilename{Convergence-Legendre-W7X-KJM-s0204-Er-0-D31-Detail}
		\begin{tikzpicture}
	\begin{axis}[
		%		height=0.85\textwidth, 
		width=\textwidth, 
		%		xtick = {60, 100, 140, 180},
		scaled y ticks=base 10:2,
		y tick label style={
			/pgf/number format/.cd,
			fixed,
			fixed zerofill,
			precision=1,
			/tikz/.cd}, 
		xlabel = $N_\xi$, ylabel=$\widehat{D}_{31}$ ${[\text{m}]}$,
		legend pos = north east, 
		legend columns = 2, 
%		ymin = -0.06 * 2.5003 *0.5132
        ymax=0.2/2.5003
		]
				
		\foreach \Nxi in {60,80,100}
		{		
			\addplot[forget plot, blue, mark=+, only marks] table[skip first n=1, 
			x expr=\thisrowno{4},
			y expr=\thisrowno{6} *0.5132,
			restrict expr to domain={\thisrowno{2}}{27:47},
			restrict expr to domain={\thisrowno{3}}{65:179},
			restrict expr to domain={\thisrowno{4}}{\Nxi:\Nxi}
			]{results/W7X-KJM/0.204/DKE_zeta_Convergence_Example_Nxi_20/Gamma_11_Gamma_31_nu_0.100E-04_E_rho_0.000E+00.plt};
				
			\addplot[forget plot, blue, mark=o, only marks] table[skip first n=1, 
			x expr=\thisrowno{4},
			y expr=-\thisrowno{7} *0.5132,
			restrict expr to domain={\thisrowno{2}}{27:47},
			restrict expr to domain={\thisrowno{3}}{65:179},
			restrict expr to domain={\thisrowno{4}}{\Nxi:\Nxi}
			]{results/W7X-KJM/0.204/DKE_zeta_Convergence_Example_Nxi_20/Gamma_11_Gamma_31_nu_0.100E-04_E_rho_0.000E+00.plt};
		}
		
		\foreach \Nxi in {120, 160}{
					
			\addplot[forget plot, blue, mark=+, only marks] table[skip first n=1, 
			x expr=\thisrowno{4},
			y expr=\thisrowno{6} *0.5132,
			restrict expr to domain={\thisrowno{2}}{47:67},
			restrict expr to domain={\thisrowno{3}}{71:179},
			restrict expr to domain={\thisrowno{4}}{\Nxi:\Nxi}
			]{results/W7X-KJM/0.204/DKE_zeta_Convergence_Example_Nxi_120/Gamma_11_Gamma_31_nu_0.100E-04_E_rho_0.000E+00.plt};
			
			\addplot[forget plot, blue, mark=o, only marks] table[skip first n=1, 
			x expr=\thisrowno{4},
			y expr=-\thisrowno{7} *0.5132,
			restrict expr to domain={\thisrowno{2}}{47:67},
			restrict expr to domain={\thisrowno{3}}{71:179},
			restrict expr to domain={\thisrowno{4}}{\Nxi:\Nxi}
			]{results/W7X-KJM/0.204/DKE_zeta_Convergence_Example_Nxi_120/Gamma_11_Gamma_31_nu_0.100E-04_E_rho_0.000E+00.plt};
			
			
		}	
		
		\foreach \Nxi in {180}{		
			\addplot[forget plot, blue, mark=+, only marks] table[skip first n=1, 
			x expr=\thisrowno{4},
			y expr=\thisrowno{6} *0.5132,
			restrict expr to domain={\thisrowno{2}}{31:67},
			restrict expr to domain={\thisrowno{3}}{131:179},
			restrict expr to domain={\thisrowno{4}}{\Nxi:\Nxi}
			]{results/W7X-KJM/0.204/DKE_zeta_Convergence_Example_Nxi_\Nxi/Monoenergetic_nu_0.100E-04_E_rho_0.000E+00.plt};
					
			\addplot[forget plot, blue, mark=o, only marks] table[skip first n=1, 
			x expr=\thisrowno{4},
			y expr=-\thisrowno{7} *0.5132,
			restrict expr to domain={\thisrowno{2}}{31:67},
			restrict expr to domain={\thisrowno{3}}{131:179},
			restrict expr to domain={\thisrowno{4}}{\Nxi:\Nxi}
			]{results/W7X-KJM/0.204/DKE_zeta_Convergence_Example_Nxi_\Nxi/Monoenergetic_nu_0.100E-04_E_rho_0.000E+00.plt};
			
		}
		
		\foreach \Nxi in {200}{		
			\addplot[blue, mark=+, only marks] table[skip first n=1, 
			x expr=\thisrowno{4},
			y expr=\thisrowno{6} *0.5132,
			restrict expr to domain={\thisrowno{2}}{31:67},
			restrict expr to domain={\thisrowno{3}}{131:179},
			restrict expr to domain={\thisrowno{4}}{\Nxi:\Nxi}
			]{results/W7X-KJM/0.204/DKE_zeta_Convergence_Example_Nxi_\Nxi/Monoenergetic_nu_0.100E-04_E_rho_0.000E+00.plt};
			\addlegendentry{ $\widehat{D}_{31}$ }
			
			\addplot[blue, mark=o, only marks] table[skip first n=1, 
			x expr=\thisrowno{4},
			y expr=-\thisrowno{7} *0.5132,
			restrict expr to domain={\thisrowno{2}}{31:67},
			restrict expr to domain={\thisrowno{3}}{131:179},
			restrict expr to domain={\thisrowno{4}}{\Nxi:\Nxi}
			]{results/W7X-KJM/0.204/DKE_zeta_Convergence_Example_Nxi_\Nxi/Monoenergetic_nu_0.100E-04_E_rho_0.000E+00.plt};
			\addlegendentry{ $-\widehat{D}_{13}$ }
			
		}
			
		\addplot[forget plot, name path=Upper2,red!20, 
		domain = 60:380] {(0.672E-01) *0.5132 + 5e-3};
		\addplot[forget plot, name path=Lower2,red!20, 
		domain = 60:380] {(0.672E-01) *0.5132 - 5e-3};		
		\addplot[red!20] fill between[of=Upper2 and Lower2];
		\addlegendentry{ $\mathcal{A}_{0.005}$ }
		
		\addplot[forget plot, name path=Upper2,blue!20, 
		domain = 60:380] {0.672E-01*1.05 *0.5132};	
		\addplot[forget plot, name path=Lower2,blue!20, 
		domain = 60:380] {0.672E-01*0.95 *0.5132};		
		\addplot[blue!20] fill between[of=Upper2 and Lower2];
		\addlegendentry{ $\mathcal{R}_5$ }	
		
		\addplot[ultra thick, Green, mark = star, mark size = 5 pt, only marks] table[skip first n=1, 
		x expr=\thisrowno{4},
		y expr=\thisrowno{6} *0.5132,
		restrict expr to domain={\thisrowno{2}}{47:67},
		restrict expr to domain={\thisrowno{3}}{131:179},
		restrict expr to domain={\thisrowno{4}}{140:140}
		]{results/W7X-KJM/0.204/DKE_zeta_Convergence_Example_Nxi_140/Monoenergetic_nu_0.100E-04_E_rho_0.000E+00.plt};
		\addlegendentry{Selected}	
		
		
		\addplot[forget plot,blue, mark=+, only marks] table[skip first n=1, 
		x expr=\thisrowno{4},
		y expr=\thisrowno{6}*0.5237,
		restrict expr to domain={\thisrowno{1}}{0:0},
		restrict expr to domain={\thisrowno{2}}{47:67},
		restrict expr to domain={\thisrowno{3}}{71:179},
		restrict expr to domain={\thisrowno{4}}{220:380}
		]{results/W7X-KJM/0.204/Convergence_Nxi/N_theta_47_N_zeta_153/monkes_Monoenergetic_Database.dat};
		
		\addplot[forget plot,blue, mark=o, only marks] table[skip first n=1, 
		x expr=\thisrowno{4},
		y expr=-\thisrowno{7}*0.5237,
		restrict expr to domain={\thisrowno{1}}{0:0},
		restrict expr to domain={\thisrowno{2}}{47:67},
		restrict expr to domain={\thisrowno{3}}{71:179},
		restrict expr to domain={\thisrowno{4}}{220:380}
		]{results/W7X-KJM/0.204/Convergence_Nxi/N_theta_47_N_zeta_153/monkes_Monoenergetic_Database.dat};
	\end{axis}
\end{tikzpicture}

		\caption{}
		\label{subfig:D31_convergence_Legendre_W7X_KJM_0204_Erho_0_Detail}
	\end{subfigure}
	%\hfill
	\begin{subfigure}[t]{0.33\textwidth}
		\tikzsetnextfilename{Convergence-theta-zeta-W7X-KJM-s0204-Er-0-D31}
		\begin{tikzpicture}
	\begin{axis}[
		%		height=0.85\textwidth, 
		width=\textwidth, 
		scaled y ticks=base 10:2,
		%		xtick={80,120,160,200, 240},
		y tick label style={
			/pgf/number format/.cd,
			fixed,
			fixed zerofill,
			precision=1,
			/tikz/.cd}, 
		%		ymin = 0.053,
		ymax = 0.13 *0.5132,
		legend pos = north east, 
		legend columns =2, 
		xlabel = $N_\zeta$, ylabel=$\widehat{D}_{31}$ ${[\text{m}]}$
		]		
		
		
		\addplot[forget plot, name path=Upper2,red!20, 
		domain = 33:131] {(0.672E-01) *0.5132 + 5e-3};
		\addplot[forget plot, name path=Lower2,red!20, 
		domain = 33:131] {(0.672E-01) *0.5132 - 5e-3};		
		\addplot[forget plot,red!20] fill between[of=Upper2 and Lower2];
		
		
		\addplot[forget plot, name path=Upper2,blue!20, 
		domain = 33:131] {0.672E-01*1.05 *0.5132};	
		\addplot[forget plot, name path=Lower2,blue!20, 
		domain = 33:131] {0.672E-01*0.95 *0.5132};		
		\addplot[forget plot, blue!20] fill between[of=Upper2 and Lower2];
		
		\foreach \Ntheta in {15,19,...,27}{		
			\addplot+[no markers] table[skip first n=1, 
			x expr=\thisrowno{3},
			y expr=\thisrowno{6} *0.5132,
			restrict expr to domain={\thisrowno{2}}{\Ntheta:\Ntheta},
			restrict expr to domain={\thisrowno{3}}{37:131},
			restrict expr to domain={\thisrowno{4}}{140:140}
			]{results/W7X-KJM/0.204/DKE_zeta_Convergence_Example_Nxi_140/Monoenergetic_nu_0.100E-04_E_rho_0.000E+00.plt};
			\addlegendimage{empty legend}
			\addlegendentry{$N_\theta=$}
			\expandafter\addlegendentry\expandafter{\Ntheta}
		}			
		
		
		\foreach \Ntheta in {23}{		
			\addplot+[ultra thick, Green, mark = star, mark size = 5 pt, only marks] table[skip first n=1, 
			x expr=\thisrowno{3},
			y expr=\thisrowno{6} *0.5132,
			restrict expr to domain={\thisrowno{2}}{\Ntheta:\Ntheta},
			restrict expr to domain={\thisrowno{3}}{63:63},
			restrict expr to domain={\thisrowno{4}}{140:140}
			]{results/W7X-KJM/0.204/DKE_zeta_Convergence_Example_Nxi_140/Monoenergetic_nu_0.100E-04_E_rho_0.000E+00.plt};
		}
		
	\end{axis}
\end{tikzpicture}

		\caption{}
		\label{subfig:D31_convergence_theta_zeta_W7X_KJM_0204_Erho_0}
	\end{subfigure}
	%	%\hfill
	\begin{subfigure}[t]{0.33\textwidth}
		\tikzsetnextfilename{Clock-time-W7X-KJM-s0204-Er-0-D31}
		\begin{tikzpicture}
	\begin{axis}[
		%		height=0.85\textwidth, 
%				ymax = 150, 
		width=0.97\textwidth, 
		xtick = {20,40,80,100},
		extra x ticks={63},		
		extra y ticks={32.6},
		extra tick style={grid=major, grid style={dashed,black}}, 
		%		scaled y ticks=base 10:-1,
		%		xtick={80,120,160,200, 240},
		y tick label style={
			/pgf/number format/.cd,
			fixed,
			fixed zerofill,
			precision=0,
			/tikz/.cd}, 
		%		ymin = 0.053,
		%		ymax = 0.102,
		legend pos = south east, 
		legend columns =2, 
		xlabel = $N_\zeta$, ylabel=Wall-clock time {[s]}
		]
		
		
		%		\addplot[forget plot, name path=Upper2,blue!20, 
		%		domain = 15:71] {0.3593392335766615E+00*1.05};	
		%		\addplot[forget plot, name path=Lower2,blue!20, 
		%		domain = 15:71] {0.3593392335766615E+00*0.95};		
		%		\addplot[forget plot, blue!20] fill between[of=Upper2 and Lower2];
		
		\foreach \Ntheta in {23}{		
			\addplot+[Brown, no markers] table[skip first n=1, 
			x expr=\thisrowno{3},
			y expr=\thisrowno{10},
			restrict expr to domain={\thisrowno{2}}{\Ntheta:\Ntheta},
			restrict expr to domain={\thisrowno{3}}{21:105},
			restrict expr to domain={\thisrowno{4}}{140:140}
			]{results/W7X-KJM/0.204/DKE_zeta_Convergence_Example_Nxi_140/Monoenergetic_nu_0.100E-04_E_rho_0.000E+00.plt};
			\addlegendimage{empty legend}
			\addlegendentry{$N_\theta=$}
			\expandafter\addlegendentry\expandafter{\Ntheta}
		}	
		
		
		
		\foreach \Ntheta in {23}{		
			\addplot+[ultra thick, Green, mark = star, mark size = 5 pt, only marks] table[skip first n=1, 
			x expr=\thisrowno{3},
			y expr=\thisrowno{10},
			restrict expr to domain={\thisrowno{2}}{\Ntheta:\Ntheta},
			restrict expr to domain={\thisrowno{3}}{63:63},
			restrict expr to domain={\thisrowno{4}}{140:140}
			]{results/W7X-KJM/0.204/DKE_zeta_Convergence_Example_Nxi_140/Monoenergetic_nu_0.100E-04_E_rho_0.000E+00.plt};
		}
		
	\end{axis}
\end{tikzpicture}

		\caption{}
		\label{subfig:D31_Clock_time_W7X_KJM_0204_Erho_0}
	\end{subfigure}
	\caption{Selection of the resolution to have a sufficiently accurate calculation of the parallel flow geometric coefficient $\widehat{D}_{31}$ for W7X-KJM at the surface labelled by $\psi/\psi_{\text{lcfs}}=0.200$, for $\hat{\nu}(v)=10^{-5}$ $\text{m}^{-1}$ and $\hat{E}_r(v)=0$ $\text{kV}\cdot\text{s}/\text{m}^2$.}
	\label{fig:Convergence_W7X_KJM_Er_0_Detail}
\end{figure}
\FloatBarrier
Again, for the high mirror configuration of W7-X in the $\sqrt{\nu}$ the (spatially converged) calculation for $N_\xi=200$ serves as the converged reference value $\widehat{D}_{31}^{\text{r}}$ as can be checked in figure \ref{subfig:D31_convergence_Legendre_W7X_KJM_0204_Erho_0}. The number of Legendre modes selected to have a calculation which satisfies convergence condition (i) is shown in \ref{subfig:D31_convergence_Legendre_W7X_KJM_0204_Erho_3e-4_Detail} is $N_\xi=140$.
\begin{figure*}[t]
	\centering
	\begin{subfigure}[t]{0.32\textwidth}
		\tikzsetnextfilename{Convergence-Legendre-W7X-KJM-s0204-Er-3e-4-D11}
		\begin{tikzpicture}
	\begin{axis}[
		%		height=0.85\textwidth, 
		width=\textwidth, 
		scaled y ticks=base 10:3,
		y tick label style={
			/pgf/number format/.cd,
			fixed,
			fixed zerofill,
			precision=1,
			/tikz/.cd}, 
		xlabel = $N_\xi$, ylabel=$\widehat{D}_{11} $ ${[\text{m}]}$
		]
		
		
		\foreach \Nxi in {20, 40, ...,100}
		{		
			\addplot[blue, mark=+, only marks] table[skip first n=1, 
			x expr=\thisrowno{4},
			y expr=\thisrowno{5}*0.5132*0.5132,
			restrict expr to domain={\thisrowno{2}}{19:47},
			restrict expr to domain={\thisrowno{3}}{65:179},
			restrict expr to domain={\thisrowno{4}}{\Nxi:\Nxi}
			]{data/W7X-KJM/MONKES/DKE_zeta_Convergence_Example_Nxi_20/Gamma_11_Gamma_31_nu_0.100E-04_E_rho_0.300E-03.plt};
		}
		
		\foreach \Nxi in {120, 140, ...,200}{		
			\addplot[blue, mark=+, only marks] table[skip first n=1, 
			x expr=\thisrowno{4},
			y expr=\thisrowno{5}*0.5132*0.5132,
			restrict expr to domain={\thisrowno{2}}{19:47},
			restrict expr to domain={\thisrowno{3}}{65:179},
			restrict expr to domain={\thisrowno{4}}{\Nxi:\Nxi}
			]{data/W7X-KJM/MONKES/DKE_zeta_Convergence_Example_Nxi_120/Gamma_11_Gamma_31_nu_0.100E-04_E_rho_0.300E-03.plt};
		}	
		
		\addplot[forget plot, blue, mark=+, only marks] table[skip first n=1, 
		x expr=\thisrowno{4},
		y expr=\thisrowno{5}*0.5132*0.5132,
		restrict expr to domain={\thisrowno{4}}{200:340},
		restrict expr to domain={\thisrowno{1}}{3e-4:3e-4}
		]{data/W7X-KJM/MONKES/Convergence_Nxi/N_theta_47_N_zeta_153/monkes_Monoenergetic_Database.dat};
		
	\end{axis}
\end{tikzpicture}

		\caption{}
		\label{subfig:D11_convergence_Legendre_W7X_KJM_0204_Erho_3e-4}
	\end{subfigure}
	%	%\hfill
	\begin{subfigure}[t]{0.32\textwidth}
		\tikzsetnextfilename{Convergence-Legendre-W7X-KJM-s0204-Er-3e-4-D33}
		\begin{tikzpicture}
	\begin{axis}[
		%		height=0.85\textwidth, 
		width=\textwidth, 
		scaled y ticks=base 10:-4,
		y tick label style={
			/pgf/number format/.cd,
			fixed,
			fixed zerofill,
			precision=1,
			/tikz/.cd}, 
		xlabel = $N_\xi$, ylabel=$\widehat{D}_{33}$ ${[\text{m}]}$
		]
		
		
		\foreach \Nxi in {20, 40, ...,100}{		
			\addplot[blue, mark=+, only marks] table[skip first n=1, 
			x expr=\thisrowno{4},
			y expr=\thisrowno{8},
			restrict expr to domain={\thisrowno{2}}{19:47},
			restrict expr to domain={\thisrowno{3}}{65:179},
			restrict expr to domain={\thisrowno{4}}{\Nxi:\Nxi}
			]{data/W7X-KJM/MONKES/DKE_zeta_Convergence_Example_Nxi_20/Gamma_11_Gamma_31_nu_0.100E-04_E_rho_0.300E-03.plt};
		}
		
		\foreach \Nxi in {120, 140, ...,200}{		
			\addplot[blue, mark=+, only marks] table[skip first n=1, 
			x expr=\thisrowno{4},
			y expr=\thisrowno{8},
			restrict expr to domain={\thisrowno{2}}{27:47},
			restrict expr to domain={\thisrowno{3}}{65:179},
			restrict expr to domain={\thisrowno{4}}{\Nxi:\Nxi}
			]{data/W7X-KJM/MONKES/DKE_zeta_Convergence_Example_Nxi_120/Gamma_11_Gamma_31_nu_0.100E-04_E_rho_0.300E-03.plt};
		}	
		
		\addplot[forget plot, blue, mark=+, only marks] table[skip first n=1, 
		x expr=\thisrowno{4},
		y expr=-\thisrowno{8},
		restrict expr to domain={\thisrowno{4}}{200:340},
		restrict expr to domain={\thisrowno{1}}{3e-4:3e-4}
		]{data/W7X-KJM/MONKES/Convergence_Nxi/N_theta_47_N_zeta_153/monkes_Monoenergetic_Database.dat};
		
		
	\end{axis}
\end{tikzpicture}
		\caption{}
		\label{subfig:D33_convergence_Legendre_W7X_KJM_0204_Erho_3e-4}
	\end{subfigure}
  
    
    \begin{subfigure}[t]{0.32\textwidth}
    	\tikzsetnextfilename{Convergence-Legendre-W7X-KJM-s0204-Er-3e-4-D31-Detail}
    	\begin{tikzpicture}
	\begin{axis}[
		%		height=0.85\textwidth, 
		width=\textwidth, 
		scaled y ticks=base 10:1,
		y tick label style={
			/pgf/number format/.cd,
			fixed,
			fixed zerofill,
			precision=1,
			/tikz/.cd}, 
		xlabel = $N_\xi$, 
		ylabel=$\widehat{D}_{31}$ ${[\text{m}]}$,
		legend pos = south east, 
		ymin = 0.05
		]		
			
		\addplot[blue, mark=+, only marks] table[skip first n=1, 
		x expr=\thisrowno{4},
		y expr=\thisrowno{6}  *0.5132,
		restrict expr to domain={\thisrowno{4}}{200:340},
		restrict expr to domain={\thisrowno{1}}{3e-4:3e-4}
		]{data/W7X-KJM/MONKES/Convergence_Nxi/N_theta_47_N_zeta_153/monkes_Monoenergetic_Database.dat};
		\addlegendentry{$\widehat{D}_{31}$}
		
		
		
		\addplot[blue, mark=o, only marks] table[skip first n=1, 
		x expr=\thisrowno{4},
		y expr=-\thisrowno{7}  *0.5132,
		restrict expr to domain={\thisrowno{4}}{200:340},
		restrict expr to domain={\thisrowno{1}}{3e-4:3e-4}
		]{data/W7X-KJM/MONKES/Convergence_Nxi/N_theta_47_N_zeta_153/monkes_Monoenergetic_Database.dat};
		\addlegendentry{$-\widehat{D}_{13}$}
		
		\addplot[forget plot, name path=Upper2,blue!20, 
		domain = 20:340] {0.269E+00*1.05  *0.5132};	
		\addplot[forget plot, name path=Lower2,blue!20, 
		domain = 20:340] {0.269E+00*0.95  *0.5132};		
		\addplot[blue!20] fill between[of=Upper2 and Lower2];
		\addlegendentry{ $\mathcal{R}_5$ }			
		
		\addplot[forget plot, name path=Upper2,red!20, 
		domain = 20:340] {(0.2687694954228240E+00)*0.5132 + 5e-3};	
		\addplot[forget plot, name path=Lower2,red!20, 
		domain = 20:340] {(0.2687694954228240E+00)*0.5132 - 5e-3};		
		\addplot[red!20] fill between[of=Upper2 and Lower2];
		\addlegendentry{ $\mathcal{A}_{0.005}$ }
		
		
		\foreach \Nxi in {20,40,60,80,100}{		
			\addplot[forget plot, blue, mark=+, only marks] table[skip first n=1, 
			x expr=\thisrowno{4},
			y expr=\thisrowno{6}  *0.5132,
			restrict expr to domain={\thisrowno{2}}{27:47},
			restrict expr to domain={\thisrowno{3}}{65:179},
			restrict expr to domain={\thisrowno{4}}{\Nxi:\Nxi}
			]{data/W7X-KJM/MONKES/DKE_zeta_Convergence_Example_Nxi_20/Gamma_11_Gamma_31_nu_0.100E-04_E_rho_0.300E-03.plt};
			
			
			\addplot[forget plot, blue, mark=o, only marks] table[skip first n=1, 
			x expr=\thisrowno{4},
			y expr=-\thisrowno{7}  *0.5132,
			restrict expr to domain={\thisrowno{2}}{27:47},
			restrict expr to domain={\thisrowno{3}}{65:179},
			restrict expr to domain={\thisrowno{4}}{\Nxi:\Nxi}
			]{data/W7X-KJM/MONKES/DKE_zeta_Convergence_Example_Nxi_20/Gamma_11_Gamma_31_nu_0.100E-04_E_rho_0.300E-03.plt};
			
		}
		
		\foreach \Nxi in {120, 140, 160}{		
			\addplot[forget plot, blue, mark=+, only marks] table[skip first n=1, 
			x expr=\thisrowno{4},
			y expr=\thisrowno{6}  *0.5132,
			restrict expr to domain={\thisrowno{2}}{47:67},
			restrict expr to domain={\thisrowno{3}}{71:179},
			restrict expr to domain={\thisrowno{4}}{\Nxi:\Nxi}
			]{data/W7X-KJM/MONKES/DKE_zeta_Convergence_Example_Nxi_120/Gamma_11_Gamma_31_nu_0.100E-04_E_rho_0.300E-03.plt};
			
					
			\addplot[forget plot, blue, mark=o, only marks] table[skip first n=1, 
			x expr=\thisrowno{4},
			y expr=-\thisrowno{7}  *0.5132,
			restrict expr to domain={\thisrowno{2}}{47:67},
			restrict expr to domain={\thisrowno{3}}{71:179},
			restrict expr to domain={\thisrowno{4}}{\Nxi:\Nxi}
			]{data/W7X-KJM/MONKES/DKE_zeta_Convergence_Example_Nxi_120/Gamma_11_Gamma_31_nu_0.100E-04_E_rho_0.300E-03.plt};
			
		}	
		
		\addplot[ultra thick, Green, mark = star, mark size = 5 pt, only marks] table[skip first n=1, 
		x expr=\thisrowno{4},
		y expr=\thisrowno{6}  *0.5132,
		restrict expr to domain={\thisrowno{2}}{35:67},
		restrict expr to domain={\thisrowno{3}}{75:179},
		restrict expr to domain={\thisrowno{4}}{180:180}
		]{data/W7X-KJM/MONKES/DKE_zeta_Convergence_Example_Nxi_180/Monoenergetic_nu_0.100E-04_E_rho_0.300E-03.plt};
		\addlegendentry{Selected}	
		
			
		
	\end{axis}
\end{tikzpicture}

    	\caption{}
    	\label{subfig:D31_convergence_Legendre_W7X_KJM_0204_Erho_3e-4_Detail}
    \end{subfigure}
    %\hfill
    \begin{subfigure}[t]{0.32\textwidth}
    	\tikzsetnextfilename{Convergence-theta-zeta-W7X-KJM-s0204-Er-3e-4-D31}
    	\begin{tikzpicture}
	\begin{axis}[
		%		height=0.85\textwidth, 
		width=\textwidth, 
		scaled y ticks=base 10:1,
		%		xtick={80,120,160,200, 240},
		y tick label style={
			/pgf/number format/.cd,
			fixed,
			fixed zerofill,
			precision=1,
			/tikz/.cd}, 
		ymin = 0.21 *0.5132,
		ymax = 0.287 *0.5132,
		legend pos = south east, 
		legend columns =2, 
		xlabel = $N_\zeta$, 
		ylabel=$\widehat{D}_{31}$ ${[\text{m}]}$
		]
		
		\addplot[forget plot, name path=Upper2,blue!20, 
		domain = 35:131] {0.269E+00*1.05 *0.5132};	
		\addplot[forget plot, name path=Lower2,blue!20, 
		domain = 35:131] {0.269E+00*0.95 *0.5132};		
		\addplot[forget plot, blue!20] fill between[of=Upper2 and Lower2];
		
		\addplot[forget plot, name path=Upper2,red!20, 
		domain = 35:131] {(0.269E+00)  *0.5132+ 5e-3};	
		\addplot[forget plot, name path=Lower2,red!20, 
		domain = 35:131] {(0.269E+00)  *0.5132- 5e-3};		
		\addplot[forget plot, red!20] fill between[of=Upper2 and Lower2];
		
		\foreach \Ntheta in {15,19,...,27}{		
			\addplot+[no markers] table[skip first n=1, 
			x expr=\thisrowno{3},
			y expr=\thisrowno{6} *0.5132,
			restrict expr to domain={\thisrowno{2}}{\Ntheta:\Ntheta},
			restrict expr to domain={\thisrowno{3}}{35:131},
			restrict expr to domain={\thisrowno{4}}{180:200}
			]{results/W7X-KJM/0.204/DKE_zeta_Convergence_Example_Nxi_180/Monoenergetic_nu_0.100E-04_E_rho_0.300E-03.plt};
			\addlegendimage{empty legend}
			\addlegendentry{$N_\theta=$}
			\expandafter\addlegendentry\expandafter{\Ntheta}
		}	
		
		
		
		\foreach \Ntheta in {19}
		{		
			\addplot+[ultra thick, Green, mark = star, mark size = 5 pt, only marks] table[skip first n=1, 
			x expr=\thisrowno{3},
			y expr=\thisrowno{6} *0.5132,
			restrict expr to domain={\thisrowno{2}}{\Ntheta:\Ntheta},
			restrict expr to domain={\thisrowno{3}}{79:79},
			restrict expr to domain={\thisrowno{4}}{160:180}
			]{results/W7X-KJM/0.204/DKE_zeta_Convergence_Example_Nxi_180/Monoenergetic_nu_0.100E-04_E_rho_0.300E-03.plt};
		}
		
	\end{axis}
\end{tikzpicture}

    	\caption{}
    	\label{subfig:D31_convergence_theta_zeta_W7X_KJM_0204_Erho_3e-4_Detail}
    \end{subfigure}
    
    
	\caption{Convergence of monoenergetic coefficients with the number of Legendre modes $N_\xi$ for W7X-KJM at the surface labelled by $\psi/\psi_{\text{lcfs}}=0.204$, for $\hat{\nu}(v)=10^{-5}$ $\text{m}^{-1}$ and $\hat{E}_r(v)=3\cdot 10^{-4}$ $\text{kV}\cdot\text{s}/\text{m}^2$.}
	\label{fig:Convergence_W7X_KJM_Er_3e-4}
\end{figure*}
\begin{figure}
	\centering
	\begin{subfigure}[t]{0.33\textwidth}
		\tikzsetnextfilename{Convergence-Legendre-W7X-KJM-s0204-Er-3e-4-D31-Detail}
		\begin{tikzpicture}
	\begin{axis}[
		%		height=0.85\textwidth, 
		width=\textwidth, 
		scaled y ticks=base 10:1,
		y tick label style={
			/pgf/number format/.cd,
			fixed,
			fixed zerofill,
			precision=1,
			/tikz/.cd}, 
		xlabel = $N_\xi$, 
		ylabel=$\widehat{D}_{31}$ ${[\text{m}]}$,
		legend pos = south east, 
		ymin = 0.05
		]		
			
		\addplot[blue, mark=+, only marks] table[skip first n=1, 
		x expr=\thisrowno{4},
		y expr=\thisrowno{6}  *0.5132,
		restrict expr to domain={\thisrowno{4}}{200:340},
		restrict expr to domain={\thisrowno{1}}{3e-4:3e-4}
		]{data/W7X-KJM/MONKES/Convergence_Nxi/N_theta_47_N_zeta_153/monkes_Monoenergetic_Database.dat};
		\addlegendentry{$\widehat{D}_{31}$}
		
		
		
		\addplot[blue, mark=o, only marks] table[skip first n=1, 
		x expr=\thisrowno{4},
		y expr=-\thisrowno{7}  *0.5132,
		restrict expr to domain={\thisrowno{4}}{200:340},
		restrict expr to domain={\thisrowno{1}}{3e-4:3e-4}
		]{data/W7X-KJM/MONKES/Convergence_Nxi/N_theta_47_N_zeta_153/monkes_Monoenergetic_Database.dat};
		\addlegendentry{$-\widehat{D}_{13}$}
		
		\addplot[forget plot, name path=Upper2,blue!20, 
		domain = 20:340] {0.269E+00*1.05  *0.5132};	
		\addplot[forget plot, name path=Lower2,blue!20, 
		domain = 20:340] {0.269E+00*0.95  *0.5132};		
		\addplot[blue!20] fill between[of=Upper2 and Lower2];
		\addlegendentry{ $\mathcal{R}_5$ }			
		
		\addplot[forget plot, name path=Upper2,red!20, 
		domain = 20:340] {(0.2687694954228240E+00)*0.5132 + 5e-3};	
		\addplot[forget plot, name path=Lower2,red!20, 
		domain = 20:340] {(0.2687694954228240E+00)*0.5132 - 5e-3};		
		\addplot[red!20] fill between[of=Upper2 and Lower2];
		\addlegendentry{ $\mathcal{A}_{0.005}$ }
		
		
		\foreach \Nxi in {20,40,60,80,100}{		
			\addplot[forget plot, blue, mark=+, only marks] table[skip first n=1, 
			x expr=\thisrowno{4},
			y expr=\thisrowno{6}  *0.5132,
			restrict expr to domain={\thisrowno{2}}{27:47},
			restrict expr to domain={\thisrowno{3}}{65:179},
			restrict expr to domain={\thisrowno{4}}{\Nxi:\Nxi}
			]{data/W7X-KJM/MONKES/DKE_zeta_Convergence_Example_Nxi_20/Gamma_11_Gamma_31_nu_0.100E-04_E_rho_0.300E-03.plt};
			
			
			\addplot[forget plot, blue, mark=o, only marks] table[skip first n=1, 
			x expr=\thisrowno{4},
			y expr=-\thisrowno{7}  *0.5132,
			restrict expr to domain={\thisrowno{2}}{27:47},
			restrict expr to domain={\thisrowno{3}}{65:179},
			restrict expr to domain={\thisrowno{4}}{\Nxi:\Nxi}
			]{data/W7X-KJM/MONKES/DKE_zeta_Convergence_Example_Nxi_20/Gamma_11_Gamma_31_nu_0.100E-04_E_rho_0.300E-03.plt};
			
		}
		
		\foreach \Nxi in {120, 140, 160}{		
			\addplot[forget plot, blue, mark=+, only marks] table[skip first n=1, 
			x expr=\thisrowno{4},
			y expr=\thisrowno{6}  *0.5132,
			restrict expr to domain={\thisrowno{2}}{47:67},
			restrict expr to domain={\thisrowno{3}}{71:179},
			restrict expr to domain={\thisrowno{4}}{\Nxi:\Nxi}
			]{data/W7X-KJM/MONKES/DKE_zeta_Convergence_Example_Nxi_120/Gamma_11_Gamma_31_nu_0.100E-04_E_rho_0.300E-03.plt};
			
					
			\addplot[forget plot, blue, mark=o, only marks] table[skip first n=1, 
			x expr=\thisrowno{4},
			y expr=-\thisrowno{7}  *0.5132,
			restrict expr to domain={\thisrowno{2}}{47:67},
			restrict expr to domain={\thisrowno{3}}{71:179},
			restrict expr to domain={\thisrowno{4}}{\Nxi:\Nxi}
			]{data/W7X-KJM/MONKES/DKE_zeta_Convergence_Example_Nxi_120/Gamma_11_Gamma_31_nu_0.100E-04_E_rho_0.300E-03.plt};
			
		}	
		
		\addplot[ultra thick, Green, mark = star, mark size = 5 pt, only marks] table[skip first n=1, 
		x expr=\thisrowno{4},
		y expr=\thisrowno{6}  *0.5132,
		restrict expr to domain={\thisrowno{2}}{35:67},
		restrict expr to domain={\thisrowno{3}}{75:179},
		restrict expr to domain={\thisrowno{4}}{180:180}
		]{data/W7X-KJM/MONKES/DKE_zeta_Convergence_Example_Nxi_180/Monoenergetic_nu_0.100E-04_E_rho_0.300E-03.plt};
		\addlegendentry{Selected}	
		
			
		
	\end{axis}
\end{tikzpicture}

		\caption{}
		\label{subfig:D31_convergence_Legendre_W7X_KJM_0204_Erho_3e-4_Detail}
	\end{subfigure}
	%\hfill
	\begin{subfigure}[t]{0.33\textwidth}
		\tikzsetnextfilename{Convergence-theta-zeta-W7X-KJM-s0204-Er-3e-4-D31}
		\begin{tikzpicture}
	\begin{axis}[
		%		height=0.85\textwidth, 
		width=\textwidth, 
		scaled y ticks=base 10:1,
		%		xtick={80,120,160,200, 240},
		y tick label style={
			/pgf/number format/.cd,
			fixed,
			fixed zerofill,
			precision=1,
			/tikz/.cd}, 
		ymin = 0.21 *0.5132,
		ymax = 0.287 *0.5132,
		legend pos = south east, 
		legend columns =2, 
		xlabel = $N_\zeta$, 
		ylabel=$\widehat{D}_{31}$ ${[\text{m}]}$
		]
		
		\addplot[forget plot, name path=Upper2,blue!20, 
		domain = 35:131] {0.269E+00*1.05 *0.5132};	
		\addplot[forget plot, name path=Lower2,blue!20, 
		domain = 35:131] {0.269E+00*0.95 *0.5132};		
		\addplot[forget plot, blue!20] fill between[of=Upper2 and Lower2];
		
		\addplot[forget plot, name path=Upper2,red!20, 
		domain = 35:131] {(0.269E+00)  *0.5132+ 5e-3};	
		\addplot[forget plot, name path=Lower2,red!20, 
		domain = 35:131] {(0.269E+00)  *0.5132- 5e-3};		
		\addplot[forget plot, red!20] fill between[of=Upper2 and Lower2];
		
		\foreach \Ntheta in {15,19,...,27}{		
			\addplot+[no markers] table[skip first n=1, 
			x expr=\thisrowno{3},
			y expr=\thisrowno{6} *0.5132,
			restrict expr to domain={\thisrowno{2}}{\Ntheta:\Ntheta},
			restrict expr to domain={\thisrowno{3}}{35:131},
			restrict expr to domain={\thisrowno{4}}{180:200}
			]{results/W7X-KJM/0.204/DKE_zeta_Convergence_Example_Nxi_180/Monoenergetic_nu_0.100E-04_E_rho_0.300E-03.plt};
			\addlegendimage{empty legend}
			\addlegendentry{$N_\theta=$}
			\expandafter\addlegendentry\expandafter{\Ntheta}
		}	
		
		
		
		\foreach \Ntheta in {19}
		{		
			\addplot+[ultra thick, Green, mark = star, mark size = 5 pt, only marks] table[skip first n=1, 
			x expr=\thisrowno{3},
			y expr=\thisrowno{6} *0.5132,
			restrict expr to domain={\thisrowno{2}}{\Ntheta:\Ntheta},
			restrict expr to domain={\thisrowno{3}}{79:79},
			restrict expr to domain={\thisrowno{4}}{160:180}
			]{results/W7X-KJM/0.204/DKE_zeta_Convergence_Example_Nxi_180/Monoenergetic_nu_0.100E-04_E_rho_0.300E-03.plt};
		}
		
	\end{axis}
\end{tikzpicture}

		\caption{}
		\label{subfig:D31_convergence_theta_zeta_W7X_KJM_0204_Erho_3e-4_Detail}
	\end{subfigure}
	%	%\hfill
	\begin{subfigure}[t]{0.33\textwidth}
		\tikzsetnextfilename{Clock-time-W7X-KJM-s0204-Er-3e-4-D31}
		\begin{tikzpicture}
	\begin{axis}[
		%		height=0.85\textwidth, 
		width=0.97\textwidth, 
		%		scaled y ticks=base 10:-1,
		extra x ticks={79},
		extra y ticks={46},
		y tick label style={
			/pgf/number format/.cd,
			fixed,
			fixed zerofill,
			precision=0,
			/tikz/.cd}, 
		extra tick style={grid=major, grid style={dashed,black}}, 
		%		ymin = 0.053,
		%		ymax = 0.102,
		legend pos = south east, 
		legend columns =2, 
		xlabel = $N_\zeta$, ylabel=Wall-clock time {[s]}
		]
		
		
		%		\addplot[forget plot, name path=Upper2,blue!20, 
		%		domain = 15:71] {0.3593392335766615E+00*1.05};	
		%		\addplot[forget plot, name path=Lower2,blue!20, 
		%		domain = 15:71] {0.3593392335766615E+00*0.95};		
		%		\addplot[forget plot, blue!20] fill between[of=Upper2 and Lower2];
		
		\foreach \Ntheta in {19}{		
			\addplot+[red, no markers] table[skip first n=1, 
			x expr=\thisrowno{3},
			y expr=\thisrowno{10},
			restrict expr to domain={\thisrowno{2}}{\Ntheta:\Ntheta},
			restrict expr to domain={\thisrowno{3}}{21:131},
			restrict expr to domain={\thisrowno{4}}{180:180}
			]{results/W7X-KJM/0.204/DKE_zeta_Convergence_Example_Nxi_180/Monoenergetic_nu_0.100E-04_E_rho_0.300E-03.plt};
			\addlegendimage{empty legend}
			\addlegendentry{$N_\theta=$}
			\expandafter\addlegendentry\expandafter{\Ntheta}
		}	
		
		
		
		\foreach \Ntheta in {19}{		
			\addplot+[ultra thick, Green, mark = star, mark size = 5 pt, only marks] table[skip first n=1, 
			x expr=\thisrowno{3},
			y expr=\thisrowno{10},
			restrict expr to domain={\thisrowno{2}}{\Ntheta:\Ntheta},
			restrict expr to domain={\thisrowno{3}}{79:79},
			restrict expr to domain={\thisrowno{4}}{180:180}
			]{results/W7X-KJM/0.204/DKE_zeta_Convergence_Example_Nxi_180/Monoenergetic_nu_0.100E-04_E_rho_0.300E-03.plt};
		}
		
	\end{axis}
\end{tikzpicture}

		\caption{}
		\label{subfig:D31_Clock_time_W7X_KJM_0204_Erho_3e-4}
	\end{subfigure}
	\caption{Selection of the resolution to have a sufficiently accurate calculation of the parallel flow geometric coefficient $\widehat{D}_{31}$ for W7X-KJM at the surface labelled by $\psi/\psi_{\text{lcfs}}=0.200$, for $\hat{\nu}(v)=10^{-5}$ $\text{m}^{-1}$ and $\hat{E}_r(v)=3\cdot 10^{-4}$ $\text{kV}\cdot\text{s}/\text{m}^2$.}
\end{figure}


%


\begin{figure*}
	\begin{subfigure}[t]{0.33\textwidth}
		\tikzsetnextfilename{Convergence-Legendre-NCSX-s0200-Er-0-D11}
		\begin{tikzpicture}
	\begin{axis}[
		%		height=0.85\textwidth, 
		width=0.95\textwidth, 
		scaled y ticks=base 10:0,
		y tick label style={
			/pgf/number format/.cd,
			fixed,
			fixed zerofill,
			precision=2,
			/tikz/.cd}, 
		xlabel = $N_\xi$, ylabel=$\widehat{D}_{11} {K}_{11}$
		]
				
		\foreach \Nxi in {20, 40, ...,120}{		
			\addplot[blue, mark=+, only marks] table[skip first n=1, 
			x expr=\thisrowno{4},
			y expr=\thisrowno{5},
			restrict expr to domain={\thisrowno{2}}{19:47},
			restrict expr to domain={\thisrowno{3}}{65:179},
			restrict expr to domain={\thisrowno{4}}{\Nxi:\Nxi}
			]{results/NCSX/0.200/DKE_zeta_Convergence_Example_Nxi_20/Monoenergetic_nu_0.100E-04_E_rho_0.000E+00.plt};
		}
		
		\foreach \Nxi in {140,160,...,200}{		
			\addplot[blue, mark=+, only marks] table[skip first n=1, 
			x expr=\thisrowno{4},
			y expr=\thisrowno{5},
			restrict expr to domain={\thisrowno{2}}{19:47},
			restrict expr to domain={\thisrowno{3}}{65:179},
			restrict expr to domain={\thisrowno{4}}{\Nxi:\Nxi}
			]{results/NCSX/0.200/DKE_zeta_Convergence_Example_Nxi_\Nxi/Monoenergetic_nu_0.100E-04_E_rho_0.000E+00.plt};
		}	
		
		
	\end{axis}
\end{tikzpicture}

	\end{subfigure}
	%\hfill
	\begin{subfigure}[t]{0.33\textwidth}
		\tikzsetnextfilename{Convergence-Legendre-NCSX-s0200-Er-0-D31}
		\begin{tikzpicture}
	\begin{axis}[
		%		height=0.85\textwidth, 
		width=0.95\textwidth, 
		scaled y ticks=base 10:0,
		y tick label style={
			/pgf/number format/.cd,
			fixed,
			fixed zerofill,
			precision=2,
			/tikz/.cd}, 
		xlabel = $N_\xi$, ylabel=$\widehat{D}_{31} {K}_{31}$,
		legend pos = south east
		]
		
		
		\foreach \Nxi in {20, 40, ...,120}{		
			\addplot[blue, mark=+, only marks] table[skip first n=1, 
			x expr=\thisrowno{4},
			y expr=\thisrowno{6},
			restrict expr to domain={\thisrowno{2}}{31:47},
			restrict expr to domain={\thisrowno{3}}{75:179},
			restrict expr to domain={\thisrowno{4}}{\Nxi:\Nxi}
			]{results/NCSX/0.200/DKE_zeta_Convergence_Example_Nxi_20/Monoenergetic_nu_0.100E-04_E_rho_0.000E+00.plt};
			
			\addplot[blue, mark=o, only marks] table[skip first n=1, 
			x expr=\thisrowno{4},
			y expr=-\thisrowno{7},
			restrict expr to domain={\thisrowno{2}}{31:47},
			restrict expr to domain={\thisrowno{3}}{75:179},
			restrict expr to domain={\thisrowno{4}}{\Nxi:\Nxi}
			]{results/NCSX/0.200/DKE_zeta_Convergence_Example_Nxi_20/Monoenergetic_nu_0.100E-04_E_rho_0.000E+00.plt};
		}
		
		\foreach \Nxi in {140, 160, ...,200}{		
			\addplot[blue, mark=+, only marks] table[skip first n=1, 
			x expr=\thisrowno{4},
			y expr=\thisrowno{6},
			restrict expr to domain={\thisrowno{2}}{31:47},
			restrict expr to domain={\thisrowno{3}}{75:179},
			restrict expr to domain={\thisrowno{4}}{\Nxi:\Nxi}
			]{results/NCSX/0.200/DKE_zeta_Convergence_Example_Nxi_\Nxi/Monoenergetic_nu_0.100E-04_E_rho_0.000E+00.plt};
			
			\addplot[blue, mark=o, only marks] table[skip first n=1, 
			x expr=\thisrowno{4},
			y expr=-\thisrowno{7},
			restrict expr to domain={\thisrowno{2}}{31:47},
			restrict expr to domain={\thisrowno{3}}{75:179},
			restrict expr to domain={\thisrowno{4}}{\Nxi:\Nxi}
			]{results/NCSX/0.200/DKE_zeta_Convergence_Example_Nxi_\Nxi/Monoenergetic_nu_0.100E-04_E_rho_0.000E+00.plt};
		}	
		
		\addlegendentry{$\widehat{D}_{31} {K}_{31}$}
		\addlegendentry{$-\widehat{D}_{13} {K}_{31}$}
	\end{axis}
\end{tikzpicture}

	\end{subfigure}
	%	%\hfill
	\begin{subfigure}[t]{0.33\textwidth}
		\tikzsetnextfilename{Convergence-Legendre-NCSX-s0200-Er-0-D33}
		\begin{tikzpicture}
	\begin{axis}[
		%		height=0.85\textwidth, 
		width=0.95\textwidth, 
		scaled y ticks=base 10:-4,
		y tick label style={
			/pgf/number format/.cd,
			fixed,
			fixed zerofill,
			precision=2,
			/tikz/.cd}, 
		xlabel = $N_\xi$, ylabel=$\widehat{D}_{33} {K}_{33}$
		]
		
		\foreach \Nxi in {20, 40, ...,120}{		
			\addplot[blue, mark=+, only marks] table[skip first n=1, 
			x expr=\thisrowno{4},
			y expr=\thisrowno{8},
			restrict expr to domain={\thisrowno{2}}{19:47},
			restrict expr to domain={\thisrowno{3}}{65:179},
			restrict expr to domain={\thisrowno{4}}{\Nxi:\Nxi}
			]{results/NCSX/0.200/DKE_zeta_Convergence_Example_Nxi_20/Monoenergetic_nu_0.100E-04_E_rho_0.000E+00.plt};
		}
		
		\foreach \Nxi in {140,160,...,200}{		
			\addplot[blue, mark=+, only marks] table[skip first n=1, 
			x expr=\thisrowno{4},
			y expr=\thisrowno{8},
			restrict expr to domain={\thisrowno{2}}{19:47},
			restrict expr to domain={\thisrowno{3}}{65:179},
			restrict expr to domain={\thisrowno{4}}{\Nxi:\Nxi}
			]{results/NCSX/0.200/DKE_zeta_Convergence_Example_Nxi_\Nxi/Monoenergetic_nu_0.100E-04_E_rho_0.000E+00.plt};
		}	
		
		
	\end{axis}
\end{tikzpicture}

	\end{subfigure}
	\caption{Convergence of monoenergetic coefficients with the number of Legendre modes $N_\xi$ for NCSX at the surface labelled by $\psi/\psi_{\text{lcfs}}=0.200$, for $\hat{\nu}(v)=10^{-5}$ $\text{m}^{-1}$ and $\hat{E}_r(v)=0$ $\text{kV}\cdot\text{s}/\text{m}^2$.}
\end{figure*}
%
\begin{figure*}
	\begin{subfigure}[t]{0.33\textwidth}
		\tikzsetnextfilename{Convergence-Legendre-NCSX-s0200-Er-0-D31-Detail}
		
		\begin{tikzpicture}
	\begin{axis}[
		%		height=0.85\textwidth, 
		width=0.95\textwidth, 
		scaled y ticks=base 10:0,
		y tick label style={
			/pgf/number format/.cd,
			fixed,
			fixed zerofill,
			precision=2,
			/tikz/.cd}, 
		ymin = 1.5,
		xlabel = $N_\xi$, ylabel=$\widehat{D}_{31}{K}_{31}$,
		legend pos = south east
		]
			
		
		\addplot[forget plot, name path=Upper2,blue!20, 
		domain = 40:200] {0.2369500222373034E+01*1.05};	
		\addplot[forget plot, name path=Lower2,blue!20, 
		domain = 40:200] {0.2369500222373034E+01*0.95};		
		\addplot[blue!20] fill between[of=Upper2 and Lower2];
		\addlegendentry{ $\mathcal{R}_5$ }	
		
		\foreach \Nxi in {60,80,100}{		
			\addplot[forget plot, blue, mark=+, only marks] table[skip first n=1, 
			x expr=\thisrowno{4},
			y expr=\thisrowno{6},
			restrict expr to domain={\thisrowno{2}}{47:55},
			restrict expr to domain={\thisrowno{3}}{75:179},
			restrict expr to domain={\thisrowno{4}}{\Nxi:\Nxi}
			]{results/NCSX/0.200/DKE_zeta_Convergence_Example_Nxi_20/Monoenergetic_nu_0.100E-04_E_rho_0.000E+00.plt};			
		}
		
		\foreach \Nxi in {160, 180, ...,200}{		
			\addplot[forget plot, blue, mark=+, only marks] table[skip first n=1, 
			x expr=\thisrowno{4},
			y expr=\thisrowno{6},
			restrict expr to domain={\thisrowno{2}}{47:67},
			restrict expr to domain={\thisrowno{3}}{111:179},
			restrict expr to domain={\thisrowno{4}}{\Nxi:\Nxi}
			]{results/NCSX/0.200/DKE_zeta_Convergence_Example_Nxi_\Nxi/Monoenergetic_nu_0.100E-04_E_rho_0.000E+00.plt};
			
		}	
		
		\addplot[ultra thick, Green, mark = star, mark size = 5 pt, only marks] table[skip first n=1,  
		x expr=\thisrowno{4},
		y expr=\thisrowno{6},
		restrict expr to domain={\thisrowno{2}}{55:55},
		restrict expr to domain={\thisrowno{3}}{95:115},
		restrict expr to domain={\thisrowno{4}}{140:140}
		]{results/NCSX/0.200/DKE_zeta_Convergence_Example_Nxi_140/Monoenergetic_nu_0.100E-04_E_rho_0.000E+00.plt};
		\addlegendentry{Selected}	
				
	\end{axis}
\end{tikzpicture}

	\end{subfigure}
	%\hfill
	\begin{subfigure}[t]{0.33\textwidth}
		\tikzsetnextfilename{Convergence-theta-zeta-NCSX-s0200-Er-0-D31}
		\begin{tikzpicture}
	\begin{axis}[
		%		height=0.85\textwidth, 
		width=0.97\textwidth, 
		scaled y ticks=base 10:0,
		%		xtick={80,120,160,200, 240},
		y tick label style={
			/pgf/number format/.cd,
			fixed,
			fixed zerofill,
			precision=2,
			/tikz/.cd}, 
		ymin = 1.8,
		legend pos = south east, 
		legend columns =2, 
		xlabel = $N_\zeta$, ylabel=$\widehat{D}_{31}{K}_{31}$
		]
		
		\addplot[forget plot, name path=Upper2,blue!20, 
		domain = 19:111] {0.2369500222373034E+01*1.05};	
		\addplot[forget plot, name path=Lower2,blue!20, 
		domain = 19:111] {0.2369500222373034E+01*0.95};		
		\addplot[forget plot, blue!20] fill between[of=Upper2 and Lower2];
		
		\foreach \Ntheta in {23,27,...,35}{		
			\addplot+[no markers] table[skip first n=1, 
			x expr=\thisrowno{3},
			y expr=\thisrowno{6},
			restrict expr to domain={\thisrowno{2}}{\Ntheta:\Ntheta},
			restrict expr to domain={\thisrowno{3}}{23:111},
			restrict expr to domain={\thisrowno{4}}{140:140}
			]{results/NCSX/0.200/DKE_zeta_Convergence_Example_Nxi_140/Monoenergetic_nu_0.100E-04_E_rho_0.000E+00.plt};
			\addlegendimage{empty legend}
			\addlegendentry{$N_\theta=$}
			\expandafter\addlegendentry\expandafter{\Ntheta}
		}	
		
		\foreach \Ntheta in {27}{		
			\addplot+[ultra thick, Green, mark = star, mark size = 5 pt, only marks] table[skip first n=1, 
			x expr=\thisrowno{3},
			y expr=\thisrowno{6},
			restrict expr to domain={\thisrowno{2}}{\Ntheta:\Ntheta},
			restrict expr to domain={\thisrowno{3}}{75:75},
			restrict expr to domain={\thisrowno{4}}{140:140}
			]{results/NCSX/0.200/DKE_zeta_Convergence_Example_Nxi_140/Monoenergetic_nu_0.100E-04_E_rho_0.000E+00.plt};
		}
		%		\foreach \Ntheta in {15}{		
		%			\addplot+[only marks, mark = oplus, mark size = 4 pt, blue ] table[skip first n=1, 
		%			x expr=\thisrowno{3},
		%			y expr=\thisrowno{6},
		%			restrict expr to domain={\thisrowno{2}}{\Ntheta:\Ntheta},
		%			restrict expr to domain={\thisrowno{3}}{119:119}
		%			]{results/CIEMAT-QI/0.250/DKE_zeta_Convergence_Example_Nxi_180/Monoenergetic_nu_0.100E-04_E_rho_0.000E+00.plt};
		
		%		}	
		%		\addlegendentry{Spread of 5\%}
		
		
	\end{axis}
\end{tikzpicture}

	\end{subfigure}
	%	%\hfill
	\begin{subfigure}[t]{0.33\textwidth}
		\tikzsetnextfilename{Clock-time-NCSX-s0200-Er-0-D31}
		\begin{tikzpicture}
	\begin{axis}[
		%		height=0.85\textwidth, 
		width=0.97\textwidth, 
		scaled y ticks=base 10:0,
		%		xtick={80,120,160,200, 240},
		extra y ticks={0.8289730000000000E+02},
		extra tick style={grid=major, grid style={dashed,black}}, 
		y tick label style={
			/pgf/number format/.cd,
			fixed,
			fixed zerofill,
			precision=0,
			/tikz/.cd}, 
		ymin = 1.8,
		legend pos = south east, 
		legend columns =2, 
		xlabel = $N_\zeta$, ylabel=Wall-clock time {[s]}
		]
		
		\foreach \Ntheta in {27}{		
			\addplot+[no markers] table[skip first n=1, 
			x expr=\thisrowno{3},
			y expr=\thisrowno{10},
			restrict expr to domain={\thisrowno{2}}{\Ntheta:\Ntheta},
			restrict expr to domain={\thisrowno{3}}{23:99},
			restrict expr to domain={\thisrowno{4}}{140:140}
			]{results/NCSX/0.200/DKE_zeta_Convergence_Example_Nxi_140/Monoenergetic_nu_0.100E-04_E_rho_0.000E+00.plt};
			\addlegendimage{empty legend}
			\addlegendentry{$N_\theta=$}
			\expandafter\addlegendentry\expandafter{\Ntheta}
		}	
		
		\foreach \Ntheta in {27}{		
			\addplot+[ultra thick, Green, mark = star, mark size = 5 pt, only marks] table[skip first n=1, 
			x expr=\thisrowno{3},
			y expr=\thisrowno{10},
			restrict expr to domain={\thisrowno{2}}{\Ntheta:\Ntheta},
			restrict expr to domain={\thisrowno{3}}{75:75},
			restrict expr to domain={\thisrowno{4}}{140:140}
			]{results/NCSX/0.200/DKE_zeta_Convergence_Example_Nxi_140/Monoenergetic_nu_0.100E-04_E_rho_0.000E+00.plt};
		}
		%		\foreach \Ntheta in {15}{		
		%			\addplot+[only marks, mark = oplus, mark size = 4 pt, blue ] table[skip first n=1, 
		%			x expr=\thisrowno{3},
		%			y expr=\thisrowno{6},
		%			restrict expr to domain={\thisrowno{2}}{\Ntheta:\Ntheta},
		%			restrict expr to domain={\thisrowno{3}}{119:119}
		%			]{results/CIEMAT-QI/0.250/DKE_zeta_Convergence_Example_Nxi_180/Monoenergetic_nu_0.100E-04_E_rho_0.000E+00.plt};
		
		%		}	
		%		\addlegendentry{Spread of 5\%}
		
		
	\end{axis}
\end{tikzpicture}

	\end{subfigure}
	\caption{Selection of the resolution to have a sufficiently accurate calculation of the parallel flow geometric coefficient $\widehat{D}_{31}$ for NCSX at the surface labelled by $\psi/\psi_{\text{lcfs}}=0.200$, for $\hat{\nu}(v)=10^{-5}$ $\text{m}^{-1}$ and $\hat{E}_r(v)=0$ $\text{kV}\cdot\text{s}/\text{m}^2$. TO DO: SUBSTITUTE THE CLOCK TIME PLOT.}
\end{figure*}
\FloatBarrier
The monoenergetic coefficients of the flat mirror configuration CIEMAT-QI are more difficult to converge due to their smaller absolute value. 
\begin{figure*}[t]
	\centering
	\begin{subfigure}[t]{0.32\textwidth}
		\tikzsetnextfilename{Convergence-Legendre-CIEMAT-QI-s0250-Er-0-D11}
		\begin{tikzpicture}
	\begin{axis}[
		%		height=0.85\textwidth, 
		width=\textwidth, 
		scaled y ticks=base 10:2,
		y tick label style={
			/pgf/number format/.cd,
			fixed,
			fixed zerofill,
			precision=1,
			/tikz/.cd}, 
		xlabel = $N_\xi$, ylabel=$\widehat{D}_{11} $ ${[\text{m}]}$
		]
		
		\addplot[blue, mark=+, only marks] table[skip first n=1, 
		x expr=\thisrowno{4},
		y expr=\thisrowno{5}*0.4674*0.4674,
		restrict expr to domain={\thisrowno{2}}{15:47},
		restrict expr to domain={\thisrowno{3}}{75:97}
		]{data/CIEMAT-QI/MONKES/DKE_zeta_Convergence_Example_Nxi_20/Gamma_11_Gamma_31_nu_0.100E-04_E_rho_0.000E+00.plt};
		
		
		\foreach \Nxi in {120,140,160}{		
			\addplot[blue, mark=+, only marks] table[skip first n=1, 
			x expr=\thisrowno{4},
			y expr=\thisrowno{5}*0.4674*0.4674,
			restrict expr to domain={\thisrowno{2}}{35:47},
			restrict expr to domain={\thisrowno{3}}{165:179}
			]{data/CIEMAT-QI/MONKES/DKE_zeta_Convergence_Example_Nxi_\Nxi/Gamma_11_Gamma_31_nu_0.100E-04_E_rho_0.000E+00.plt};
		}	
	
	    
	    \foreach \Nxi in {180,200}{		
	    	\addplot[blue, mark=+, only marks] table[skip first n=1, 
	    	x expr=\thisrowno{4},
	    	y expr=\thisrowno{5}*0.4674*0.4674,
	    	restrict expr to domain={\thisrowno{2}}{31:47},
	    	restrict expr to domain={\thisrowno{3}}{165:229}
	    	]{data/CIEMAT-QI/MONKES/DKE_zeta_Convergence_Example_Nxi_\Nxi/Monoenergetic_nu_0.100E-04_E_rho_0.000E+00.plt};
	    }
    
        
        \addplot[forget plot, blue, mark=+, only marks] table[skip first n=1, 
        x expr=\thisrowno{4},
        y expr=\thisrowno{5}*0.4674*0.4674,
        restrict expr to domain={\thisrowno{1}}{0:0},
        restrict expr to domain={\thisrowno{4}}{220:400}
        ]{data/CIEMAT-QI/MONKES/Convergence_Nxi/N_theta_47_N_zeta_215/monkes_Monoenergetic_Database.dat};
        
        
	\end{axis}
\end{tikzpicture}

		\caption{}
		\label{subfig:D11_convergence_Legendre_CIEMAT_QI_0250_Erho_0}
	\end{subfigure}
	%	%\hfill
	\begin{subfigure}[t]{0.32\textwidth}
		\tikzsetnextfilename{Convergence-Legendre-CIEMAT-QI-s0250-Er-0-D33}
		\begin{tikzpicture}
	\begin{axis}[
		%		height=0.85\textwidth, 
		width=\textwidth, 
		scaled y ticks=base 10:-4,
%		ytick={0.9e4,1.2e4,1.5e4,1.8e4},
		y tick label style={
			/pgf/number format/.cd,
			fixed,
			fixed zerofill,
			precision=1,
			/tikz/.cd}, 
		xlabel = $N_\xi$, ylabel=$\widehat{D}_{33}$ ${[\text{m}]}$
		]
		
		\addplot[blue, mark=+, only marks] table[skip first n=1, 
		x expr=\thisrowno{4},
		y expr=\thisrowno{8},
		restrict expr to domain={\thisrowno{2}}{15:47},
		restrict expr to domain={\thisrowno{3}}{85:97}
		]{results/CIEMAT-QI/0.250/DKE_zeta_Convergence_Example_Nxi_20/Gamma_11_Gamma_31_nu_0.100E-04_E_rho_0.000E+00.plt};
				
		\foreach \Nxi in {120,140,160}{		
			\addplot[blue, mark=+, only marks] table[skip first n=1, 
			x expr=\thisrowno{4},
			y expr=abs(\thisrowno{8}),
			restrict expr to domain={\thisrowno{2}}{15:47},
			restrict expr to domain={\thisrowno{3}}{135:179}
			]{results/CIEMAT-QI/0.250/DKE_zeta_Convergence_Example_Nxi_\Nxi/Gamma_11_Gamma_31_nu_0.100E-04_E_rho_0.000E+00.plt};
		}		
	    	    
	    \foreach \Nxi in {180,200}{		
	    	\addplot[blue, mark=+, only marks] table[skip first n=1, 
	    	x expr=\thisrowno{4},
	    	y expr=abs(\thisrowno{8}),
	    	restrict expr to domain={\thisrowno{2}}{25:47},
	    	restrict expr to domain={\thisrowno{3}}{175:279}
	    	]{results/CIEMAT-QI/0.250/DKE_zeta_Convergence_Example_Nxi_\Nxi/Monoenergetic_nu_0.100E-04_E_rho_0.000E+00.plt};
	    }	
    
        
        \addplot[forget plot, blue, mark=+, only marks] table[skip first n=1, 
        x expr=\thisrowno{4},
        y expr=abs(\thisrowno{8}),
        restrict expr to domain={\thisrowno{1}}{0:0},
        restrict expr to domain={\thisrowno{4}}{220:400}
        ]{results/CIEMAT-QI/0.250/Convergence_nu_1e-5/Convergence_Nxi/N_theta_23_N_zeta_163/monkes_Monoenergetic_Database.dat};
        
        
	\end{axis}
\end{tikzpicture}

		\caption{}
		\label{subfig:D33_convergence_Legendre_CIEMAT_QI_0250_Erho_0}
	\end{subfigure}


    
    \begin{subfigure}[t]{0.32\textwidth}
    	\tikzsetnextfilename{Convergence-Legendre-CIEMAT-QI-s0250-Er-0-D31-Detail}
    	\begin{tikzpicture}
	\begin{axis}[
		%		height=0.85\textwidth, 
		width=\textwidth, 
		scaled y ticks=base 10:2,
		xtick={80,180,280,...,480},
		y tick label style={
			/pgf/number format/.cd,
			fixed,
			fixed zerofill,
			precision=1,
			/tikz/.cd}, 
		ymax = 0.0999,
		xlabel = $N_\xi$, ylabel=$\widehat{D}_{31}$ ${[\text{m}]}$,
		legend columns=1, legend pos = north east
		]
		\addplot[forget plot, blue, mark=+, only marks] table[skip first n=1, 
		x expr=\thisrowno{4},
		y expr=\thisrowno{6}  * 0.4674,
		restrict expr to domain={\thisrowno{4}}{80:100},
		restrict expr to domain={\thisrowno{2}}{19:47},
		restrict expr to domain={\thisrowno{3}}{91:197}
		]{results/CIEMAT-QI/0.250/DKE_zeta_Convergence_Example_Nxi_20/Gamma_11_Gamma_31_nu_0.100E-04_E_rho_0.000E+00.plt};
		
		
		\addplot[forget plot, blue, mark=o, only marks] table[skip first n=1, 
		x expr=\thisrowno{4},
		y expr=-\thisrowno{7}  * 0.4674,
		restrict expr to domain={\thisrowno{4}}{80:100},
		restrict expr to domain={\thisrowno{2}}{19:47},
		restrict expr to domain={\thisrowno{3}}{91:197}
		]{results/CIEMAT-QI/0.250/DKE_zeta_Convergence_Example_Nxi_20/Gamma_11_Gamma_31_nu_0.100E-04_E_rho_0.000E+00.plt};
		
		
		\foreach \Nxi in {120}{		
			\addplot[forget plot, blue, mark=+, only marks] table[skip first n=1, 
			x expr=\thisrowno{4},
			y expr=\thisrowno{6}  * 0.4674,
			restrict expr to domain={\thisrowno{2}}{27:47},
			restrict expr to domain={\thisrowno{3}}{149:230}
			]{results/CIEMAT-QI/0.250/DKE_zeta_Convergence_Example_Nxi_\Nxi/Gamma_11_Gamma_31_nu_0.100E-04_E_rho_0.000E+00.plt};	
			
			\addplot[forget plot, blue, mark=o, only marks] table[skip first n=1, 
			x expr=\thisrowno{4},
			y expr=-\thisrowno{7}  * 0.4674,
			restrict expr to domain={\thisrowno{2}}{27:47},
			restrict expr to domain={\thisrowno{3}}{149:230}
			]{results/CIEMAT-QI/0.250/DKE_zeta_Convergence_Example_Nxi_\Nxi/Gamma_11_Gamma_31_nu_0.100E-04_E_rho_0.000E+00.plt};
		}	
		
		\foreach \Nxi in {140, 160}{		
			\addplot[forget plot, blue, mark=+, only marks] table[skip first n=1, 
			x expr=\thisrowno{4},
			y expr=\thisrowno{6}  * 0.4674,
			restrict expr to domain={\thisrowno{2}}{27:47},
			restrict expr to domain={\thisrowno{3}}{169:230}
			]{results/CIEMAT-QI/0.250/DKE_zeta_Convergence_Example_Nxi_\Nxi/Gamma_11_Gamma_31_nu_0.100E-04_E_rho_0.000E+00.plt};
			
					
			\addplot[forget plot, blue, mark=o, only marks] table[skip first n=1, 
			x expr=\thisrowno{4},
			y expr=-\thisrowno{7}  * 0.4674,
			restrict expr to domain={\thisrowno{2}}{27:47},
			restrict expr to domain={\thisrowno{3}}{169:230}
			]{results/CIEMAT-QI/0.250/DKE_zeta_Convergence_Example_Nxi_\Nxi/Gamma_11_Gamma_31_nu_0.100E-04_E_rho_0.000E+00.plt};
			
		}	
		
		\foreach \Nxi in {200}{		
			\addplot[forget plot, blue, mark=+, only marks] table[skip first n=1, 
			x expr=\thisrowno{4},
			y expr=\thisrowno{6}  * 0.4674,
			restrict expr to domain={\thisrowno{2}}{27:47},
			restrict expr to domain={\thisrowno{3}}{189:290}
			]{results/CIEMAT-QI/0.250/DKE_zeta_Convergence_Example_Nxi_\Nxi/Monoenergetic_nu_0.100E-04_E_rho_0.000E+00.plt};	
			
			
			\addplot[forget plot, blue, mark=o, only marks] table[skip first n=1, 
			x expr=\thisrowno{4},
			y expr=-\thisrowno{7}  * 0.4674,
			restrict expr to domain={\thisrowno{2}}{27:47},
			restrict expr to domain={\thisrowno{3}}{189:290}
			]{results/CIEMAT-QI/0.250/DKE_zeta_Convergence_Example_Nxi_\Nxi/Monoenergetic_nu_0.100E-04_E_rho_0.000E+00.plt};
		}	
		    
	    \addplot[blue, mark=+, only marks] table[skip first n=1, 
	    x expr=\thisrowno{4},
	    y expr=\thisrowno{6} * 0.4674,
	    restrict expr to domain={\thisrowno{1}}{0:0},
	    restrict expr to domain={\thisrowno{4}}{220:400}
	    ]{results/CIEMAT-QI/0.250/Convergence_nu_1e-5/Convergence_Nxi/N_theta_47_N_zeta_215/monkes_Monoenergetic_Database.dat};
	    \addlegendentry{$\widehat{D}_{31}$}
	    
	    \addplot[blue, mark=o, only marks] table[skip first n=1, 
	    x expr=\thisrowno{4},
	    y expr=-\thisrowno{7} * 0.4674,
	    restrict expr to domain={\thisrowno{1}}{0:0},
	    restrict expr to domain={\thisrowno{4}}{220:400}
	    ]{results/CIEMAT-QI/0.250/Convergence_nu_1e-5/Convergence_Nxi/N_theta_47_N_zeta_215/monkes_Monoenergetic_Database.dat};
	    \addlegendentry{$-\widehat{D}_{13}$}
	    
	    
	    
	    \addplot[forget plot, name path=Upper2,red!20, 
	    domain = 100:400] {(0.888E-01)  * 0.4674+ 5e-3};	
	    \addplot[forget plot, name path=Lower2,red!20, 
	    domain = 100:400] {(0.888E-01)  * 0.4674- 5e-3};			
	    \addplot[red!20] fill between[of=Upper2 and Lower2];
	    \addlegendentry{ $\mathcal{A}_{0.005}$ }	
	    
		\addplot[forget plot, name path=Upper2,blue!20, 
		domain = 100:400] {0.888E-01*1.07 * 0.4674};	
		\addplot[forget plot, name path=Lower2,blue!20, 
		domain = 100:400] {0.888E-01*0.93 * 0.4674};		
		\addplot[blue!20] fill between[of=Upper2 and Lower2];
		\addlegendentry{ $\mathcal{R}_{7}$ }	
		
				    
		
		
		\foreach \Nxi in {180}{		
			\addplot[ultra thick, Green, mark = star, mark size = 5 pt, only marks] table[skip first n=1, 
			x expr=\thisrowno{4},
			y expr=\thisrowno{6} * 0.4674,
			restrict expr to domain={\thisrowno{2}}{27:47},
			restrict expr to domain={\thisrowno{3}}{199:230}
			]{results/CIEMAT-QI/0.250/DKE_zeta_Convergence_Example_Nxi_\Nxi/Monoenergetic_nu_0.100E-04_E_rho_0.000E+00_sorted.plt};
		}	
		\addlegendentry{Selected}		
		
		
	\end{axis}
\end{tikzpicture}

    	\caption{}
    	\label{subfig:D31_convergence_Legendre_CIEMAT_QI_0250_Erho_0_Detail}
    \end{subfigure}
    %\hfill
    \begin{subfigure}[t]{0.32\textwidth}
    	\tikzsetnextfilename{Convergence-theta-zeta-CIEMAT-QI-s0250-Er-0-D31}
    	\begin{tikzpicture}
	\begin{axis}[
		%		height=0.85\textwidth, 
		width=\textwidth, 
		scaled y ticks=base 10:2,
		xtick={40, 80,119,160,200, 240},
		y tick label style={
			/pgf/number format/.cd,
			fixed,
			fixed zerofill,
			precision=1,
			/tikz/.cd}, 
		ymin = 0.040 * 0.4674,
		ymax = 0.102 * 0.4674,
		legend pos = south east, 
		legend columns =2, 
		xlabel = $N_\zeta$, ylabel=$\widehat{D}_{31}$ ${[\text{m}]}$
		]		
		
		\addplot[forget plot, name path=Upper2,red!20, 
		domain = 73:243] {0.888E-01* 0.4674+ 5e-3  };	
		\addplot[forget plot, name path=Lower2,red!20, 
		domain = 73:243] {0.888E-01* 0.4674- 5e-3  };			
		\addplot[forget plot,red!20] fill between[of=Upper2 and Lower2];
%		\addlegendentry{ $\pm 0.01$ }
		
		\addplot[forget plot, name path=Upper2,blue!20, 
		domain = 73:243] {0.888E-01*1.07 * 0.4674};	
		\addplot[forget plot, name path=Lower2,blue!20, 
		domain = 73:243] {0.888E-01*0.93 * 0.4674};	
		\addplot[forget plot, blue!20] fill between[of=Upper2 and Lower2];
		
		\foreach \Ntheta in {15,19,...,27}{		
			\addplot+[no markers] table[skip first n=1, 
			x expr=\thisrowno{3},
			y expr=\thisrowno{6} * 0.4674,
			restrict expr to domain={\thisrowno{2}}{\Ntheta:\Ntheta},
			restrict expr to domain={\thisrowno{3}}{75:270}
			]{results/CIEMAT-QI/0.250/DKE_zeta_Convergence_Example_Nxi_180/Monoenergetic_nu_0.100E-04_E_rho_0.000E+00.plt};
%			]{results/CIEMAT-QI/0.250/DKE_zeta_Convergence_Example_Nxi_160/Gamma_11_Gamma_31_nu_0.100E-04_E_rho_0.000E+00.plt};
			\addlegendimage{empty legend}
			\addlegendentry{$N_\theta=$}
			\expandafter\addlegendentry\expandafter{\Ntheta}
		}	
	
	    
		
		\foreach \Ntheta in {15}{		
			\addplot+[ultra thick, Green, mark = star, mark size = 5 pt, only marks] table[skip first n=1, 
			x expr=\thisrowno{3},
			y expr=\thisrowno{6} * 0.4674,
			restrict expr to domain={\thisrowno{2}}{\Ntheta:\Ntheta},
			restrict expr to domain={\thisrowno{3}}{119:119}
			]{results/CIEMAT-QI/0.250/DKE_zeta_Convergence_Example_Nxi_180/Monoenergetic_nu_0.100E-04_E_rho_0.000E+00.plt};
			
		}	
		%		\addlegendentry{Spread of 5\%}
		
		
	\end{axis}
\end{tikzpicture}
	\caption{}\label{subfig:D31_convergence_theta_zeta_CIEMAT_QI_0250_Erho_0}
    \end{subfigure}



	\caption{Convergence of monoenergetic coefficients with the number of Legendre modes $N_\xi$ for CIEMAT-QI at the surface labelled by $\psi/\psi_{\text{lcfs}}=0.25$, for $\hat{\nu}(v)=10^{-5}$ $\text{m}^{-1}$ and $\hat{E}_r(v)=0$ $\text{kV}\cdot\text{s}/\text{m}^2$.}
	\label{fig:Convergence_CIEMAT_QI_Er_0}
\end{figure*}
\begin{figure}
	\centering
	\begin{subfigure}[t]{0.33\textwidth}
		\tikzsetnextfilename{Convergence-Legendre-CIEMAT-QI-s0250-Er-0-D31-Detail}
		\begin{tikzpicture}
	\begin{axis}[
		%		height=0.85\textwidth, 
		width=\textwidth, 
		scaled y ticks=base 10:2,
		xtick={80,180,280,...,480},
		y tick label style={
			/pgf/number format/.cd,
			fixed,
			fixed zerofill,
			precision=1,
			/tikz/.cd}, 
		ymax = 0.0999,
		xlabel = $N_\xi$, ylabel=$\widehat{D}_{31}$ ${[\text{m}]}$,
		legend columns=1, legend pos = north east
		]
		\addplot[forget plot, blue, mark=+, only marks] table[skip first n=1, 
		x expr=\thisrowno{4},
		y expr=\thisrowno{6}  * 0.4674,
		restrict expr to domain={\thisrowno{4}}{80:100},
		restrict expr to domain={\thisrowno{2}}{19:47},
		restrict expr to domain={\thisrowno{3}}{91:197}
		]{results/CIEMAT-QI/0.250/DKE_zeta_Convergence_Example_Nxi_20/Gamma_11_Gamma_31_nu_0.100E-04_E_rho_0.000E+00.plt};
		
		
		\addplot[forget plot, blue, mark=o, only marks] table[skip first n=1, 
		x expr=\thisrowno{4},
		y expr=-\thisrowno{7}  * 0.4674,
		restrict expr to domain={\thisrowno{4}}{80:100},
		restrict expr to domain={\thisrowno{2}}{19:47},
		restrict expr to domain={\thisrowno{3}}{91:197}
		]{results/CIEMAT-QI/0.250/DKE_zeta_Convergence_Example_Nxi_20/Gamma_11_Gamma_31_nu_0.100E-04_E_rho_0.000E+00.plt};
		
		
		\foreach \Nxi in {120}{		
			\addplot[forget plot, blue, mark=+, only marks] table[skip first n=1, 
			x expr=\thisrowno{4},
			y expr=\thisrowno{6}  * 0.4674,
			restrict expr to domain={\thisrowno{2}}{27:47},
			restrict expr to domain={\thisrowno{3}}{149:230}
			]{results/CIEMAT-QI/0.250/DKE_zeta_Convergence_Example_Nxi_\Nxi/Gamma_11_Gamma_31_nu_0.100E-04_E_rho_0.000E+00.plt};	
			
			\addplot[forget plot, blue, mark=o, only marks] table[skip first n=1, 
			x expr=\thisrowno{4},
			y expr=-\thisrowno{7}  * 0.4674,
			restrict expr to domain={\thisrowno{2}}{27:47},
			restrict expr to domain={\thisrowno{3}}{149:230}
			]{results/CIEMAT-QI/0.250/DKE_zeta_Convergence_Example_Nxi_\Nxi/Gamma_11_Gamma_31_nu_0.100E-04_E_rho_0.000E+00.plt};
		}	
		
		\foreach \Nxi in {140, 160}{		
			\addplot[forget plot, blue, mark=+, only marks] table[skip first n=1, 
			x expr=\thisrowno{4},
			y expr=\thisrowno{6}  * 0.4674,
			restrict expr to domain={\thisrowno{2}}{27:47},
			restrict expr to domain={\thisrowno{3}}{169:230}
			]{results/CIEMAT-QI/0.250/DKE_zeta_Convergence_Example_Nxi_\Nxi/Gamma_11_Gamma_31_nu_0.100E-04_E_rho_0.000E+00.plt};
			
					
			\addplot[forget plot, blue, mark=o, only marks] table[skip first n=1, 
			x expr=\thisrowno{4},
			y expr=-\thisrowno{7}  * 0.4674,
			restrict expr to domain={\thisrowno{2}}{27:47},
			restrict expr to domain={\thisrowno{3}}{169:230}
			]{results/CIEMAT-QI/0.250/DKE_zeta_Convergence_Example_Nxi_\Nxi/Gamma_11_Gamma_31_nu_0.100E-04_E_rho_0.000E+00.plt};
			
		}	
		
		\foreach \Nxi in {200}{		
			\addplot[forget plot, blue, mark=+, only marks] table[skip first n=1, 
			x expr=\thisrowno{4},
			y expr=\thisrowno{6}  * 0.4674,
			restrict expr to domain={\thisrowno{2}}{27:47},
			restrict expr to domain={\thisrowno{3}}{189:290}
			]{results/CIEMAT-QI/0.250/DKE_zeta_Convergence_Example_Nxi_\Nxi/Monoenergetic_nu_0.100E-04_E_rho_0.000E+00.plt};	
			
			
			\addplot[forget plot, blue, mark=o, only marks] table[skip first n=1, 
			x expr=\thisrowno{4},
			y expr=-\thisrowno{7}  * 0.4674,
			restrict expr to domain={\thisrowno{2}}{27:47},
			restrict expr to domain={\thisrowno{3}}{189:290}
			]{results/CIEMAT-QI/0.250/DKE_zeta_Convergence_Example_Nxi_\Nxi/Monoenergetic_nu_0.100E-04_E_rho_0.000E+00.plt};
		}	
		    
	    \addplot[blue, mark=+, only marks] table[skip first n=1, 
	    x expr=\thisrowno{4},
	    y expr=\thisrowno{6} * 0.4674,
	    restrict expr to domain={\thisrowno{1}}{0:0},
	    restrict expr to domain={\thisrowno{4}}{220:400}
	    ]{results/CIEMAT-QI/0.250/Convergence_nu_1e-5/Convergence_Nxi/N_theta_47_N_zeta_215/monkes_Monoenergetic_Database.dat};
	    \addlegendentry{$\widehat{D}_{31}$}
	    
	    \addplot[blue, mark=o, only marks] table[skip first n=1, 
	    x expr=\thisrowno{4},
	    y expr=-\thisrowno{7} * 0.4674,
	    restrict expr to domain={\thisrowno{1}}{0:0},
	    restrict expr to domain={\thisrowno{4}}{220:400}
	    ]{results/CIEMAT-QI/0.250/Convergence_nu_1e-5/Convergence_Nxi/N_theta_47_N_zeta_215/monkes_Monoenergetic_Database.dat};
	    \addlegendentry{$-\widehat{D}_{13}$}
	    
	    
	    
	    \addplot[forget plot, name path=Upper2,red!20, 
	    domain = 100:400] {(0.888E-01)  * 0.4674+ 5e-3};	
	    \addplot[forget plot, name path=Lower2,red!20, 
	    domain = 100:400] {(0.888E-01)  * 0.4674- 5e-3};			
	    \addplot[red!20] fill between[of=Upper2 and Lower2];
	    \addlegendentry{ $\mathcal{A}_{0.005}$ }	
	    
		\addplot[forget plot, name path=Upper2,blue!20, 
		domain = 100:400] {0.888E-01*1.07 * 0.4674};	
		\addplot[forget plot, name path=Lower2,blue!20, 
		domain = 100:400] {0.888E-01*0.93 * 0.4674};		
		\addplot[blue!20] fill between[of=Upper2 and Lower2];
		\addlegendentry{ $\mathcal{R}_{7}$ }	
		
				    
		
		
		\foreach \Nxi in {180}{		
			\addplot[ultra thick, Green, mark = star, mark size = 5 pt, only marks] table[skip first n=1, 
			x expr=\thisrowno{4},
			y expr=\thisrowno{6} * 0.4674,
			restrict expr to domain={\thisrowno{2}}{27:47},
			restrict expr to domain={\thisrowno{3}}{199:230}
			]{results/CIEMAT-QI/0.250/DKE_zeta_Convergence_Example_Nxi_\Nxi/Monoenergetic_nu_0.100E-04_E_rho_0.000E+00_sorted.plt};
		}	
		\addlegendentry{Selected}		
		
		
	\end{axis}
\end{tikzpicture}

		\caption{}
		\label{subfig:D31_convergence_Legendre_CIEMAT_QI_0250_Erho_0_Detail}
	\end{subfigure}
	%\hfill
	\begin{subfigure}[t]{0.33\textwidth}
		\tikzsetnextfilename{Convergence-theta-zeta-CIEMAT-QI-s0250-Er-0-D31}
		\begin{tikzpicture}
	\begin{axis}[
		%		height=0.85\textwidth, 
		width=\textwidth, 
		scaled y ticks=base 10:2,
		xtick={40, 80,119,160,200, 240},
		y tick label style={
			/pgf/number format/.cd,
			fixed,
			fixed zerofill,
			precision=1,
			/tikz/.cd}, 
		ymin = 0.040 * 0.4674,
		ymax = 0.102 * 0.4674,
		legend pos = south east, 
		legend columns =2, 
		xlabel = $N_\zeta$, ylabel=$\widehat{D}_{31}$ ${[\text{m}]}$
		]		
		
		\addplot[forget plot, name path=Upper2,red!20, 
		domain = 73:243] {0.888E-01* 0.4674+ 5e-3  };	
		\addplot[forget plot, name path=Lower2,red!20, 
		domain = 73:243] {0.888E-01* 0.4674- 5e-3  };			
		\addplot[forget plot,red!20] fill between[of=Upper2 and Lower2];
%		\addlegendentry{ $\pm 0.01$ }
		
		\addplot[forget plot, name path=Upper2,blue!20, 
		domain = 73:243] {0.888E-01*1.07 * 0.4674};	
		\addplot[forget plot, name path=Lower2,blue!20, 
		domain = 73:243] {0.888E-01*0.93 * 0.4674};	
		\addplot[forget plot, blue!20] fill between[of=Upper2 and Lower2];
		
		\foreach \Ntheta in {15,19,...,27}{		
			\addplot+[no markers] table[skip first n=1, 
			x expr=\thisrowno{3},
			y expr=\thisrowno{6} * 0.4674,
			restrict expr to domain={\thisrowno{2}}{\Ntheta:\Ntheta},
			restrict expr to domain={\thisrowno{3}}{75:270}
			]{results/CIEMAT-QI/0.250/DKE_zeta_Convergence_Example_Nxi_180/Monoenergetic_nu_0.100E-04_E_rho_0.000E+00.plt};
%			]{results/CIEMAT-QI/0.250/DKE_zeta_Convergence_Example_Nxi_160/Gamma_11_Gamma_31_nu_0.100E-04_E_rho_0.000E+00.plt};
			\addlegendimage{empty legend}
			\addlegendentry{$N_\theta=$}
			\expandafter\addlegendentry\expandafter{\Ntheta}
		}	
	
	    
		
		\foreach \Ntheta in {15}{		
			\addplot+[ultra thick, Green, mark = star, mark size = 5 pt, only marks] table[skip first n=1, 
			x expr=\thisrowno{3},
			y expr=\thisrowno{6} * 0.4674,
			restrict expr to domain={\thisrowno{2}}{\Ntheta:\Ntheta},
			restrict expr to domain={\thisrowno{3}}{119:119}
			]{results/CIEMAT-QI/0.250/DKE_zeta_Convergence_Example_Nxi_180/Monoenergetic_nu_0.100E-04_E_rho_0.000E+00.plt};
			
		}	
		%		\addlegendentry{Spread of 5\%}
		
		
	\end{axis}
\end{tikzpicture}
	\caption{}\label{subfig:D31_convergence_theta_zeta_CIEMAT_QI_0250_Erho_0}
	\end{subfigure}
	%	%\hfill
	\begin{subfigure}[t]{0.33\textwidth}
		\tikzsetnextfilename{Clock-time-CIEMAT-QI-s0250-Er-0-D31}
		\begin{tikzpicture}
	\begin{axis}[
		%		height=0.85\textwidth, 
		width=0.97\textwidth, 
				extra x ticks={119},
		extra y ticks={78},
		extra tick style={grid=major, grid style={dashed,black}}, 
		%		scaled y ticks=base 10:-2,
		xtick={80,160,200, 240},
		ytick={0,150, 225,300},
		y tick label style={
			/pgf/number format/.cd,
			fixed,
			fixed zerofill,
			precision=0,
			/tikz/.cd}, 
		%		ymax = 0.0205,
		legend pos = south east, 
		legend columns =2, 
		%		xmax = 320, 
		%		ymin = 0.0165,
		xlabel = $N_\zeta$, ylabel=Wall-clock time {[s]}
		]
		
		\foreach \Ntheta in {15}{		
			\addplot+[no markers] table[skip first n=1, 
			x expr=\thisrowno{3},
			y expr=\thisrowno{10},
			restrict expr to domain={\thisrowno{2}}{\Ntheta:\Ntheta},
			restrict expr to domain={\thisrowno{3}}{60:200}
			]{results/CIEMAT-QI/0.250/DKE_zeta_Convergence_Example_Nxi_180/Monoenergetic_nu_0.100E-04_E_rho_0.000E+00.plt};
			\addlegendimage{empty legend}
			\addlegendentry{$N_\theta=$}
			\expandafter\addlegendentry\expandafter{\Ntheta}
		}	
		
		\foreach \Ntheta in {15}{		
			\addplot+[ultra thick, Green, mark = star, mark size = 5 pt, only marks] table[skip first n=1, 
			x expr=\thisrowno{3},
			y expr=\thisrowno{10},
			restrict expr to domain={\thisrowno{2}}{\Ntheta:\Ntheta},
			restrict expr to domain={\thisrowno{3}}{119:119}
			]{results/CIEMAT-QI/0.250/DKE_zeta_Convergence_Example_Nxi_180/Monoenergetic_nu_0.100E-04_E_rho_0.000E+00.plt};
			
		}	
		
		
	\end{axis}
\end{tikzpicture}

		\caption{}
		\label{subfig:D31_Clock_time_CIEMAT_QI_0250_Erho_0}
	\end{subfigure}
	\caption{Selection of the resolution to have a sufficiently accurate calculation of the parallel flow geometric coefficient $\widehat{D}_{31}$ for CIEMAT-QI at the surface labelled by $\psi/\psi_{\text{lcfs}}=0.25$, for $\hat{\nu}(v)=10^{-5}$ $\text{m}^{-1}$ and $\hat{E}_r(v)=0$ $\text{kV}\cdot\text{s}/\text{m}^2$.}
	\label{fig:Convergence_CIEMAT_QI_Er_0_Detail}
\end{figure}

The monoenergetic coefficients of the flat mirror configuration CIEMAT-QI are more difficult to converge due to their smaller absolute value. 
\begin{figure*}[t]
	\centering
	\begin{subfigure}[t]{0.32\textwidth}
		\tikzsetnextfilename{Convergence-Legendre-CIEMAT-QI-s0250-Er-1e-3-D11}
		\begin{tikzpicture}
	\begin{axis}[
		%		height=0.85\textwidth, 
		width=\textwidth, 
		scaled y ticks=base 10:4,
		y tick label style={
			/pgf/number format/.cd,
			fixed,
			fixed zerofill,
			precision=1,
			/tikz/.cd}, 
		xlabel = $N_\xi$, ylabel=$\widehat{D}_{11} $ ${[\text{m}]}$
		]
		
		\addplot[blue, mark=+, only marks, forget plot] table[skip first n=1, 
		x expr=\thisrowno{4},
		y expr=\thisrowno{5}*0.4674*0.4674,
		restrict expr to domain={\thisrowno{2}}{15:47},
		restrict expr to domain={\thisrowno{3}}{75:97}
		]{data/CIEMAT-QI/MONKES/DKE_zeta_Convergence_Example_Nxi_20/Gamma_11_Gamma_31_nu_0.100E-04_E_rho_0.100E-02.plt};
				
		\foreach \Nxi in {120,140,160,180,200}
		{		
			\addplot[forget plot, blue, mark=+, only marks] table[skip first n=1, 
			x expr=\thisrowno{4},
			y expr=\thisrowno{5}*0.4674*0.4674,
			restrict expr to domain={\thisrowno{2}}{15:47},
			restrict expr to domain={\thisrowno{3}}{95:179}
			]{data/CIEMAT-QI/MONKES/DKE_zeta_Convergence_Example_Nxi_\Nxi/Monoenergetic_nu_0.100E-04_E_rho_0.100E-02.plt};
		}		
	    
	    \addplot[forget plot, blue, mark=+, only marks] table[skip first n=1, 
	    x expr=\thisrowno{4},
	    y expr=\thisrowno{5}* 0.4674 * 0.4674,
	    restrict expr to domain={\thisrowno{1}}{1e-3:1e-3},
	    restrict expr to domain={\thisrowno{4}}{200:380}
	    ]{data/CIEMAT-QI/MONKES/Convergence_Nxi/N_theta_47_N_zeta_215/monkes_Monoenergetic_Database.dat};
	    	    
	\end{axis}
\end{tikzpicture}
%
		\caption{}
		\label{subfig:D11_convergence_Legendre_CIEMAT_QI_0250_Erho_1e-3}
	\end{subfigure}
%	%\hfill
	\begin{subfigure}[t]{0.32\textwidth}
		\tikzsetnextfilename{Convergence-Legendre-CIEMAT-QI-s0250-Er-1e-3-D33}
		\begin{tikzpicture}
	\begin{axis}[
		%		height=0.85\textwidth, 
		width=\textwidth, 
		scaled y ticks=base 10:-4,
		y tick label style={
			/pgf/number format/.cd,
			fixed,
			fixed zerofill,
			precision=1,
			/tikz/.cd}, 
		xlabel = $N_\xi$, ylabel=$\widehat{D}_{33}$ ${[\text{m}]}$
		]
		
		\addplot[blue, mark=+, only marks] table[skip first n=1, 
		x expr=\thisrowno{4},
		y expr=\thisrowno{8},
		restrict expr to domain={\thisrowno{2}}{15:47},
		restrict expr to domain={\thisrowno{3}}{85:97}
		]{results/CIEMAT-QI/0.250/DKE_zeta_Convergence_Example_Nxi_20/Gamma_11_Gamma_31_nu_0.100E-04_E_rho_0.100E-02.plt};
				
		\foreach \Nxi in {120,140,160,180,200}
		{		
			\addplot[blue, mark=+, only marks] table[skip first n=1, 
			x expr=\thisrowno{4},
			y expr=abs(\thisrowno{8}),
			restrict expr to domain={\thisrowno{2}}{15:47},
			restrict expr to domain={\thisrowno{3}}{135:179}
			]{results/CIEMAT-QI/0.250/DKE_zeta_Convergence_Example_Nxi_\Nxi/Monoenergetic_nu_0.100E-04_E_rho_0.100E-02.plt};
		}
		   
	   \addplot[forget plot, blue, mark=+, only marks] table[skip first n=1, 
	   x expr=\thisrowno{4},
	   y expr=-\thisrowno{8},
	   restrict expr to domain={\thisrowno{1}}{1e-3:1e-3},
	   restrict expr to domain={\thisrowno{4}}{200:380}
	   ]{results/CIEMAT-QI/0.250/Convergence_nu_1e-5/Convergence_Nxi/N_theta_47_N_zeta_215/monkes_Monoenergetic_Database.dat};
	   	
	\end{axis}
\end{tikzpicture}
%
		\caption{}
		\label{subfig:D33_convergence_Legendre_CIEMAT_QI_0250_Erho_1e-3}
	\end{subfigure}
    
    \begin{subfigure}[t]{0.32\textwidth}
    	\tikzsetnextfilename{Convergence-Legendre-CIEMAT-QI-s0250-Er-1e-3-D31-Detail}
    	\begin{tikzpicture}
	\begin{axis}[
		%		height=0.85\textwidth, 
		width=\textwidth, 
		scaled y ticks=base 10:2,
		xtick={100,180, 260, 340},
				y tick label style={
						/pgf/number format/.cd,
						fixed,
						fixed zerofill,
						precision=1,
						/tikz/.cd}, 
		ymax = 0.018,
        legend columns=1, legend pos=north east, 
		ymin = 0.0156 * 0.4674,
		xlabel = $N_\xi$, ylabel=$\widehat{D}_{31}$ ${[\text{m}]}$
		]
		
		\addplot[forget plot, blue, mark=+, only marks] table[skip first n=1, 
		x expr=\thisrowno{4},
		y expr=\thisrowno{6} * 0.4674,
		restrict expr to domain={\thisrowno{4}}{80:100},
		restrict expr to domain={\thisrowno{2}}{19:47},
		restrict expr to domain={\thisrowno{3}}{91:197}
		]{data/CIEMAT-QI/MONKES/DKE_zeta_Convergence_Example_Nxi_20/Gamma_11_Gamma_31_nu_0.100E-04_E_rho_0.100E-02.plt};
				
		\addplot[forget plot, blue, mark=o, only marks] table[skip first n=1, 
		x expr=\thisrowno{4},
		y expr=-\thisrowno{7} * 0.4674,
		restrict expr to domain={\thisrowno{4}}{80:100},
		restrict expr to domain={\thisrowno{2}}{19:47},
		restrict expr to domain={\thisrowno{3}}{91:197}
		]{data/CIEMAT-QI/MONKES/DKE_zeta_Convergence_Example_Nxi_20/Gamma_11_Gamma_31_nu_0.100E-04_E_rho_0.100E-02.plt};
		
		
		\addplot[blue, mark=+, only marks] table[skip first n=1, 
		x expr=\thisrowno{4},
		y expr=\thisrowno{6} * 0.4674,
		restrict expr to domain={\thisrowno{1}}{1e-3:1e-3},
		restrict expr to domain={\thisrowno{4}}{200:380}
		]{data/CIEMAT-QI/MONKES/Convergence_Nxi/N_theta_47_N_zeta_215/monkes_Monoenergetic_Database.dat};
		\addlegendentry{$\widehat{D}_{31}$}
		
		\addplot[blue, mark=o, only marks] table[skip first n=1, 
		x expr=\thisrowno{4},
		y expr=-\thisrowno{7} * 0.4674,
		restrict expr to domain={\thisrowno{1}}{1e-3:1e-3},
		restrict expr to domain={\thisrowno{4}}{200:380}
		]{data/CIEMAT-QI/MONKES/Convergence_Nxi/N_theta_47_N_zeta_215/monkes_Monoenergetic_Database.dat};
		\addlegendentry{$-\widehat{D}_{13}$}
		
		
		\foreach \Nxi in {120}{		
			\addplot[forget plot, blue, mark=+, only marks] table[skip first n=1, 
			x expr=\thisrowno{4},
			y expr=\thisrowno{6} * 0.4674,
			restrict expr to domain={\thisrowno{2}}{39:47},
			restrict expr to domain={\thisrowno{3}}{149:230}
			]{data/CIEMAT-QI/MONKES/DKE_zeta_Convergence_Example_Nxi_\Nxi/Monoenergetic_nu_0.100E-04_E_rho_0.100E-02.plt};		
			
			\addplot[forget plot, blue, mark=o, only marks] table[skip first n=1, 
			x expr=\thisrowno{4},
			y expr=-\thisrowno{7} * 0.4674,
			restrict expr to domain={\thisrowno{2}}{39:47},
			restrict expr to domain={\thisrowno{3}}{149:230}
			]{data/CIEMAT-QI/MONKES/DKE_zeta_Convergence_Example_Nxi_\Nxi/Monoenergetic_nu_0.100E-04_E_rho_0.100E-02.plt};
		}	
	    
	    \foreach \Nxi in {140, 160}{		
	    	\addplot[forget plot, blue, mark=+, only marks] table[skip first n=1, 
	    	x expr=\thisrowno{4},
	    	y expr=\thisrowno{6} * 0.4674,
	    	restrict expr to domain={\thisrowno{2}}{39:47},
	    	restrict expr to domain={\thisrowno{3}}{189:230}
	    	]{data/CIEMAT-QI/MONKES/DKE_zeta_Convergence_Example_Nxi_\Nxi/Monoenergetic_nu_0.100E-04_E_rho_0.100E-02.plt};
	    	
	    			
	    	\addplot[forget plot, blue, mark=o, only marks] table[skip first n=1, 
	    	x expr=\thisrowno{4},
	    	y expr=-\thisrowno{7} * 0.4674,
	    	restrict expr to domain={\thisrowno{2}}{39:47},
	    	restrict expr to domain={\thisrowno{3}}{189:230}
	    	]{data/CIEMAT-QI/MONKES/DKE_zeta_Convergence_Example_Nxi_\Nxi/Monoenergetic_nu_0.100E-04_E_rho_0.100E-02.plt};
	    }	
    			    
		\addplot[forget plot, name path=Upper2,red!20, 
		domain = 80:380] {(0.189E-01* 0.4674) +1e-3};	
		\addplot[forget plot, name path=Lower2,red!20, 
		domain = 80:380] {(0.189E-01* 0.4674) -1e-3};		
		\addplot[red!20] fill between[of=Upper2 and Lower2];
		\addlegendentry{$\mathcal{A}_{0.001}$}	
		
		\addplot[forget plot, name path=Upper2,blue!20, 
		domain = 80:380] {0.189E-01*1.07 * 0.4674};	
		\addplot[forget plot, name path=Lower2,blue!20, 
		domain = 80:380] {0.189E-01*0.93 * 0.4674};		
		\addplot[blue!20] fill between[of=Upper2 and Lower2];
		\addlegendentry{$\mathcal{R}_{7}$}			    
		 
		\foreach \Nxi in {180}{		
			\addplot+[only marks, ultra thick, Green, mark = star, mark size = 5 pt ] table[skip first n=1, 
			x expr=\thisrowno{4},
			y expr=\thisrowno{6} * 0.4674,
			restrict expr to domain={\thisrowno{2}}{39:47},
			restrict expr to domain={\thisrowno{3}}{211:230}
			]{data/CIEMAT-QI/MONKES/DKE_zeta_Convergence_Example_Nxi_\Nxi/Monoenergetic_nu_0.100E-04_E_rho_0.100E-02_sorted.plt};
		}	
	    \addlegendentry{Selected}		
		
	\end{axis}
\end{tikzpicture}
%
    	\caption{}
    	\label{subfig:D31_convergence_Legendre_CIEMAT_QI_0250_Erho_1e-3_Detail}
    \end{subfigure}
    %\hfill
    \begin{subfigure}[t]{0.32\textwidth}
    	\tikzsetnextfilename{Convergence-theta-zeta-CIEMAT-QI-s0250-Er-1e-3-D31}
    	\begin{tikzpicture}
	\begin{axis}[
		%		height=0.85\textwidth, 
		width=\textwidth, 
		scaled y ticks=base 10:2,
		xtick={40,80,119,160,200, 240},
				y tick label style={
						/pgf/number format/.cd,
						fixed,
						fixed zerofill,
						precision=1,
						/tikz/.cd}, 
		legend pos = south east, 
		legend columns =2, 
%		xmax = 320, 
		ymax = 0.022 * 0.4674,
		ymin = 0.008 * 0.4674,
		xlabel = $N_\zeta$, ylabel=$\widehat{D}_{31} $ ${[\text{m}]}$
		]
				
		\addplot[forget plot, name path=Upper2,red!20, 
		domain = 50:227] {(0.189E-01* 0.4674) +1e-3};	
		\addplot[forget plot, name path=Lower2,red!20, 
		domain = 50:227] {(0.189E-01* 0.4674) -1e-3};		
		\addplot[forget plot, red!20] fill between[of=Upper2 and Lower2];		
		
		\addplot[forget plot, name path=Upper2,blue!20, 
		domain = 50:227] {0.189E-01*1.07 * 0.4674};	
		\addplot[forget plot, name path=Lower2,blue!20, 
		domain = 50:227] {0.189E-01 *0.93 * 0.4674};		
		\addplot[forget plot, blue!20] fill between[of=Upper2 and Lower2];
		
		\foreach \Ntheta in {15,19,...,27}
		{		
			\addplot+[no markers] table[skip first n=1, 
			x expr=\thisrowno{3},
			y expr=\thisrowno{6} * 0.4674,
			restrict expr to domain={\thisrowno{2}}{\Ntheta:\Ntheta},
			restrict expr to domain={\thisrowno{3}}{50:270}
			]{data/CIEMAT-QI/MONKES/DKE_zeta_Convergence_Example_Nxi_180/Monoenergetic_nu_0.100E-04_E_rho_0.100E-02_sorted.plt};
			\addlegendimage{empty legend}
			\addlegendentry{$N_\theta=$}
			\expandafter\addlegendentry\expandafter{\Ntheta}
		}	
		
		\foreach \Ntheta in {15}
		{		
			\addplot+[only marks, ultra thick, Green, mark = star, mark size = 5 pt ] table[skip first n=1, 
			x expr=\thisrowno{3},
			y expr=\thisrowno{6} * 0.4674,
			restrict expr to domain={\thisrowno{2}}{\Ntheta:\Ntheta},
			restrict expr to domain={\thisrowno{3}}{119:119}
			]{data/CIEMAT-QI/MONKES/DKE_zeta_Convergence_Example_Nxi_180/Monoenergetic_nu_0.100E-04_E_rho_0.100E-02.plt};
			
		}	 
        
	\end{axis}
\end{tikzpicture}%
    	\caption{}
    	\label{subfig:D31_convergence_theta_zeta_CIEMAT_QI_0250_Erho_1e-3}
    \end{subfigure}
   	\caption{Convergence of monoenergetic coefficients with the number of Legendre modes $N_\xi$ for CIEMAT-QI at the surface labelled by $\psi/\psi_{\text{lcfs}}=0.25$, for $\hat{\nu}(v)=10^{-5}$ $\text{m}^{-1}$ and $\hat{E}_r(v)=10^{-3}$ $\text{kV}\cdot\text{s}/\text{m}^2$.}
	\label{fig:Convergence_CIEMAT_QI_Er_1e-3}
\end{figure*}
\begin{figure}
	\centering
	\begin{subfigure}[t]{0.33\textwidth}
		\tikzsetnextfilename{Convergence-Legendre-CIEMAT-QI-s0250-Er-1e-3-D31-Detail}
		\begin{tikzpicture}
	\begin{axis}[
		%		height=0.85\textwidth, 
		width=\textwidth, 
		scaled y ticks=base 10:2,
		xtick={100,180, 260, 340},
				y tick label style={
						/pgf/number format/.cd,
						fixed,
						fixed zerofill,
						precision=1,
						/tikz/.cd}, 
		ymax = 0.018,
        legend columns=1, legend pos=north east, 
		ymin = 0.0156 * 0.4674,
		xlabel = $N_\xi$, ylabel=$\widehat{D}_{31}$ ${[\text{m}]}$
		]
		
		\addplot[forget plot, blue, mark=+, only marks] table[skip first n=1, 
		x expr=\thisrowno{4},
		y expr=\thisrowno{6} * 0.4674,
		restrict expr to domain={\thisrowno{4}}{80:100},
		restrict expr to domain={\thisrowno{2}}{19:47},
		restrict expr to domain={\thisrowno{3}}{91:197}
		]{data/CIEMAT-QI/MONKES/DKE_zeta_Convergence_Example_Nxi_20/Gamma_11_Gamma_31_nu_0.100E-04_E_rho_0.100E-02.plt};
				
		\addplot[forget plot, blue, mark=o, only marks] table[skip first n=1, 
		x expr=\thisrowno{4},
		y expr=-\thisrowno{7} * 0.4674,
		restrict expr to domain={\thisrowno{4}}{80:100},
		restrict expr to domain={\thisrowno{2}}{19:47},
		restrict expr to domain={\thisrowno{3}}{91:197}
		]{data/CIEMAT-QI/MONKES/DKE_zeta_Convergence_Example_Nxi_20/Gamma_11_Gamma_31_nu_0.100E-04_E_rho_0.100E-02.plt};
		
		
		\addplot[blue, mark=+, only marks] table[skip first n=1, 
		x expr=\thisrowno{4},
		y expr=\thisrowno{6} * 0.4674,
		restrict expr to domain={\thisrowno{1}}{1e-3:1e-3},
		restrict expr to domain={\thisrowno{4}}{200:380}
		]{data/CIEMAT-QI/MONKES/Convergence_Nxi/N_theta_47_N_zeta_215/monkes_Monoenergetic_Database.dat};
		\addlegendentry{$\widehat{D}_{31}$}
		
		\addplot[blue, mark=o, only marks] table[skip first n=1, 
		x expr=\thisrowno{4},
		y expr=-\thisrowno{7} * 0.4674,
		restrict expr to domain={\thisrowno{1}}{1e-3:1e-3},
		restrict expr to domain={\thisrowno{4}}{200:380}
		]{data/CIEMAT-QI/MONKES/Convergence_Nxi/N_theta_47_N_zeta_215/monkes_Monoenergetic_Database.dat};
		\addlegendentry{$-\widehat{D}_{13}$}
		
		
		\foreach \Nxi in {120}{		
			\addplot[forget plot, blue, mark=+, only marks] table[skip first n=1, 
			x expr=\thisrowno{4},
			y expr=\thisrowno{6} * 0.4674,
			restrict expr to domain={\thisrowno{2}}{39:47},
			restrict expr to domain={\thisrowno{3}}{149:230}
			]{data/CIEMAT-QI/MONKES/DKE_zeta_Convergence_Example_Nxi_\Nxi/Monoenergetic_nu_0.100E-04_E_rho_0.100E-02.plt};		
			
			\addplot[forget plot, blue, mark=o, only marks] table[skip first n=1, 
			x expr=\thisrowno{4},
			y expr=-\thisrowno{7} * 0.4674,
			restrict expr to domain={\thisrowno{2}}{39:47},
			restrict expr to domain={\thisrowno{3}}{149:230}
			]{data/CIEMAT-QI/MONKES/DKE_zeta_Convergence_Example_Nxi_\Nxi/Monoenergetic_nu_0.100E-04_E_rho_0.100E-02.plt};
		}	
	    
	    \foreach \Nxi in {140, 160}{		
	    	\addplot[forget plot, blue, mark=+, only marks] table[skip first n=1, 
	    	x expr=\thisrowno{4},
	    	y expr=\thisrowno{6} * 0.4674,
	    	restrict expr to domain={\thisrowno{2}}{39:47},
	    	restrict expr to domain={\thisrowno{3}}{189:230}
	    	]{data/CIEMAT-QI/MONKES/DKE_zeta_Convergence_Example_Nxi_\Nxi/Monoenergetic_nu_0.100E-04_E_rho_0.100E-02.plt};
	    	
	    			
	    	\addplot[forget plot, blue, mark=o, only marks] table[skip first n=1, 
	    	x expr=\thisrowno{4},
	    	y expr=-\thisrowno{7} * 0.4674,
	    	restrict expr to domain={\thisrowno{2}}{39:47},
	    	restrict expr to domain={\thisrowno{3}}{189:230}
	    	]{data/CIEMAT-QI/MONKES/DKE_zeta_Convergence_Example_Nxi_\Nxi/Monoenergetic_nu_0.100E-04_E_rho_0.100E-02.plt};
	    }	
    			    
		\addplot[forget plot, name path=Upper2,red!20, 
		domain = 80:380] {(0.189E-01* 0.4674) +1e-3};	
		\addplot[forget plot, name path=Lower2,red!20, 
		domain = 80:380] {(0.189E-01* 0.4674) -1e-3};		
		\addplot[red!20] fill between[of=Upper2 and Lower2];
		\addlegendentry{$\mathcal{A}_{0.001}$}	
		
		\addplot[forget plot, name path=Upper2,blue!20, 
		domain = 80:380] {0.189E-01*1.07 * 0.4674};	
		\addplot[forget plot, name path=Lower2,blue!20, 
		domain = 80:380] {0.189E-01*0.93 * 0.4674};		
		\addplot[blue!20] fill between[of=Upper2 and Lower2];
		\addlegendentry{$\mathcal{R}_{7}$}			    
		 
		\foreach \Nxi in {180}{		
			\addplot+[only marks, ultra thick, Green, mark = star, mark size = 5 pt ] table[skip first n=1, 
			x expr=\thisrowno{4},
			y expr=\thisrowno{6} * 0.4674,
			restrict expr to domain={\thisrowno{2}}{39:47},
			restrict expr to domain={\thisrowno{3}}{211:230}
			]{data/CIEMAT-QI/MONKES/DKE_zeta_Convergence_Example_Nxi_\Nxi/Monoenergetic_nu_0.100E-04_E_rho_0.100E-02_sorted.plt};
		}	
	    \addlegendentry{Selected}		
		
	\end{axis}
\end{tikzpicture}

		\caption{}
		\label{subfig:D31_convergence_Legendre_CIEMAT_QI_0250_Erho_1e-3_Detail}
	\end{subfigure}
	%\hfill
	\begin{subfigure}[t]{0.33\textwidth}
		\tikzsetnextfilename{Convergence-theta-zeta-CIEMAT-QI-s0250-Er-1e-3-D31}
		\begin{tikzpicture}
	\begin{axis}[
		%		height=0.85\textwidth, 
		width=\textwidth, 
		scaled y ticks=base 10:2,
		xtick={40,80,119,160,200, 240},
				y tick label style={
						/pgf/number format/.cd,
						fixed,
						fixed zerofill,
						precision=1,
						/tikz/.cd}, 
		legend pos = south east, 
		legend columns =2, 
%		xmax = 320, 
		ymax = 0.022 * 0.4674,
		ymin = 0.008 * 0.4674,
		xlabel = $N_\zeta$, ylabel=$\widehat{D}_{31} $ ${[\text{m}]}$
		]
				
		\addplot[forget plot, name path=Upper2,red!20, 
		domain = 50:227] {(0.189E-01* 0.4674) +1e-3};	
		\addplot[forget plot, name path=Lower2,red!20, 
		domain = 50:227] {(0.189E-01* 0.4674) -1e-3};		
		\addplot[forget plot, red!20] fill between[of=Upper2 and Lower2];		
		
		\addplot[forget plot, name path=Upper2,blue!20, 
		domain = 50:227] {0.189E-01*1.07 * 0.4674};	
		\addplot[forget plot, name path=Lower2,blue!20, 
		domain = 50:227] {0.189E-01 *0.93 * 0.4674};		
		\addplot[forget plot, blue!20] fill between[of=Upper2 and Lower2];
		
		\foreach \Ntheta in {15,19,...,27}
		{		
			\addplot+[no markers] table[skip first n=1, 
			x expr=\thisrowno{3},
			y expr=\thisrowno{6} * 0.4674,
			restrict expr to domain={\thisrowno{2}}{\Ntheta:\Ntheta},
			restrict expr to domain={\thisrowno{3}}{50:270}
			]{data/CIEMAT-QI/MONKES/DKE_zeta_Convergence_Example_Nxi_180/Monoenergetic_nu_0.100E-04_E_rho_0.100E-02_sorted.plt};
			\addlegendimage{empty legend}
			\addlegendentry{$N_\theta=$}
			\expandafter\addlegendentry\expandafter{\Ntheta}
		}	
		
		\foreach \Ntheta in {15}
		{		
			\addplot+[only marks, ultra thick, Green, mark = star, mark size = 5 pt ] table[skip first n=1, 
			x expr=\thisrowno{3},
			y expr=\thisrowno{6} * 0.4674,
			restrict expr to domain={\thisrowno{2}}{\Ntheta:\Ntheta},
			restrict expr to domain={\thisrowno{3}}{119:119}
			]{data/CIEMAT-QI/MONKES/DKE_zeta_Convergence_Example_Nxi_180/Monoenergetic_nu_0.100E-04_E_rho_0.100E-02.plt};
			
		}	 
        
	\end{axis}
\end{tikzpicture}%
		\caption{}
		\label{subfig:D31_convergence_theta_zeta_CIEMAT_QI_0250_Erho_1e-3}
	\end{subfigure}
	%	%\hfill
	\begin{subfigure}[t]{0.33\textwidth}
		\tikzsetnextfilename{Clock-time-CIEMAT-QI-s0250-Er-1e-3-D31}
        \begin{tikzpicture}
	\begin{axis}[
		%		height=0.85\textwidth, 
		width=0.97\textwidth, 
		extra x ticks={119},
		extra y ticks={78},
		extra tick style={grid=major, grid style={dashed,black}}, 
%		scaled y ticks=base 10:-2,
		xtick={80,160,200, 240},
		y tick label style={
			/pgf/number format/.cd,
			fixed,
			fixed zerofill,
			precision=0,
			/tikz/.cd}, 
		%		ymax = 0.0205,
		legend pos = south east, 
		legend columns =2, 
		%		xmax = 320, 
		%		ymin = 0.0165,
		xlabel = $N_\zeta$, ylabel=Wall-clock time {[s]}
		]
		
		\foreach \Ntheta in {15}{		
			\addplot+[no markers] table[skip first n=1, 
			x expr=\thisrowno{3},
			y expr=\thisrowno{10},
			restrict expr to domain={\thisrowno{2}}{\Ntheta:\Ntheta},
			restrict expr to domain={\thisrowno{3}}{60:270}
			]{results/CIEMAT-QI/0.250/DKE_zeta_Convergence_Example_Nxi_180/Monoenergetic_nu_0.100E-04_E_rho_0.100E-02_sorted.plt};
			\addlegendimage{empty legend}
			\addlegendentry{$N_\theta=$}
			\expandafter\addlegendentry\expandafter{\Ntheta}
		}	
		
		\foreach \Ntheta in {15}{		
			\addplot+[only marks, ultra thick, Green, mark = star, mark size = 5 pt  ] table[skip first n=1, 
			x expr=\thisrowno{3},
			y expr=\thisrowno{10},
			restrict expr to domain={\thisrowno{2}}{\Ntheta:\Ntheta},
			restrict expr to domain={\thisrowno{3}}{119:119}
			]{results/CIEMAT-QI/0.250/DKE_zeta_Convergence_Example_Nxi_180/Monoenergetic_nu_0.100E-04_E_rho_0.100E-02.plt};
			
		}	
		
		
	\end{axis}
\end{tikzpicture}

        \caption{}
        \label{subfig:D31_Clock_time_CIEMAT_QI_0250_Erho_1e-3}
	\end{subfigure}
	\caption{Selection of the resolution to have a sufficiently accurate calculation of the parallel flow geometric coefficient $\widehat{D}_{31}$ for CIEMAT-QI at the surface labelled by $\psi/\psi_{\text{lcfs}}=0.25$, for $\hat{\nu}(v)=10^{-5}$ $\text{m}^{-1}$ and $\hat{E}_r(v)=10^{-3}$ $\text{kV}\cdot\text{s}/\text{m}^2$.}
	\label{fig:Convergence_CIEMAT_QI_Er_1e-3_Detail}
\end{figure}
