 In this subsection we will compare {\MONKES} and {\DKES} performance in terms of the wall-clock time and describe {\MONKES} scaling properties. For the wall-clock time comparison, a convergence study (similar to the one explained in subsection \ref{subsec:Convergence}) has been carried out with {\DKES} on \ref{sec:Appendix_DKES_Bounds}. This convergence study is done to compare the wall-clock times between {\MONKES} and {\DKES} for the same level of relative convergence with respect to $\widehat{D}_{31}^{\text{r}}$. The comparison is displayed in table \ref{tab:DKES_MONKES_Comparison} along with the minimum number of Legendre modes for which {\DKES} results satisfy convergence condition (i). In all six cases, {\MONKES} is faster than {\DKES} despite using more Legendre modes. Even for W7-X EIM, in which we have taken $N_\xi =40$ for {\DKES} calculations with finite $\widehat{E}_r$, {\MONKES} is four times faster using almost four times the number of Legendre modes. For the W7-X EIM case without radial electric field, the speed-up is also of four. For the high mirror configuration, {\MONKES} is $\sim 20$ times faster than {\DKES} without $\hat{E}_r$ and $\sim 10$ times faster than {\DKES} when $\hat{E}_r \ne 0$. In the case of CIEMAT-QI, {\MONKES} is more than $\sim 13$ times faster than {\DKES} without radial electric field. In the case with finite $\hat{E}_r$, {\MONKES} calculations are around 64 times faster than {\DKES} ones. One calculation of {\MONKES} takes less than a minute and a half and the same calculation with {\DKES} requires waiting for almost an hour and a half. We point out that the wall-clock times for all the calculations shown are those from one of the partitions of CIEMAT's cluster XULA. Specifically, partition number 2 has been used, whose nodes run with Intel Xeon Gold 6254 cores at 3.10 GHz. 
 
 %
 \begin{table}[h]
 	\centering
 	\begin{tabular}{lccc}
 		\toprule
 		Case   & $N_\xi^{\DKES}$ & $t_{\text{clock}}^{\DKES}$  [s] & $t_{\text{clock}}^{\MONKES}$  {[}s{]} \\ \midrule
 		W7X-EIM $\hat{E}_r=0$                                   & 80          &  90      &         22      \\
 		W7X-EIM $\hat{E}_r\ne 0$                                & {40}          &  {172}     &         40      \\ 
 		W7X-KJM $\hat{E}_r=0$                                   & 160         &  698     &         33      \\
 		W7X-KJM $\hat{E}_r\ne 0$                                & 60          &  421     &         46      \\
 		CIEMAT-QI $\hat{E}_r=0$                                 & 160         &  1060    &         78      \\
 		CIEMAT-QI $\hat{E}_r\ne 0$                              & 160         &  4990    &         78                    \\\bottomrule
 	\end{tabular}
 	\caption{Comparison between the wall-clock time of {\DKES} and {\MONKES}.}
 	\label{tab:DKES_MONKES_Comparison}
 \end{table}
 
\begin{figure}[h]
	\centering
	\begin{subfigure}[t]{0.35\textwidth}
		\tikzsetnextfilename{MONKES-Scaling-Legendre}
		%		\hspace{0.37cm}
		\begin{tikzpicture}
	\begin{axis}[
		%		height=0.85\textwidth, 
		width=\textwidth,  
		xlabel = $N_\xi$, ylabel=Wall-clock time {[s]},
		mark size=2 pt,
		legend pos=south east
		]		
				
	    \addplot+[only marks, blue, mark=+] table[skip first n=1, 
		x expr=\thisrowno{4},
		y expr=\thisrowno{10}*1,
		restrict expr to domain={\thisrowno{4}}{0:600}
		]{results/W7X-EIM/0.200/Convergence_Nxi/N_theta_27_N_zeta_75/monkes_Monoenergetic_Database.dat};
		\addlegendentry{$N_{\text{fs}}=2025$}
		
		\addplot+[no markers, black] table[skip first n=1, 
		x expr=\thisrowno{4},
		y expr=\thisrowno{4}*0.61,
		restrict expr to domain={\thisrowno{4}}{0:600}
		]{results/W7X-EIM/0.200/Convergence_Nxi/N_theta_27_N_zeta_75/monkes_Monoenergetic_Database.dat};
		\addlegendentry{$0.61 N_{\xi}$}
		
		
	\end{axis}
\end{tikzpicture}
		\caption{}
		\label{subfig:MONKES_Scaling_Legendre}
	\end{subfigure}    
	\begin{subfigure}[t]{0.35\textwidth}
		\tikzsetnextfilename{MONKES-Scaling-Flux-Surface}
		\begin{tikzpicture}
	\begin{axis}[
		enlarge x limits=false, 
		scaled x ticks=base 10:-3,
		width=\textwidth, 
		xtick={0, 1000,...,5000}, 
		ytick={0, 1,...,10}, 
		ytick={0.0, 0.5, ..., 3.5}, 
		xlabel = $N_{\text{fs}}$, 
		ylabel={$ \frac{\text{Wall-clock time}}{\text{No. Legendre modes}}$ {[s]}},
%		ylabel=Wall-clock time$/(N_\xi+1)$ {[s]},        
        y tick label style={
        	/pgf/number format/.cd,
        	fixed,
        	fixed zerofill,
        	precision=1,
        	/tikz/.cd}, 
		mark size=2.5 pt,
		legend pos = north west, legend columns=1,
		xmin = 0, xmax=4800,
		ymax = 3.15, ymin=0, 
%		xmode=log, ymode=log
		]		
				
		\addplot+[only marks, mark repeat=2, mark options={thick,fill=white, fill opacity=0}] table[skip first n=1, 
		x expr=\thisrowno{2}*\thisrowno{3},
		y expr=(\thisrowno{10}/(1+\thisrowno{4})),
		restrict expr to domain={\thisrowno{10}/(1+\thisrowno{4})}{0:3.7}
		]{results/CIEMAT-QI/0.250/DKE_zeta_Convergence_Example_Nxi_180/Monoenergetic_nu_0.100E-04_E_rho_0.000E+00.plt};
				
	   \foreach \Ncores in {2,4,8,16}
	   {				
		    \addplot+[only marks, mark options={thick,fill=white, fill opacity=0}, mark repeat = 2] table[skip first n=1, 
	    	x expr=\thisrowno{2}*\thisrowno{3},
	    	y expr=(\thisrowno{10}/(1+\thisrowno{4})),
	    	restrict expr to domain={\thisrowno{10}/(1+\thisrowno{4})}{0:3.7}
	    	]{results/MONKES_Scaling/\Ncores_cores/Convergence_nu_1e-5_Er_0/Nxi_100/monkes_Monoenergetic_Database.dat};
	    	
	   }
	   \legend{1 core,2 cores,4 cores,8 cores,16 cores, $C_{\text{alg}}N_{\text{fs}}^3$}
	   
	   
	   \addplot[no markers, black, very thick] table[skip first n=1, 
	   x expr=\thisrowno{2}*\thisrowno{3},
	   y expr=0.61*(\thisrowno{2}*\thisrowno{3} /2025)^3,
	   restrict expr to domain={\thisrowno{10}/(1+\thisrowno{4})}{0:3.7}
	   ]{results/CIEMAT-QI/0.250/DKE_zeta_Convergence_Example_Nxi_180/Monoenergetic_nu_0.100E-04_E_rho_0.000E+00.plt};
	\end{axis}
\end{tikzpicture}
		\caption{}
		\label{subfig:MONKES_Scaling_Nfs}
	\end{subfigure}
	\caption{Scaling of {\MONKES} wall-clock time. (a) Linear scaling with the number of Legendre modes for $N_{\text{fs}}=27\times 75 = 2025$ discretization points. (b) Cubic scaling with $N_{\text{fs}}$ for different number of cores used.}
	\label{fig:MONKES_Scaling}
\end{figure}

We next check that the arithmetic complexity of the algorithm described in section \ref{sec:Algorithm} holds in practice. The scaling of {\MONKES} with the number of Legendre modes $N_\xi$ and the number of points in which the flux-surface is discretized is shown in figure \ref{fig:MONKES_Scaling}. To demonstrate the linear scaling, the wall-clock time as a function of $N_\xi$ for $N_{\text{fs}}=2025$ points is represented in figure \ref{subfig:MONKES_Scaling_Legendre} and compared with the line of slope 0.61 seconds per Legendre mode. As can be seen in figure \ref{subfig:MONKES_Scaling_Nfs}, the wall-clock time (per Legendre mode) scales cubicly with the number of points in which the flux-surface is discretized $N_{\text{fs}}$. As it was mentioned at the end of section \ref{sec:Algorithm}, the constant $C_{\text{alg}}$ in a single core is sufficiently small to give accurate calculations up to $\hat{\nu}\sim 10^{-5}$ $\text{m}^{-1}$. We have plotted in figure \ref{subfig:MONKES_Scaling_Nfs} the cubic fit $C_{\text{alg}}N_{\text{fs}}^3$, where $C_{\text{alg}}=0.61(1/2025)^3\sim 7\cdot 10^{-11}$ s. 

As the {\texttt{LAPACK}} library is multithreaded and allows to parallelize the linear algebra operations through several cores, the scaling of {\MONKES} when running in parallel is represented. Additionally, for the resolutions selected in subsection \ref{subsec:Convergence}, we display in table \ref{tab:MONKES_Times_Multicore} the wall-clock time when running {\MONKES} using several cores in parallel. Note that for the W7-X cases, which require a smaller value of $N_{\text{fs}}$, the speed-up stalls at 8 cores. For CIEMAT-QI, that requires discretizing the flux-surface on a finer mesh, this does not happen in the range of cores considered.


\begin{table}[h]
	\centering
	\begin{tabular}{lccccc}
		\toprule
		%$t_{\text{clock}}^{\MONKES}$ {[}s{]}
		\backslashbox{Case}{No. cores}   & 1  & 2  &  4 &  8 & 16\\ \midrule
		W7X-EIM $\hat{E}_r=0$        & 22    & 13  &  8   &  5 &  5 \\
		W7X-EIM $\hat{E}_r\ne 0$   & 40    & 20  & 12    & 8 &  6 \\ 
		W7X-KJM $\hat{E}_r=0$        & 33    & 17  &  12 & 7 &   7   \\
		W7X-KJM $\hat{E}_r\ne 0$   & 46    & 17  &  13 & 7 & 7  \\
		CIEMAT-QI $\hat{E}_r=0$      & 78    & 45 & 29 & 21 & 16 \\
		CIEMAT-QI $\hat{E}_r\ne 0$ & 78    & 45 & 29 & 21 & 16 \\\bottomrule
	\end{tabular}
	\caption{Wall-clock time of {\MONKES} in seconds for the selected triplets $(N_\theta,N_\zeta,N_\xi)$ when running in several cores.}
	\label{tab:MONKES_Times_Multicore}
\end{table}

\FloatBarrier